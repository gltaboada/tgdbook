\documentclass[]{book}
\usepackage{lmodern}
\usepackage{amssymb,amsmath}
\usepackage{ifxetex,ifluatex}
\usepackage{fixltx2e} % provides \textsubscript
\ifnum 0\ifxetex 1\fi\ifluatex 1\fi=0 % if pdftex
  \usepackage[T1]{fontenc}
  \usepackage[utf8]{inputenc}
\else % if luatex or xelatex
  \ifxetex
    \usepackage{mathspec}
  \else
    \usepackage{fontspec}
  \fi
  \defaultfontfeatures{Ligatures=TeX,Scale=MatchLowercase}
\fi
% use upquote if available, for straight quotes in verbatim environments
\IfFileExists{upquote.sty}{\usepackage{upquote}}{}
% use microtype if available
\IfFileExists{microtype.sty}{%
\usepackage{microtype}
\UseMicrotypeSet[protrusion]{basicmath} % disable protrusion for tt fonts
}{}
\usepackage[margin=1in]{geometry}
\usepackage{hyperref}
\hypersetup{unicode=true,
            pdftitle={Prácticas de Tecnologías de Gestión de Datos},
            pdfauthor={Guillermo López Taboada (guillermo.lopez.taboada@udc.es) y Rubén F. Casal (ruben.fcasal@udc.es)},
            pdfborder={0 0 0},
            breaklinks=true}
\urlstyle{same}  % don't use monospace font for urls
\usepackage{natbib}
\bibliographystyle{apalike}
\usepackage{color}
\usepackage{fancyvrb}
\newcommand{\VerbBar}{|}
\newcommand{\VERB}{\Verb[commandchars=\\\{\}]}
\DefineVerbatimEnvironment{Highlighting}{Verbatim}{commandchars=\\\{\}}
% Add ',fontsize=\small' for more characters per line
\usepackage{framed}
\definecolor{shadecolor}{RGB}{248,248,248}
\newenvironment{Shaded}{\begin{snugshade}}{\end{snugshade}}
\newcommand{\KeywordTok}[1]{\textcolor[rgb]{0.13,0.29,0.53}{\textbf{#1}}}
\newcommand{\DataTypeTok}[1]{\textcolor[rgb]{0.13,0.29,0.53}{#1}}
\newcommand{\DecValTok}[1]{\textcolor[rgb]{0.00,0.00,0.81}{#1}}
\newcommand{\BaseNTok}[1]{\textcolor[rgb]{0.00,0.00,0.81}{#1}}
\newcommand{\FloatTok}[1]{\textcolor[rgb]{0.00,0.00,0.81}{#1}}
\newcommand{\ConstantTok}[1]{\textcolor[rgb]{0.00,0.00,0.00}{#1}}
\newcommand{\CharTok}[1]{\textcolor[rgb]{0.31,0.60,0.02}{#1}}
\newcommand{\SpecialCharTok}[1]{\textcolor[rgb]{0.00,0.00,0.00}{#1}}
\newcommand{\StringTok}[1]{\textcolor[rgb]{0.31,0.60,0.02}{#1}}
\newcommand{\VerbatimStringTok}[1]{\textcolor[rgb]{0.31,0.60,0.02}{#1}}
\newcommand{\SpecialStringTok}[1]{\textcolor[rgb]{0.31,0.60,0.02}{#1}}
\newcommand{\ImportTok}[1]{#1}
\newcommand{\CommentTok}[1]{\textcolor[rgb]{0.56,0.35,0.01}{\textit{#1}}}
\newcommand{\DocumentationTok}[1]{\textcolor[rgb]{0.56,0.35,0.01}{\textbf{\textit{#1}}}}
\newcommand{\AnnotationTok}[1]{\textcolor[rgb]{0.56,0.35,0.01}{\textbf{\textit{#1}}}}
\newcommand{\CommentVarTok}[1]{\textcolor[rgb]{0.56,0.35,0.01}{\textbf{\textit{#1}}}}
\newcommand{\OtherTok}[1]{\textcolor[rgb]{0.56,0.35,0.01}{#1}}
\newcommand{\FunctionTok}[1]{\textcolor[rgb]{0.00,0.00,0.00}{#1}}
\newcommand{\VariableTok}[1]{\textcolor[rgb]{0.00,0.00,0.00}{#1}}
\newcommand{\ControlFlowTok}[1]{\textcolor[rgb]{0.13,0.29,0.53}{\textbf{#1}}}
\newcommand{\OperatorTok}[1]{\textcolor[rgb]{0.81,0.36,0.00}{\textbf{#1}}}
\newcommand{\BuiltInTok}[1]{#1}
\newcommand{\ExtensionTok}[1]{#1}
\newcommand{\PreprocessorTok}[1]{\textcolor[rgb]{0.56,0.35,0.01}{\textit{#1}}}
\newcommand{\AttributeTok}[1]{\textcolor[rgb]{0.77,0.63,0.00}{#1}}
\newcommand{\RegionMarkerTok}[1]{#1}
\newcommand{\InformationTok}[1]{\textcolor[rgb]{0.56,0.35,0.01}{\textbf{\textit{#1}}}}
\newcommand{\WarningTok}[1]{\textcolor[rgb]{0.56,0.35,0.01}{\textbf{\textit{#1}}}}
\newcommand{\AlertTok}[1]{\textcolor[rgb]{0.94,0.16,0.16}{#1}}
\newcommand{\ErrorTok}[1]{\textcolor[rgb]{0.64,0.00,0.00}{\textbf{#1}}}
\newcommand{\NormalTok}[1]{#1}
\usepackage{longtable,booktabs}
\usepackage{graphicx,grffile}
\makeatletter
\def\maxwidth{\ifdim\Gin@nat@width>\linewidth\linewidth\else\Gin@nat@width\fi}
\def\maxheight{\ifdim\Gin@nat@height>\textheight\textheight\else\Gin@nat@height\fi}
\makeatother
% Scale images if necessary, so that they will not overflow the page
% margins by default, and it is still possible to overwrite the defaults
% using explicit options in \includegraphics[width, height, ...]{}
\setkeys{Gin}{width=\maxwidth,height=\maxheight,keepaspectratio}
\IfFileExists{parskip.sty}{%
\usepackage{parskip}
}{% else
\setlength{\parindent}{0pt}
\setlength{\parskip}{6pt plus 2pt minus 1pt}
}
\setlength{\emergencystretch}{3em}  % prevent overfull lines
\providecommand{\tightlist}{%
  \setlength{\itemsep}{0pt}\setlength{\parskip}{0pt}}
\setcounter{secnumdepth}{5}
% Redefines (sub)paragraphs to behave more like sections
\ifx\paragraph\undefined\else
\let\oldparagraph\paragraph
\renewcommand{\paragraph}[1]{\oldparagraph{#1}\mbox{}}
\fi
\ifx\subparagraph\undefined\else
\let\oldsubparagraph\subparagraph
\renewcommand{\subparagraph}[1]{\oldsubparagraph{#1}\mbox{}}
\fi

%%% Use protect on footnotes to avoid problems with footnotes in titles
\let\rmarkdownfootnote\footnote%
\def\footnote{\protect\rmarkdownfootnote}

%%% Change title format to be more compact
\usepackage{titling}

% Create subtitle command for use in maketitle
\providecommand{\subtitle}[1]{
  \posttitle{
    \begin{center}\large#1\end{center}
    }
}

\setlength{\droptitle}{-2em}

  \title{Prácticas de Tecnologías de Gestión de Datos}
    \pretitle{\vspace{\droptitle}\centering\huge}
  \posttitle{\par}
    \author{Guillermo López Taboada
(\href{mailto:guillermo.lopez.taboada@udc.es}{\nolinkurl{guillermo.lopez.taboada@udc.es}})
y Rubén F. Casal
(\href{mailto:ruben.fcasal@udc.es}{\nolinkurl{ruben.fcasal@udc.es}})}
    \preauthor{\centering\large\emph}
  \postauthor{\par}
      \predate{\centering\large\emph}
  \postdate{\par}
    \date{2019-09-10}

\usepackage{booktabs}
\usepackage{amsthm}
%\usepackage{animate}
\ifxetex
  \usepackage{polyglossia}
  \setmainlanguage{spanish}
  % Tabla en lugar de cuadro
  \gappto\captionsspanish{\renewcommand{\tablename}{Tabla}
          \renewcommand{\listtablename}{Índice de tablas}}

\else
  \usepackage[spanish,es-tabla]{babel}
\fi
\makeatletter
\def\thm@space@setup{%
  \thm@preskip=8pt plus 2pt minus 4pt
  \thm@postskip=\thm@preskip
}
\makeatother

\begin{document}
\maketitle

{
\setcounter{tocdepth}{1}
\tableofcontents
}
\chapter*{Prólogo}\label{prologo}
\addcontentsline{toc}{chapter}{Prólogo}

Este libro contiene algunas de las prácticas de la asignatura de
\href{http://eamo.usc.es/pub/mte/index.php/es/?option=com_content\&view=article\&id=2202\&idm=38\&a\%C3\%B1o=2019}{Tecnologías
de Gestión de Datos} del \href{http://eio.usc.es/pub/mte}{Máster
interuniversitario en Técnicas Estadísticas}).

Este libro ha sido escrito en
\href{http://rmarkdown.rstudio.com}{R-Markdown} empleando el paquete
\href{https://bookdown.org/yihui/bookdown/}{\texttt{bookdown}} y está
disponible en el repositorio Github:
\href{https://github.com/gltaboada/tgdbook}{gltaboada/tgdbook}. Se puede
acceder a la versión en línea a través del siguiente enlace:

\url{https://gltaboada.github.io/tgdbook}.

donde puede descargarse en formato
\href{https://gltaboada.github.io/tgdbook/Practicas_de_TGD.pdf}{pdf}.

Para generar el libro (compilar) se recomendaría consultar el libro de
\href{https://rubenfcasal.github.io/bookdown_intro}{``Escritura de
libros con bookdown''} en castellano.

\includegraphics[width=1.22in]{images/by-nc-nd-88x31}

Este obra está bajo una licencia de
\href{https://creativecommons.org/licenses/by-nc-nd/4.0/deed.es_ES}{Creative
Commons Reconocimiento-NoComercial-SinObraDerivada 4.0 Internacional}
(esperamos poder liberarlo bajo una licencia menos restrictiva más
adelante\ldots{}).

\chapter{Introducción a las Tecnologías de Gestión de
Datos}\label{introduccion-a-las-tecnologias-de-gestion-de-datos}

La información relevante de la materia está disponible en la guía
docente y la ficha de la asignatura

En particular, los resultados de aprendizaje son:

\begin{itemize}
\item
  Manejar de forma autónoma y solvente el software necesario para
  acceder a conjuntos de datos en entornos profesionales y/o en la nube.
\item
  Saber gestionar conjuntos de datos masivos en un entorno
  multidisciplinar que permita la participación en proyectos
  profesionales complejos que requieran el uso de técnicas estadísticas.
\item
  Saber relacionar el software de diseño y gestión de bases de datos con
  el específicamente implementado para el análisis de datos.
\end{itemize}

\section{Contenidos}\label{contenidos}

\begin{enumerate}
\def\labelenumi{\arabic{enumi}.}
\tightlist
\item
  Introducción al lenguaje SQL

  \begin{itemize}
  \tightlist
  \item
    Bases de datos relacionales
  \item
    Sintaxis SQL
  \item
    Conexión con bases de datos desde R
  \end{itemize}
\item
  Introducción a tecnologías NoSQL

  \begin{itemize}
  \tightlist
  \item
    Conceptos y tipos de bases de datos NoSQL (documental, columnar,
    clave/valor y de grafos)
  \item
    Conexión de R a NoSQL
  \end{itemize}
\item
  Tecnologías para el tratamiento de datos masivos

  \begin{itemize}
  \tightlist
  \item
    Tecnologías Big Data (Hadoop, Spark, Hive, Rspark, Sparklyr)
  \item
    Visualización y generación de cuadros de mando
  \item
    Introducción al análisis de datos masivos.
  \end{itemize}
\end{enumerate}

\section{Planificación}\label{planificacion}

\begin{itemize}
\item
  Clase 1 (12/9): Seminario R: Manipulación de datos con el paquete
  básico de R
\item
  Clase 2 (19/9): Tema 1: Conceptos de bases de datos
\item
  Clase 3 (26/9): Tema 1: Introducción a SQL
\item
  Clase 4 (3/10): Seminario dplyr
\item
  Clase 5 (10/10): Tema 1: Ejercicios prácticos de Entidad-relación y
  SQL
\item
  Clase 6 (17/10): Tema 1: Continuación de ejercicios prácticos SQL
\item
  Clase 7 (24/10): Tema 2: Introducción a NoSQL
\item
  Clase 8 (31/10): Tema 2: Ejercicios prácticos de NoSQL
\item
  Clase 9 (7/11): Tema 3: Ecosistema Big Data (Hadoop, Spark)
\item
  Clase 10 (14/11): Tema 3: Tecnologías Big Data (Rspark/sparklyr)
\item
  Clase 11 (21/11): Seminario visualización con power BI
\item
  Clase 12 (28/11): Seminario machine learning CESGA/localhost
\item
  Clase 13 (5/12): Tema 3: Introducción al análisis de datos masivos
\item
  Clase 14 (12/12): Seminario conceptos avanzados de R
\item
  Clase 15 (19/12): Seminario aplicaciones Big Data en investigación e
  industria
\end{itemize}

\subsection{Evaluación}\label{evaluacion}

\begin{itemize}
\item
  \textbf{Examen} (60\%): El examen de la materia evaluará los
  siguientes aspectos: Conceptos de la materia: Dominio de los
  conocimientos teóricos y operativos de la materia. Asimilación
  práctica de materia: Asimilación y comprensión de los conocimientos
  teóricos y operativos de la materia.
\item
  \textbf{Prácticas de laboratorio} (30\%): Evaluación de las prácticas
  de laboratorio desarrolladas por los estudiantes.
\item
  \textbf{Trabajos tutelados} (10\%): Evaluación de los trabajos
  tutelados desarrollados por los estudiantes.
\end{itemize}

\subsubsection{Observaciones sobre la
evaluación:}\label{observaciones-sobre-la-evaluacion}

\begin{itemize}
\item
  Las prácticas de laboratorio se realizarán en grupos de 2.
\item
  Para poder aprobar la asignatura en la primera oportunidad será
  necesario obtener como mínimo el 30\% de la nota máxima de la suma de
  las prácticas de laboratorio y trabajos tutelados e, igualmente, el
  30\% de la nota máxima final de la Prueba mixta (examen), y tener una
  nota total (prácticas más trabajos tutelados más prueba mixta) igual o
  superior al 50\% de la nota máxima.
\item
  En la segunda oportunidad solamente se podrá recuperar la nota del
  examen. Las notas de prácticas y de trabajos tutelados serán las
  obtenidas durante el curso. Para los alumnos que utilicen la
  oportunidad adelantada de diciembre se utilizarán las notas de
  prácticas y trabajos tutelados que obtuvieran en su último curso. En
  esta oportunidad solo será necesario para aprobar obtener una nota
  total igual o superior al 50\% de la nota máxima.
\item
  Una vez que un estudiante es evaluado en una práctica de laboratorio o
  en un trabajo tutelado implica que será calificado. Por tanto, la
  calificación ``No Presentado'' no es posible una vez que una
  práctica/trabajo ha sido evaluada.
\end{itemize}

\section{Fuentes de información:}\label{fuentes-de-informacion}

\subsection{Básica}\label{basica}

\begin{itemize}
\tightlist
\item
  Daroczi, G. (2015). Mastering Data Analysis with R. Packt Publishing
\item
  Grolemund, G. y Wickham, H. (2016). R for Data Science.
  \url{https://r4ds.had.co.nz/} \& O'Reilly
\item
  Silberschatz, A., Korth, H. y Sudarshan, S. (2014). Fundamentos de
  Bases de Datos. Mc Graw Hill
\item
  Rubén Fernández Casal (2019). Ayuda y Recursos para el Aprendizaje de
  R.
  \url{https://rubenfcasal.github.io/post/ayuda-y-recursos-para-el-aprendizaje-de-r/}
\end{itemize}

\subsection{Complementaria:}\label{complementaria}

\begin{itemize}
\tightlist
\item
  Wes McKinney (2017). Python for Data Analysis: Data Wrangling with
  Pandas, NumPy, and IPython. O'Reilly (2ª ed.)
\item
  Tom White (2015). Hadoop: The Definitive Guide. O'Reilly (4ª ed.)
\item
  Alex Holmes (2014). Hadoop in practice. Manning (2ª ed.)
\item
  Centro de Supercomputación de Galicia (2019). Servicio de Big Data del
  CESGA. \url{https://bigdata.cesga.es/}
\end{itemize}

\chapter{Manipulación de datos con R}\label{manipulacion-de-datos-con-r}

\section{Lectura, importación y exportación de
datos}\label{lectura-importacion-y-exportacion-de-datos}

La mayoría de los estudios estadísticos requieren disponer de un
conjunto de datos. Además de la introducción directa, \texttt{R} es
capaz de importar datos externos en múltiples formatos:

\begin{itemize}
\item
  bases de datos disponibles en librerías de \texttt{R}
\item
  archivos de texto en formato ASCII
\item
  archivos en otros formatos: Excel, SPSS, \ldots{}
\item
  bases de datos relacionales: MySQL, Oracle, \ldots{}
\item
  formatos web: HTML, XML, JSON, \ldots{}
\item
  \ldots{}.
\end{itemize}

\subsection{Acceso a datos en
paquetes}\label{acceso-a-datos-en-paquetes}

\texttt{R} dispone de múltiples conjuntos de datos en distintos
paquetes, especialmente en el paquete \texttt{datasets} que se carga por
defecto al abrir \texttt{R}. Con el comando \texttt{data()} podemos
obtener un listado de las bases de datos disponibles.

Para cargar una base de datos concreta se utiliza el comando
\texttt{data(nombre)} (aunque en algunos casos se cargan automáticamente
al emplearlos). Por ejemplo, \texttt{data(cars)} carga la base de datos
llamada \texttt{cars} en el entorno de trabajo (\texttt{".GlobalEnv"}) y
\texttt{?cars} muestra la ayuda correspondiente con la descripición de
la base de datos.

\subsection{Lectura de archivos de
texto}\label{lectura-de-archivos-de-texto}

En \texttt{R} para leer archivos de texto se suele utilizar la función
\texttt{read.table()}. Supóngase, por ejemplo, que en el directorio
actual está el fichero \emph{empleados.txt}. La lectura de este fichero
vendría dada por el código:

\begin{Shaded}
\begin{Highlighting}[]
\NormalTok{datos <-}\StringTok{ }\KeywordTok{read.table}\NormalTok{(}\DataTypeTok{file =} \StringTok{"empleados.txt"}\NormalTok{, }\DataTypeTok{header =} \OtherTok{TRUE}\NormalTok{)}
\KeywordTok{head}\NormalTok{(datos)}
\end{Highlighting}
\end{Shaded}

\begin{verbatim}
##   id   sexo   fechnac educ         catlab salario salini tiempemp expprev
## 1  1 Hombre  2/3/1952   15      Directivo   57000  27000       98     144
## 2  2 Hombre 5/23/1958   16 Administrativo   40200  18750       98      36
## 3  3  Mujer 7/26/1929   12 Administrativo   21450  12000       98     381
## 4  4  Mujer 4/15/1947    8 Administrativo   21900  13200       98     190
## 5  5 Hombre  2/9/1955   15 Administrativo   45000  21000       98     138
## 6  6 Hombre 8/22/1958   15 Administrativo   32100  13500       98      67
##   minoria
## 1      No
## 2      No
## 3      No
## 4      No
## 5      No
## 6      No
\end{verbatim}

Si el fichero estuviese en el directorio \emph{c:\textbackslash{}datos}
bastaría con especificar \texttt{file\ =\ "c:/datos/empleados.txt"}.
Nótese también que para la lectura del fichero anterior se ha
establecido el argumento \texttt{header=TRUE} para indicar que la
primera línea del fichero contiene los nombres de las variables.

Los argumentos utilizados habitualmente para esta función son:

\begin{itemize}
\item
  \texttt{header}: indica si el fichero tiene cabecera
  (\texttt{header=TRUE}) o no (\texttt{header=FALSE}). Por defecto toma
  el valor \texttt{header=FALSE}.
\item
  \texttt{sep}: carácter separador de columnas que por defecto es un
  espacio en blanco (\texttt{sep=""}). Otras opciones serían:
  \texttt{sep=","} si el separador es un ``;'', \texttt{sep="*"} si el
  separador es un ``*'', etc.
\item
  \texttt{dec}: carácter utilizado en el fichero para los números
  decimales. Por defecto se establece \texttt{dec\ =\ "."}. Si los
  decimales vienen dados por ``,'' se utiliza \texttt{dec\ =\ ","}
\end{itemize}

Resumiendo, los (principales) argumentos por defecto de la función
\texttt{read.table} son los que se muestran en la siguiente línea:

\begin{Shaded}
\begin{Highlighting}[]
\KeywordTok{read.table}\NormalTok{(file, }\DataTypeTok{header =} \OtherTok{FALSE}\NormalTok{, }\DataTypeTok{sep =} \StringTok{""}\NormalTok{, }\DataTypeTok{dec =} \StringTok{"."}\NormalTok{)  }
\end{Highlighting}
\end{Shaded}

Para más detalles sobre esta función véase \texttt{help(read.table)}.

Estan disponibles otras funciones con valores por defecto de los
parámetros adecuados para otras situaciones. Por ejemplo, para ficheros
separados por tabuladores se puede utilizar \texttt{read.delim()} o
\texttt{read.delim2()}:

\begin{Shaded}
\begin{Highlighting}[]
\KeywordTok{read.delim}\NormalTok{(file, }\DataTypeTok{header =} \OtherTok{TRUE}\NormalTok{, }\DataTypeTok{sep =} \StringTok{"}\CharTok{\textbackslash{}t}\StringTok{"}\NormalTok{, }\DataTypeTok{dec =} \StringTok{"."}\NormalTok{)}
\KeywordTok{read.delim2}\NormalTok{(file, }\DataTypeTok{header =} \OtherTok{TRUE}\NormalTok{, }\DataTypeTok{sep =} \StringTok{"}\CharTok{\textbackslash{}t}\StringTok{"}\NormalTok{, }\DataTypeTok{dec =} \StringTok{","}\NormalTok{)}
\end{Highlighting}
\end{Shaded}

\subsection{Importación desde SPSS}\label{importacion-desde-spss}

El programa \texttt{R} permite lectura de ficheros de datos en formato
SPSS (extensión \emph{.sav}) sin necesidad de tener instalado dicho
programa en el ordenador. Para ello se necesita:

\begin{itemize}
\item
  cargar la librería \texttt{foreign}
\item
  utilizar la función \texttt{read.spss}
\end{itemize}

Por ejemplo:

\begin{Shaded}
\begin{Highlighting}[]
\KeywordTok{library}\NormalTok{(foreign)}
\NormalTok{datos <-}\StringTok{ }\KeywordTok{read.spss}\NormalTok{(}\DataTypeTok{file =} \StringTok{"Employee data.sav"}\NormalTok{, }\DataTypeTok{to.data.frame =} \OtherTok{TRUE}\NormalTok{)}
\end{Highlighting}
\end{Shaded}

\begin{verbatim}
## re-encoding from CP1252
\end{verbatim}

\begin{Shaded}
\begin{Highlighting}[]
\KeywordTok{head}\NormalTok{(datos)}
\end{Highlighting}
\end{Shaded}

\begin{verbatim}
##   id   sexo     fechnac educ         catlab salario salini tiempemp
## 1  1 Hombre 11654150400   15      Directivo   57000  27000       98
## 2  2 Hombre 11852956800   16 Administrativo   40200  18750       98
## 3  3  Mujer 10943337600   12 Administrativo   21450  12000       98
## 4  4  Mujer 11502518400    8 Administrativo   21900  13200       98
## 5  5 Hombre 11749363200   15 Administrativo   45000  21000       98
## 6  6 Hombre 11860819200   15 Administrativo   32100  13500       98
##   expprev minoria
## 1     144      No
## 2      36      No
## 3     381      No
## 4     190      No
## 5     138      No
## 6      67      No
\end{verbatim}

\textbf{Nota}: Si hay fechas, puede ser recomendable emplear la función
\texttt{spss.get()} del paquete \texttt{Hmisc}.

\subsection{Importación desde Excel}\label{importacion-desde-excel}

Se pueden leer fichero de Excel (con extensión \emph{.xls}) utilizando
por ejemplo funciones de las librerías \texttt{openxlsx},
\texttt{XLConnect} o \texttt{RODBC}. Sin embargo, puede ser recomendable
exportar los datos desde Excel a un archivo de texto separado por
tabuladores (extensión \emph{.csv}).

Por ejemplo, supongamos que queremos leer el fichero \emph{coches.xls}:

\begin{itemize}
\item
  Desde Excel se selecciona el menú
  \texttt{Archivo\ -\textgreater{}\ Guardar\ como\ -\textgreater{}\ Guardar\ como}
  y en \texttt{Tipo} se escoge la opción de archivo CSV. De esta forma
  se guardarán los datos en el archivo \emph{coches.csv}.
\item
  El fichero \emph{coches.csv} es un fichero de texto plano (se puede
  editar con Notepad), con cabecera, las columnas separadas por ``;'', y
  siendo ``,'' el carácter decimal.
\item
  Por lo tanto, la lectura de este fichero se puede hacer con:

\begin{Shaded}
\begin{Highlighting}[]
\NormalTok{datos <-}\StringTok{ }\KeywordTok{read.table}\NormalTok{(}\StringTok{"coches.csv"}\NormalTok{, }\DataTypeTok{header =} \OtherTok{TRUE}\NormalTok{, }\DataTypeTok{sep =} \StringTok{";"}\NormalTok{, }\DataTypeTok{dec =} \StringTok{","}\NormalTok{)}
\end{Highlighting}
\end{Shaded}
\end{itemize}

Otra posibilidad es utilizar la función \texttt{read.csv2}, que es una
adaptación de la función general \texttt{read.table} con las siguientes
opciones:

\begin{Shaded}
\begin{Highlighting}[]
\KeywordTok{read.csv2}\NormalTok{(file, }\DataTypeTok{header =} \OtherTok{TRUE}\NormalTok{, }\DataTypeTok{sep =} \StringTok{";"}\NormalTok{, }\DataTypeTok{dec =} \StringTok{","}\NormalTok{)}
\end{Highlighting}
\end{Shaded}

Por lo tanto, la lectura del fichero \emph{coches.csv} se puede hacer de
modo más directo con:

\begin{Shaded}
\begin{Highlighting}[]
\NormalTok{datos <-}\StringTok{ }\KeywordTok{read.csv2}\NormalTok{(}\StringTok{"coches.csv"}\NormalTok{)}
\end{Highlighting}
\end{Shaded}

\subsection{Exportación de datos}\label{exportacion-de-datos}

Es evidente que además de la lectura e importación de datos también será
de gran interés la exportación de datos para que puedan leídos con otros
programas. Para ello, es de gran utilidad la función
\texttt{write.table}. Esta función es similar, pero funcionando en
sentido inverso, a la ya vista \texttt{read.table}.

Veamos un ejemplo:

\begin{Shaded}
\begin{Highlighting}[]
\NormalTok{tipo <-}\StringTok{ }\KeywordTok{c}\NormalTok{(}\StringTok{"A"}\NormalTok{, }\StringTok{"B"}\NormalTok{, }\StringTok{"C"}\NormalTok{)}
\NormalTok{longitud <-}\StringTok{ }\KeywordTok{c}\NormalTok{(}\FloatTok{120.34}\NormalTok{, }\FloatTok{99.45}\NormalTok{, }\FloatTok{115.67}\NormalTok{)}
\NormalTok{datos <-}\StringTok{ }\KeywordTok{data.frame}\NormalTok{(tipo, longitud)}
\NormalTok{datos}
\end{Highlighting}
\end{Shaded}

\begin{verbatim}
##   tipo longitud
## 1    A   120.34
## 2    B    99.45
## 3    C   115.67
\end{verbatim}

Para guardar el data.frame \texttt{datos} en un fichero de texto se
puede utilizar:

\begin{Shaded}
\begin{Highlighting}[]
\KeywordTok{write.table}\NormalTok{(datos, }\DataTypeTok{file =} \StringTok{"datos.txt"}\NormalTok{)}
\end{Highlighting}
\end{Shaded}

Otra posibilidad es utilizar la función:

\begin{Shaded}
\begin{Highlighting}[]
\KeywordTok{write.csv2}\NormalTok{(datos, }\DataTypeTok{file =} \StringTok{"datos.csv"}\NormalTok{)}
\end{Highlighting}
\end{Shaded}

que dará lugar al fichero \emph{datos.csv} importable directamente desde
Excel.

\section{Manipulación de datos}\label{manipulacion-de-datos}

Una vez cargada una (o varias) bases de datos hay una series de
operaciones que serán de interés para el tratamiento de datos:

\begin{itemize}
\tightlist
\item
  Operaciones con variables:

  \begin{itemize}
  \tightlist
  \item
    creación
  \item
    recodificación (e.g.~categorización)
  \item
    \ldots{}
  \end{itemize}
\item
  Operaciones con casos:

  \begin{itemize}
  \tightlist
  \item
    ordenación
  \item
    filtrado de datos
  \item
    \ldots{}
  \end{itemize}
\end{itemize}

A continuación se tratan las operaciones más básicas.

\subsection{Operaciones con variables}\label{operaciones-con-variables}

\subsubsection{Creación (y eliminación) de
variables}\label{creacion-y-eliminacion-de-variables}

Consideremos de nuevo la base de datos \texttt{cars} incluida en el
paquete \texttt{datasets}:

\begin{Shaded}
\begin{Highlighting}[]
\KeywordTok{data}\NormalTok{(cars)}
\KeywordTok{head}\NormalTok{(cars)}
\end{Highlighting}
\end{Shaded}

\begin{verbatim}
##   speed dist
## 1     4    2
## 2     4   10
## 3     7    4
## 4     7   22
## 5     8   16
## 6     9   10
\end{verbatim}

Utilizando el comando \texttt{help(cars)} se obtiene que \texttt{cars}
es un data.frame con 50 observaciones y dos variables:

\begin{itemize}
\item
  \texttt{speed}: Velocidad (millas por hora)
\item
  \texttt{dist}: tiempo hasta detenerse (pies)
\end{itemize}

Recordemos que, para acceder a la variable \texttt{speed} se puede hacer
directamente con su nombre o bien utilizando notación ``matricial''.

\begin{Shaded}
\begin{Highlighting}[]
\NormalTok{cars}\OperatorTok{$}\NormalTok{speed}
\end{Highlighting}
\end{Shaded}

\begin{verbatim}
##  [1]  4  4  7  7  8  9 10 10 10 11 11 12 12 12 12 13 13 13 13 14 14 14 14
## [24] 15 15 15 16 16 17 17 17 18 18 18 18 19 19 19 20 20 20 20 20 22 23 24
## [47] 24 24 24 25
\end{verbatim}

\begin{Shaded}
\begin{Highlighting}[]
\NormalTok{cars[, }\DecValTok{1}\NormalTok{]  }\CommentTok{# Equivalente}
\end{Highlighting}
\end{Shaded}

\begin{verbatim}
##  [1]  4  4  7  7  8  9 10 10 10 11 11 12 12 12 12 13 13 13 13 14 14 14 14
## [24] 15 15 15 16 16 17 17 17 18 18 18 18 19 19 19 20 20 20 20 20 22 23 24
## [47] 24 24 24 25
\end{verbatim}

Supongamos ahora que queremos transformar la variable original
\texttt{speed} (millas por hora) en una nueva variable
\texttt{velocidad} (kilómetros por hora) y añadir esta nueva variable al
data.frame \texttt{cars}. La transformación que permite pasar millas a
kilómetros es \texttt{kilómetros=millas/0.62137} que en \texttt{R} se
hace directamente con:

\begin{Shaded}
\begin{Highlighting}[]
\NormalTok{cars}\OperatorTok{$}\NormalTok{speed}\OperatorTok{/}\FloatTok{0.62137}
\end{Highlighting}
\end{Shaded}

Finalmente, incluimos la nueva variable que llamaremos
\texttt{velocidad} en \texttt{cars}:

\begin{Shaded}
\begin{Highlighting}[]
\NormalTok{cars}\OperatorTok{$}\NormalTok{velocidad <-}\StringTok{ }\NormalTok{cars}\OperatorTok{$}\NormalTok{speed }\OperatorTok{/}\StringTok{ }\FloatTok{0.62137}
\KeywordTok{head}\NormalTok{(cars)}
\end{Highlighting}
\end{Shaded}

\begin{verbatim}
##   speed dist velocidad
## 1     4    2  6.437388
## 2     4   10  6.437388
## 3     7    4 11.265430
## 4     7   22 11.265430
## 5     8   16 12.874777
## 6     9   10 14.484124
\end{verbatim}

También transformaremos la variable \texttt{dist} (en pies) en una nueva
variable \texttt{distancia} (en metros). Ahora la transformación deseada
es \texttt{metros=pies/3.2808}:

\begin{Shaded}
\begin{Highlighting}[]
\NormalTok{cars}\OperatorTok{$}\NormalTok{distancia <-}\StringTok{ }\NormalTok{cars}\OperatorTok{$}\NormalTok{dis }\OperatorTok{/}\StringTok{ }\FloatTok{3.2808}
\KeywordTok{head}\NormalTok{(cars)}
\end{Highlighting}
\end{Shaded}

\begin{verbatim}
##   speed dist velocidad distancia
## 1     4    2  6.437388 0.6096074
## 2     4   10  6.437388 3.0480371
## 3     7    4 11.265430 1.2192148
## 4     7   22 11.265430 6.7056815
## 5     8   16 12.874777 4.8768593
## 6     9   10 14.484124 3.0480371
\end{verbatim}

Ahora, eliminaremos las variables originales \texttt{speed} y
\texttt{dist}, y guardaremos el data.frame resultante con el nombre
\texttt{coches}. En primer lugar, veamos varias formas de acceder a las
variables de interés:

\begin{Shaded}
\begin{Highlighting}[]
\NormalTok{cars[, }\KeywordTok{c}\NormalTok{(}\DecValTok{3}\NormalTok{, }\DecValTok{4}\NormalTok{)]}
\NormalTok{cars[, }\KeywordTok{c}\NormalTok{(}\StringTok{"velocidad"}\NormalTok{, }\StringTok{"distancia"}\NormalTok{)]}
\NormalTok{cars[, }\OperatorTok{-}\KeywordTok{c}\NormalTok{(}\DecValTok{1}\NormalTok{, }\DecValTok{2}\NormalTok{)]}
\end{Highlighting}
\end{Shaded}

Utilizando alguna de las opciones anteriores se obtiene el
\texttt{data.frame} deseado:

\begin{Shaded}
\begin{Highlighting}[]
\NormalTok{coches <-}\StringTok{ }\NormalTok{cars[, }\KeywordTok{c}\NormalTok{(}\StringTok{"velocidad"}\NormalTok{, }\StringTok{"distancia"}\NormalTok{)]}
\KeywordTok{head}\NormalTok{(coches)}
\end{Highlighting}
\end{Shaded}

\begin{verbatim}
##   velocidad distancia
## 1  6.437388 0.6096074
## 2  6.437388 3.0480371
## 3 11.265430 1.2192148
## 4 11.265430 6.7056815
## 5 12.874777 4.8768593
## 6 14.484124 3.0480371
\end{verbatim}

Finalmente los datos anteriores podrían ser guardados en un fichero
exportable a Excel con el siguiente comando:

\begin{Shaded}
\begin{Highlighting}[]
\KeywordTok{write.csv2}\NormalTok{(coches, }\DataTypeTok{file =} \StringTok{"coches.csv"}\NormalTok{)}
\end{Highlighting}
\end{Shaded}

\subsection{Operaciones con casos}\label{operaciones-con-casos}

\subsubsection{Ordenación}\label{ordenacion}

Continuemos con el data.frame \texttt{cars}. Se puede comprobar que los
datos disponibles están ordenados por los valores de \texttt{speed}. A
continuación haremos la ordenación utilizando los valores de
\texttt{dist}. Para ello utilizaremos el conocido como vector de índices
de ordenación. Este vector establece el orden en que tienen que ser
elegidos los elementos para obtener la ordenación deseada. Veamos un
ejemplo sencillo:

\begin{Shaded}
\begin{Highlighting}[]
\NormalTok{x <-}\StringTok{ }\KeywordTok{c}\NormalTok{(}\FloatTok{2.5}\NormalTok{, }\FloatTok{4.3}\NormalTok{, }\FloatTok{1.2}\NormalTok{, }\FloatTok{3.1}\NormalTok{, }\FloatTok{5.0}\NormalTok{) }\CommentTok{# valores originales}
\NormalTok{ii <-}\StringTok{ }\KeywordTok{order}\NormalTok{(x)}
\NormalTok{ii    }\CommentTok{# vector de ordenación}
\end{Highlighting}
\end{Shaded}

\begin{verbatim}
## [1] 3 1 4 2 5
\end{verbatim}

\begin{Shaded}
\begin{Highlighting}[]
\NormalTok{x[ii] }\CommentTok{# valores ordenados}
\end{Highlighting}
\end{Shaded}

\begin{verbatim}
## [1] 1.2 2.5 3.1 4.3 5.0
\end{verbatim}

En el caso de vectores, el procedimiento anterior se podría hacer
directamente con:

\begin{Shaded}
\begin{Highlighting}[]
\KeywordTok{sort}\NormalTok{(x)}
\end{Highlighting}
\end{Shaded}

Sin embargo, para ordenar data.frames será necesario la utilización del
vector de índices de ordenación. A continuación, los datos de
\texttt{cars} ordenados por \texttt{dist}:

\begin{Shaded}
\begin{Highlighting}[]
\NormalTok{ii <-}\StringTok{ }\KeywordTok{order}\NormalTok{(cars}\OperatorTok{$}\NormalTok{dist) }\CommentTok{# Vector de índices de ordenación}
\NormalTok{cars2 <-}\StringTok{ }\NormalTok{cars[ii, ]    }\CommentTok{# Datos ordenados por dist}
\KeywordTok{head}\NormalTok{(cars2)}
\end{Highlighting}
\end{Shaded}

\begin{verbatim}
##    speed dist velocidad distancia
## 1      4    2  6.437388 0.6096074
## 3      7    4 11.265430 1.2192148
## 2      4   10  6.437388 3.0480371
## 6      9   10 14.484124 3.0480371
## 12    12   14 19.312165 4.2672519
## 5      8   16 12.874777 4.8768593
\end{verbatim}

\subsubsection{Filtrado}\label{filtrado}

El filtrado de datos consiste en elegir una submuestra que cumpla
determinadas condiciones. Para ello se puede utilizar la función
\texttt{subset}. A continuación se muestran un par de ejemplos:

\begin{Shaded}
\begin{Highlighting}[]
\KeywordTok{subset}\NormalTok{(cars, dist }\OperatorTok{>}\StringTok{ }\DecValTok{85}\NormalTok{) }\CommentTok{# datos con dis>85}
\end{Highlighting}
\end{Shaded}

\begin{verbatim}
##    speed dist velocidad distancia
## 47    24   92  38.62433  28.04194
## 48    24   93  38.62433  28.34674
## 49    24  120  38.62433  36.57644
\end{verbatim}

\begin{Shaded}
\begin{Highlighting}[]
\KeywordTok{subset}\NormalTok{(cars, speed }\OperatorTok{>}\StringTok{ }\DecValTok{10} \OperatorTok{&}\StringTok{ }\NormalTok{speed }\OperatorTok{<}\StringTok{ }\DecValTok{15} \OperatorTok{&}\StringTok{ }\NormalTok{dist }\OperatorTok{>}\StringTok{ }\DecValTok{45}\NormalTok{) }\CommentTok{# speed en (10,15) y dist>45}
\end{Highlighting}
\end{Shaded}

\begin{verbatim}
##    speed dist velocidad distancia
## 19    13   46  20.92151  14.02097
## 22    14   60  22.53086  18.28822
## 23    14   80  22.53086  24.38430
\end{verbatim}

También se pueden hacer el filtrado empleando directamente los
correspondientes vectores de índices:

\begin{Shaded}
\begin{Highlighting}[]
\NormalTok{ii <-}\StringTok{ }\NormalTok{cars}\OperatorTok{$}\NormalTok{dist }\OperatorTok{>}\StringTok{ }\DecValTok{85}
\NormalTok{cars[ii, ]   }\CommentTok{# dis>85}
\end{Highlighting}
\end{Shaded}

\begin{verbatim}
##    speed dist velocidad distancia
## 47    24   92  38.62433  28.04194
## 48    24   93  38.62433  28.34674
## 49    24  120  38.62433  36.57644
\end{verbatim}

\begin{Shaded}
\begin{Highlighting}[]
\NormalTok{ii <-}\StringTok{ }\NormalTok{cars}\OperatorTok{$}\NormalTok{speed }\OperatorTok{>}\StringTok{ }\DecValTok{10} \OperatorTok{&}\StringTok{ }\NormalTok{cars}\OperatorTok{$}\NormalTok{speed }\OperatorTok{<}\StringTok{ }\DecValTok{15} \OperatorTok{&}\StringTok{ }\NormalTok{cars}\OperatorTok{$}\NormalTok{dist }\OperatorTok{>}\StringTok{ }\DecValTok{45}
\NormalTok{cars[ii, ]  }\CommentTok{# speed en (10,15) y dist>45}
\end{Highlighting}
\end{Shaded}

\begin{verbatim}
##    speed dist velocidad distancia
## 19    13   46  20.92151  14.02097
## 22    14   60  22.53086  18.28822
## 23    14   80  22.53086  24.38430
\end{verbatim}

\chapter{Introducción al lenguaje
SQL}\label{introduccion-al-lenguaje-sql}

Los sistemas de información gestionan repositorios de información en
múltiples formatos, siendo el más popular las bases de datos
relacionales a las que se accede mediante SQL (Structured Query
Language)

\section{Bases de Datos Relacionales}\label{bases-de-datos-relacionales}

\subsection{Definiciones}\label{definiciones}

\begin{itemize}
\item
  \textbf{Dominio}: contexto (organización, empresa, evento\ldots{})
  objeto de gestión de la información.
\item
  \textbf{Dato}: hecho con significado implícito, registable, relevante
  en un determinado dominio.
\item
  \textbf{Base de datos}: colección de datos de un determinado dominio
  relacionados entre sí, organizados de forma que sea posible
  manipularlos y recuperarlos de forma eficiente.
\item
  Sistema de Gestión de Bases de Datos (\textbf{SGBD}) (en inglés
  \textbf{RDBMS}, Relational Database Management System): software que
  permite a los usuarios crear y manipular bases de datos mediante
  operaciones CRUD:

  \begin{itemize}
  \tightlist
  \item
    Crear / Insertar Datos (Create)
  \item
    Consultar / Leer (Read)
  \item
    Actualizar / Modificar (Update)
  \item
    Eliminar (Delete)
  \end{itemize}
\end{itemize}

\begin{center}\rule{0.5\linewidth}{\linethickness}\end{center}

\begin{itemize}
\item
  \textbf{Modelo de datos}: abstracción conceptual que propone una
  manera de organizar y manipular los datos. Definido mediante:

  \begin{itemize}
  \tightlist
  \item
    Estructura: elementos para organizar datos
  \item
    Integridad: reglas para relaciones los elementos
  \item
    Manipulación: operaciones sobre los datos adaptadas a la estructura
    y reglas
  \end{itemize}
\item
  Modelo de datos conceptual \textbf{Entidad Relación} (entidades,
  relaciones, atributos)
\item
  Modelo de datos lógico o de representación (\textbf{modelo relacional}
  de Codd)

  \begin{itemize}
  \tightlist
  \item
    Datos en relaciones (tablas)
  \item
    Base matemática formal
  \item
    Flexible
  \end{itemize}
\item
  Modelo de datos físico (tal y como se almacenan los datos)
\end{itemize}

Una fila de la tabla (relación) es una tupla y una columna un atributo
(ver Figura \ref{fig:relacion}).

(ver Figura \ref{fig:relacion})

(ver Figura \ref{fig:relacion})

\begin{figure}[!htb]

{\centering \includegraphics[width=0.7\linewidth]{images/Relacion} 

}

\caption{Esquema de una relación.}\label{fig:relacion}
\end{figure}

Una base de datos es un conjunto de tablas (al menos una).

\begin{figure}
\centering
\includegraphics[width=6.25000in]{images/BBDD.png}
\caption{}
\end{figure}

La tabla no es una relación porque la relación es un conjunto sin orden
y una tabla puede tener filas repetidas y tiene orden.

\begin{center}\rule{0.5\linewidth}{\linethickness}\end{center}

\begin{itemize}
\item
  \textbf{Esquema}: estructura de la base de datos
\item
  \textbf{Estado}: contenido de la base de datos
\item
  Restricción de \textbf{integridad}: regla que debe cumplir la
  información registrada en la base de datos para garantizar la
  integridad de la información.
\end{itemize}

Cualquier Base de Datos basada en el modelo relacional ha de cumplir
como mínimo estas restricciones (además de las propias del dominio):

\begin{itemize}
\item
  Restricción de dominio: el valor de cada atributo debe de ser único
  (teléfono, no valor único), no descomponible (nombre completo
  descomponible en nombre y apellidos, domicilio en calle, CP,
  localidad, etc\ldots{})
\item
  Una relación es un conjunto de tuplas, por tanto todas las tuplas son
  distintas.
\item
  Una \textbf{superclave} es un subconjunto de atributos tal que no
  existen dos tuplas con la misma superclave.
\end{itemize}

\begin{quote}
Ejercicio. En la relación Empleado(dni, nombre, apellidos, email)
¿cuántas superclaves existen?
\end{quote}

\begin{itemize}
\tightlist
\item
  Una \textbf{clave candidata} es una superclave mínima (superclave
  mínima es la clave a la que no se le puede eliminar un atributo).
\end{itemize}

\begin{quote}
¿Cuántas claves candidatas hay en el ejemplo anterior?
\end{quote}

\begin{itemize}
\item
  \textbf{Clave primaria} es la clave candidata que elegimos que
  identificar de forma unívoca las tuplas de una relación. Restricción
  de integridad de entidad: Ningún valor de la clave primaria puede ser
  un valor nulo.
\item
  \textbf{Clave foránea} es un conjunto de atributos de una relación
  R\_1 que, para cada tupla, identifican a otra tupla de una relación
  R\_2 con la que está relacionada. La Restricción de integridad
  referencial nos dice que la clave foránea ha de corresponderse con la
  clave primaria de R\_2, y si la clave foránea no es nula ha de refir a
  una tupla en R\_2.
\end{itemize}

\begin{figure}
\centering
\includegraphics[width=6.25000in]{images/ClaveForanea.png}
\caption{}
\end{figure}

\begin{figure}
\centering
\includegraphics[width=6.25000in]{images/IntegridadReferencial.png}
\caption{}
\end{figure}

Si borramos/actualizamos un valor de clave foránea podemos: (a) prohibir
el cambio, o (b) poner a nulo la clave foránea (borrado) o propogar el
cambio (modificación).

\begin{center}\rule{0.5\linewidth}{\linethickness}\end{center}

\begin{itemize}
\tightlist
\item
  Ventajas de SGBD:

  \begin{itemize}
  \tightlist
  \item
    Administración centralizada de los datos (por un administrador en un
    servidor/plataforma central que evita la información en silos
    -redundante/inconsistente)
  \item
    Desacoplado del almacenamiento físico de los datos (no es necesario
    conocerlo)
  \item
    Simplicidad de acceso (ODBC + SQL, lenguaje declarativo)
  \item
    Control de integridad (restricciones genéricas, integridad de
    entidad y referencial, de dominio, y las del dominio en software)
  \item
    Control de acceso concurrente (evita inconsistencia)
  \item
    Seguridad (autenticación, roles de acceso)
  \item
    Recuperación ante fallos (backup, logs y transacciones -rollback-)
  \end{itemize}
\end{itemize}

\section{Sintaxis SQL}\label{sintaxis-sql}

A continuación 27 clásulas SQL básicas

\subsection{Extracción SQL (11
statements)}\label{extraccion-sql-11-statements}

\begin{Shaded}
\begin{Highlighting}[]
\NormalTok{SELECT column1, column2....columnN}
\NormalTok{FROM   table_name;}

\NormalTok{SELECT DISTINCT column1, column2....columnN}
\NormalTok{FROM   table_name;}

\NormalTok{SELECT column1, column2....columnN}
\NormalTok{FROM   table_name}
\NormalTok{WHERE  CONDITION;}

\NormalTok{SELECT column1, column2....columnN}
\NormalTok{FROM   table_name}
\NormalTok{WHERE  CONDITION}\OperatorTok{-}\DecValTok{1}\NormalTok{ \{AND}\OperatorTok{|}\NormalTok{OR\} CONDITION}\OperatorTok{-}\DecValTok{2}\NormalTok{;}

\NormalTok{SELECT column1, column2....columnN}
\NormalTok{FROM   table_name}
\NormalTok{WHERE  column_name }\KeywordTok{IN}\NormalTok{ (val}\OperatorTok{-}\DecValTok{1}\NormalTok{, val}\OperatorTok{-}\DecValTok{2}\NormalTok{,...val}\OperatorTok{-}\NormalTok{N);}

\NormalTok{SELECT column1, column2....columnN}
\NormalTok{FROM   table_name}
\NormalTok{WHERE  column_name BETWEEN val}\OperatorTok{-}\DecValTok{1}\NormalTok{ AND val}\OperatorTok{-}\DecValTok{2}\NormalTok{;}

\NormalTok{SELECT column1, column2....columnN}
\NormalTok{FROM   table_name}
\NormalTok{WHERE  column_name LIKE \{ PATTERN \};}

\NormalTok{SELECT column1, column2....columnN}
\NormalTok{FROM   table_name}
\NormalTok{WHERE  CONDITION}
\NormalTok{ORDER BY column_name \{ASC}\OperatorTok{|}\NormalTok{DESC\};}

\NormalTok{SELECT }\KeywordTok{SUM}\NormalTok{(column_name)}
\NormalTok{FROM   table_name}
\NormalTok{WHERE  CONDITION}
\NormalTok{GROUP BY column_name;}

\NormalTok{SELECT }\KeywordTok{COUNT}\NormalTok{(column_name)}
\NormalTok{FROM   table_name}
\NormalTok{WHERE  CONDITION;}

\NormalTok{SELECT }\KeywordTok{SUM}\NormalTok{(column_name)}
\NormalTok{FROM   table_name}
\NormalTok{WHERE  CONDITION}
\NormalTok{GROUP BY column_name}
\KeywordTok{HAVING}\NormalTok{ (arithematic }\ControlFlowTok{function}\NormalTok{ condition);}
\end{Highlighting}
\end{Shaded}

\subsection{Crear/Actualizar/Borrar tablas SQL (8
statements)}\label{crearactualizarborrar-tablas-sql-8-statements}

\begin{Shaded}
\begin{Highlighting}[]
\NormalTok{CREATE TABLE }\KeywordTok{table_name}\NormalTok{(}
\NormalTok{column1 datatype,}
\NormalTok{column2 datatype,}
\NormalTok{column3 datatype,}
\NormalTok{.....}
\NormalTok{columnN datatype,}
\NormalTok{PRIMARY }\KeywordTok{KEY}\NormalTok{( one or more columns )}
\NormalTok{);}

\NormalTok{DROP TABLE table_name;}

\NormalTok{CREATE UNIQUE INDEX index_name}
\NormalTok{ON }\KeywordTok{table_name}\NormalTok{ ( column1, column2,...columnN);}

\NormalTok{ALTER TABLE table_name}
\NormalTok{DROP INDEX index_name;}

\NormalTok{DESC table_name;}

\NormalTok{TRUNCATE TABLE table_name;}

\NormalTok{ALTER TABLE table_name \{ADD}\OperatorTok{|}\NormalTok{DROP}\OperatorTok{|}\NormalTok{MODIFY\} column_name \{data_ype\};}

\NormalTok{ALTER TABLE table_name RENAME TO new_table_name;}
\end{Highlighting}
\end{Shaded}

\subsection{Añadir/Actualizar/Borrar tuplas en SQL (3
statements)}\label{anadiractualizarborrar-tuplas-en-sql-3-statements}

\begin{Shaded}
\begin{Highlighting}[]
\NormalTok{INSERT INTO }\KeywordTok{table_name}\NormalTok{( column1, column2....columnN)}
\KeywordTok{VALUES}\NormalTok{ ( value1, value2....valueN);}

\NormalTok{UPDATE table_name}
\NormalTok{SET column1 =}\StringTok{ }\NormalTok{value1, column2 =}\StringTok{ }\NormalTok{value2....columnN=valueN}
\NormalTok{[ WHERE  CONDITION ];}

\NormalTok{DELETE FROM table_name}
\NormalTok{WHERE  \{CONDITION\};}
\end{Highlighting}
\end{Shaded}

\subsection{Gestión Bases de Datos (5
statements)}\label{gestion-bases-de-datos-5-statements}

\begin{Shaded}
\begin{Highlighting}[]
\NormalTok{CREATE DATABASE database_name;}

\NormalTok{DROP DATABASE database_name;}

\NormalTok{USE database_name;}

\NormalTok{COMMIT;}

\NormalTok{ROLLBACK;}
\end{Highlighting}
\end{Shaded}

\subsection{Ejemplos de consultas SQL}\label{ejemplos-de-consultas-sql}

\begin{Shaded}
\begin{Highlighting}[]
\NormalTok{SELECT Nombre, Apellido1, Apellido2, Municipio, Provincia }
\NormalTok{FROM Cliente}
\NormalTok{WHERE Municipio =}\StringTok{ 'Lugo'}
\NormalTok{ORDER BY Apellido1}

\NormalTok{INSERT }\KeywordTok{Proveedor}\NormalTok{(Nombre, PersonaContacto, Ciudad, País)}
\KeywordTok{VALUES}\NormalTok{ (}\StringTok{'Café Candelas'}\NormalTok{, }\StringTok{'Ivana Candelas'}\NormalTok{, }\StringTok{'Lugo'}\NormalTok{, }\StringTok{'España'}\NormalTok{)}

\NormalTok{UPDATE Pedidos}
\NormalTok{SET Cantidad =}\StringTok{ }\DecValTok{2}
\NormalTok{WHERE IdProducto =}\StringTok{ }\DecValTok{963}

\NormalTok{DELETE Cliente}
\NormalTok{WHERE Email =}\StringTok{ 'alexandregb@gmail.com'}
\end{Highlighting}
\end{Shaded}

\section{Conexión con bases de datos desde
R}\label{conexion-con-bases-de-datos-desde-r}

\subsection{Introducción a SQL en R}\label{introduccion-a-sql-en-r}

SQL se usa para manipular datos dentro de una base de datos. Si la base
de datos no es muy grande se puede cargar toda en un data.frame. No
obstante, por escalabilidad y offloading de la carga de trabajo al
servidor SGBD utilizaremos SQL.

Existen varios SGBD (SQLite, Microsoft SQL Server, PostgreSQL, etc) los
cuales comparten el soporte de SQL (en concreto ANSI SQL) aunque cada
gestor extiende SQL de formas sutiles buscando minar cierta portabilidad
de código (\emph{vendor-locking}). En efecto, un código SQL desarrollado
para SQLite es probable que falle con MySQL aunque tras aplicar ligeras
modificaciones ya funcionará. Asimismo el mecanismo de conexión,
configuración, rendimiento y operación suele diferir entre SGBD.

A continuación se lista una serie de paquetes utilizados en el acceso a
los datos, lo que suele ser el principal esfuerzo a realizar cuando se
trabaja con SGBD:

\begin{itemize}
\tightlist
\item
  \href{https://cran.r-project.org/web/packages/DBI/index.html}{DBI}
\item
  \href{https://cran.r-project.org/web/packages/RODBC/index.html}{RODBC}
\item
  \href{https://cran.r-project.org/web/packages/dbConnect/index.html}{dbConnect}
\item
  \href{https://cran.r-project.org/web/packages/RSQLite/index.html}{RSQLite}
\item
  \href{https://cran.r-project.org/web/packages/RMySQL/index.html}{RMySQL}
\item
  \href{https://cran.r-project.org/web/packages/RPostgreSQL/index.html}{RPostgreSQL}
\end{itemize}

\subsection{El paquete sqldf}\label{el-paquete-sqldf}

A continuación se presenta una serie de ejercicios con la sintaxis de
SQL operando sobre un data.frame con el paquete sqldf. Esto inicialmente
no incluye los detalles de conectarse a un SGBD, ni modificar los datos,
solamente el uso de SQL para extraer datos con el objetivo de ser
analizados en R.

\begin{Shaded}
\begin{Highlighting}[]
\KeywordTok{library}\NormalTok{(sqldf)}
\end{Highlighting}
\end{Shaded}

\begin{Shaded}
\begin{Highlighting}[]
\KeywordTok{sqldf}\NormalTok{(}\StringTok{'SELECT age, circumference FROM Orange WHERE Tree = 1 ORDER BY circumference ASC'}\NormalTok{)}
\end{Highlighting}
\end{Shaded}

\begin{verbatim}
##    age circumference
## 1  118            30
## 2  484            58
## 3  664            87
## 4 1004           115
## 5 1231           120
## 6 1372           142
## 7 1582           145
\end{verbatim}

\subsection{SQL Queries}\label{sql-queries}

El comando inicial es SELECT. SQL no es case-sensitive, por lo que esto
va a funcionar:

\begin{Shaded}
\begin{Highlighting}[]
\KeywordTok{sqldf}\NormalTok{(}\StringTok{"SELECT * FROM iris"}\NormalTok{)}
\KeywordTok{sqldf}\NormalTok{(}\StringTok{"select * from iris"}\NormalTok{)}
\end{Highlighting}
\end{Shaded}

pero lo siguiente no va a funcionar (a menos que tengamos un objeto
IRIS:

\begin{Shaded}
\begin{Highlighting}[]
\KeywordTok{sqldf}\NormalTok{(}\StringTok{"SELECT * FROM IRIS"}\NormalTok{)}
\end{Highlighting}
\end{Shaded}

La sintaxis básica de SELECT es:

\begin{Shaded}
\begin{Highlighting}[]
\NormalTok{SELECT variable1, variable2 FROM data}
\end{Highlighting}
\end{Shaded}

\subsubsection{Asterisco/Wildcard}\label{asteriscowildcard}

Lo extrae todo

\begin{Shaded}
\begin{Highlighting}[]
\NormalTok{bod2 <-}\StringTok{ }\KeywordTok{sqldf}\NormalTok{(}\StringTok{'SELECT * FROM BOD'}\NormalTok{)}
\end{Highlighting}
\end{Shaded}

\subsubsection{Limit}\label{limit}

Limita el número de resultados

\begin{Shaded}
\begin{Highlighting}[]
\KeywordTok{sqldf}\NormalTok{(}\StringTok{'SELECT * FROM iris LIMIT 5'}\NormalTok{)}
\end{Highlighting}
\end{Shaded}

\begin{verbatim}
##   Sepal.Length Sepal.Width Petal.Length Petal.Width Species
## 1          5.1         3.5          1.4         0.2  setosa
## 2          4.9         3.0          1.4         0.2  setosa
## 3          4.7         3.2          1.3         0.2  setosa
## 4          4.6         3.1          1.5         0.2  setosa
## 5          5.0         3.6          1.4         0.2  setosa
\end{verbatim}

\subsubsection{Order By}\label{order-by}

Ordena las variables

\begin{Shaded}
\begin{Highlighting}[]
\NormalTok{ORDER BY var1 \{ASC}\OperatorTok{/}\NormalTok{DESC\}, var2 \{ASC}\OperatorTok{/}\NormalTok{DESC\}}
\end{Highlighting}
\end{Shaded}

\begin{Shaded}
\begin{Highlighting}[]
\KeywordTok{sqldf}\NormalTok{(}\StringTok{"SELECT * FROM Orange ORDER BY age ASC, circumference DESC LIMIT 5"}\NormalTok{)}
\end{Highlighting}
\end{Shaded}

\begin{verbatim}
##   Tree age circumference
## 1    2 118            33
## 2    4 118            32
## 3    1 118            30
## 4    3 118            30
## 5    5 118            30
\end{verbatim}

\subsubsection{Where}\label{where}

Sentencias condicionales, donde se puede incorporar operadores lógicos
AND y OR, expresando el orden de evaluación con paréntesis en caso de
ser necesario.

\begin{Shaded}
\begin{Highlighting}[]
\KeywordTok{sqldf}\NormalTok{(}\StringTok{'SELECT demand FROM BOD WHERE Time < 3'}\NormalTok{)}
\end{Highlighting}
\end{Shaded}

\begin{verbatim}
##   demand
## 1    8.3
## 2   10.3
\end{verbatim}

\begin{Shaded}
\begin{Highlighting}[]
\KeywordTok{sqldf}\NormalTok{(}\StringTok{'SELECT * FROM rock WHERE (peri > 5000 AND shape < .05) OR perm > 1000'}\NormalTok{)}
\end{Highlighting}
\end{Shaded}

\begin{verbatim}
##   area     peri    shape perm
## 1 5048  941.543 0.328641 1300
## 2 1016  308.642 0.230081 1300
## 3 5605 1145.690 0.464125 1300
## 4 8793 2280.490 0.420477 1300
\end{verbatim}

Y extendiendo su uso con IN o LIKE (es último sólo con \%), pudiendo
aplicárseles el NOT:

\begin{Shaded}
\begin{Highlighting}[]
\KeywordTok{sqldf}\NormalTok{(}\StringTok{'SELECT * FROM BOD WHERE Time IN (1,7)'}\NormalTok{)}
\end{Highlighting}
\end{Shaded}

\begin{verbatim}
##   Time demand
## 1    1    8.3
## 2    7   19.8
\end{verbatim}

\begin{Shaded}
\begin{Highlighting}[]
\KeywordTok{sqldf}\NormalTok{(}\StringTok{'SELECT * FROM BOD WHERE Time NOT IN (1,7)'}\NormalTok{)}
\end{Highlighting}
\end{Shaded}

\begin{verbatim}
##   Time demand
## 1    2   10.3
## 2    3   19.0
## 3    4   16.0
## 4    5   15.6
\end{verbatim}

\begin{Shaded}
\begin{Highlighting}[]
\KeywordTok{sqldf}\NormalTok{(}\StringTok{'SELECT * FROM chickwts WHERE feed LIKE "%bean" LIMIT 5'}\NormalTok{)}
\end{Highlighting}
\end{Shaded}

\begin{verbatim}
##   weight      feed
## 1    179 horsebean
## 2    160 horsebean
## 3    136 horsebean
## 4    227 horsebean
## 5    217 horsebean
\end{verbatim}

\begin{Shaded}
\begin{Highlighting}[]
\KeywordTok{sqldf}\NormalTok{(}\StringTok{'SELECT * FROM chickwts WHERE feed NOT LIKE "%bean" LIMIT 5'}\NormalTok{)}
\end{Highlighting}
\end{Shaded}

\begin{verbatim}
##   weight    feed
## 1    309 linseed
## 2    229 linseed
## 3    181 linseed
## 4    141 linseed
## 5    260 linseed
\end{verbatim}

\chapter{\texorpdfstring{Manipulación de datos con
\texttt{dplyr}}{Manipulación de datos con dplyr}}\label{manipulacion-de-datos-con-dplyr}

\section{\texorpdfstring{El paquete
\textbf{dplyr}}{El paquete dplyr}}\label{el-paquete-dplyr}

\begin{Shaded}
\begin{Highlighting}[]
\KeywordTok{library}\NormalTok{(dplyr)}
\end{Highlighting}
\end{Shaded}

\textbf{dplyr}\footnote{\texttt{dplyr} es una mejora del paquete
  \texttt{plyr}, para una comparación ver p.e.
  \url{https://blog.rstudio.com/2014/01/17/introducing-dplyr}} permite
sustituir funciones base de R (como \texttt{split()}, \texttt{subset()},
\texttt{apply()}, \texttt{sapply()}, \texttt{lapply()},
\texttt{tapply()} y \texttt{aggregate()}) mediante una ``gramática'' más
sencilla para la manipulación de datos:

\begin{itemize}
\tightlist
\item
  \texttt{select()} seleccionar variables/columnas (también
  \texttt{rename()}).
\item
  \texttt{mutate()} crear variables/columnas (también
  \texttt{transmute()}).
\item
  \texttt{filter()} seleccionar casos/filas (también \texttt{slice()}).
\item
  \texttt{arrange()} ordenar o organizar casos/filas.
\item
  \texttt{summarise()} resumir valores.
\item
  \texttt{group\_by()} permite operaciones por grupo empleando el
  concepto ``dividir-aplicar-combinar'' (\texttt{ungroup()} elimina el
  agrupamiento).
\end{itemize}

Puede trabajar con conjuntos de datos en distintos formatos:

\begin{itemize}
\tightlist
\item
  \texttt{data.frame}, \texttt{data.table}, \texttt{tibble}, \ldots{}
\item
  bases de datos relacionales (lenguaje SQL), \ldots{}
\item
  bases de datos \emph{Hadoop} (paquete \texttt{plyrmr}).
\end{itemize}

En lugar de operar sobre vectores como las funciones base, opera sobre
objetos de este tipo (en este capítulo nos centraremos en
\texttt{data.frame}).

El tipo de objeto que emplea por defecto \texttt{dplyr} para almacenar
datos es el \href{https://tibble.tidyverse.org}{\texttt{tibble}}, una
modificación del \texttt{data.frame} para facilitar su manejo. Para más
detalles se puede ver el
\href{https://r4ds.had.co.nz/tibbles.html}{Capítulo 10} del libro
\href{https://r4ds.had.co.nz/}{R for Data Science}.

\subsection{Datos de ejemplo}\label{datos-de-ejemplo}

El fichero \emph{empleados.RData} contiene datos de empleados de un
banco. Supongamos por ejemplo que estamos interesados en estudiar si hay
discriminación por cuestión de sexo o raza.

\section{Operaciones con variables
(columnas)}\label{operaciones-con-variables-columnas}

\subsection{\texorpdfstring{Seleccionar variables con
\textbf{select()}}{Seleccionar variables con select()}}\label{seleccionar-variables-con-select}

\begin{Shaded}
\begin{Highlighting}[]
\NormalTok{emplea2 <-}\StringTok{ }\KeywordTok{select}\NormalTok{(empleados, id, sexo, minoria, tiempemp, salini, salario)}
\KeywordTok{head}\NormalTok{(emplea2)}
\end{Highlighting}
\end{Shaded}

\begin{verbatim}
##   id   sexo minoria tiempemp salini salario
## 1  1 Hombre      No       98  27000   57000
## 2  2 Hombre      No       98  18750   40200
## 3  3  Mujer      No       98  12000   21450
## 4  4  Mujer      No       98  13200   21900
## 5  5 Hombre      No       98  21000   45000
## 6  6 Hombre      No       98  13500   32100
\end{verbatim}

Se puede cambiar el nombre (ver también \emph{?rename()})

\begin{Shaded}
\begin{Highlighting}[]
\KeywordTok{head}\NormalTok{(}\KeywordTok{select}\NormalTok{(empleados, sexo, }\DataTypeTok{noblanca =}\NormalTok{ minoria, salario))}
\end{Highlighting}
\end{Shaded}

\begin{verbatim}
##     sexo noblanca salario
## 1 Hombre       No   57000
## 2 Hombre       No   40200
## 3  Mujer       No   21450
## 4  Mujer       No   21900
## 5 Hombre       No   45000
## 6 Hombre       No   32100
\end{verbatim}

Se pueden emplear los nombres de variables como índices:

\begin{Shaded}
\begin{Highlighting}[]
\KeywordTok{head}\NormalTok{(}\KeywordTok{select}\NormalTok{(empleados, sexo}\OperatorTok{:}\NormalTok{salario))}
\end{Highlighting}
\end{Shaded}

\begin{verbatim}
##     sexo    fechnac educ         catlab salario
## 1 Hombre 1952-02-03   15      Directivo   57000
## 2 Hombre 1958-05-23   16 Administrativo   40200
## 3  Mujer 1929-07-26   12 Administrativo   21450
## 4  Mujer 1947-04-15    8 Administrativo   21900
## 5 Hombre 1955-02-09   15 Administrativo   45000
## 6 Hombre 1958-08-22   15 Administrativo   32100
\end{verbatim}

\begin{Shaded}
\begin{Highlighting}[]
\KeywordTok{head}\NormalTok{(}\KeywordTok{select}\NormalTok{(empleados, }\OperatorTok{-}\NormalTok{(sexo}\OperatorTok{:}\NormalTok{salario)))}
\end{Highlighting}
\end{Shaded}

\begin{verbatim}
##   id salini tiempemp expprev minoria     sexoraza
## 1  1  27000       98     144      No Blanca varón
## 2  2  18750       98      36      No Blanca varón
## 3  3  12000       98     381      No Blanca mujer
## 4  4  13200       98     190      No Blanca mujer
## 5  5  21000       98     138      No Blanca varón
## 6  6  13500       98      67      No Blanca varón
\end{verbatim}

Hay opciones para considerar distintos criterios:
\texttt{starts\_with()}, \texttt{ends\_with()}, \texttt{contains()},
\texttt{matches()}, \texttt{one\_of()} (ver \emph{?select}).

\begin{Shaded}
\begin{Highlighting}[]
\KeywordTok{head}\NormalTok{(}\KeywordTok{select}\NormalTok{(empleados, }\KeywordTok{starts_with}\NormalTok{(}\StringTok{"s"}\NormalTok{)))}
\end{Highlighting}
\end{Shaded}

\begin{verbatim}
##     sexo salario salini     sexoraza
## 1 Hombre   57000  27000 Blanca varón
## 2 Hombre   40200  18750 Blanca varón
## 3  Mujer   21450  12000 Blanca mujer
## 4  Mujer   21900  13200 Blanca mujer
## 5 Hombre   45000  21000 Blanca varón
## 6 Hombre   32100  13500 Blanca varón
\end{verbatim}

\subsection{\texorpdfstring{Generar nuevas variables con
\textbf{mutate()}}{Generar nuevas variables con mutate()}}\label{generar-nuevas-variables-con-mutate}

\begin{Shaded}
\begin{Highlighting}[]
\KeywordTok{head}\NormalTok{(}\KeywordTok{mutate}\NormalTok{(emplea2, }\DataTypeTok{incsal =}\NormalTok{ salario }\OperatorTok{-}\StringTok{ }\NormalTok{salini, }\DataTypeTok{tsal =}\NormalTok{ incsal}\OperatorTok{/}\NormalTok{tiempemp ))}
\end{Highlighting}
\end{Shaded}

\begin{verbatim}
##   id   sexo minoria tiempemp salini salario incsal      tsal
## 1  1 Hombre      No       98  27000   57000  30000 306.12245
## 2  2 Hombre      No       98  18750   40200  21450 218.87755
## 3  3  Mujer      No       98  12000   21450   9450  96.42857
## 4  4  Mujer      No       98  13200   21900   8700  88.77551
## 5  5 Hombre      No       98  21000   45000  24000 244.89796
## 6  6 Hombre      No       98  13500   32100  18600 189.79592
\end{verbatim}

\section{Operaciones con casos
(filas)}\label{operaciones-con-casos-filas}

\subsection{\texorpdfstring{Seleccionar casos con
\textbf{filter()}}{Seleccionar casos con filter()}}\label{seleccionar-casos-con-filter}

\begin{Shaded}
\begin{Highlighting}[]
\KeywordTok{head}\NormalTok{(}\KeywordTok{filter}\NormalTok{(emplea2, sexo }\OperatorTok{==}\StringTok{ "Mujer"}\NormalTok{, minoria }\OperatorTok{==}\StringTok{ "Sí"}\NormalTok{))}
\end{Highlighting}
\end{Shaded}

\begin{verbatim}
##   id  sexo minoria tiempemp salini salario
## 1 14 Mujer      Sí       98  16800   35100
## 2 23 Mujer      Sí       97  11100   24000
## 3 24 Mujer      Sí       97   9000   16950
## 4 25 Mujer      Sí       97   9000   21150
## 5 40 Mujer      Sí       96   9000   19200
## 6 41 Mujer      Sí       96  11550   23550
\end{verbatim}

\subsection{\texorpdfstring{Organizar casos con
\textbf{arrange()}}{Organizar casos con arrange()}}\label{organizar-casos-con-arrange}

\begin{Shaded}
\begin{Highlighting}[]
\KeywordTok{head}\NormalTok{(}\KeywordTok{arrange}\NormalTok{(emplea2, salario))}
\end{Highlighting}
\end{Shaded}

\begin{verbatim}
##    id  sexo minoria tiempemp salini salario
## 1 378 Mujer      No       70  10200   15750
## 2 338 Mujer      No       74  10200   15900
## 3  90 Mujer      No       92   9750   16200
## 4 224 Mujer      No       82  10200   16200
## 5 411 Mujer      No       68  10200   16200
## 6 448 Mujer      Sí       66  10200   16350
\end{verbatim}

\begin{Shaded}
\begin{Highlighting}[]
\KeywordTok{head}\NormalTok{(}\KeywordTok{arrange}\NormalTok{(emplea2, }\KeywordTok{desc}\NormalTok{(salini), salario))}
\end{Highlighting}
\end{Shaded}

\begin{verbatim}
##    id   sexo minoria tiempemp salini salario
## 1  29 Hombre      No       96  79980  135000
## 2 343 Hombre      No       73  60000  103500
## 3 205 Hombre      No       83  52500   66750
## 4 160 Hombre      No       86  47490   66000
## 5 431 Hombre      No       66  45000   86250
## 6  32 Hombre      No       96  45000  110625
\end{verbatim}

\section{\texorpdfstring{Resumir valores con
\textbf{summarise()}}{Resumir valores con summarise()}}\label{resumir-valores-con-summarise}

\begin{Shaded}
\begin{Highlighting}[]
\KeywordTok{summarise}\NormalTok{(empleados, }\DataTypeTok{sal.med =} \KeywordTok{mean}\NormalTok{(salario), }\DataTypeTok{n =} \KeywordTok{n}\NormalTok{())}
\end{Highlighting}
\end{Shaded}

\begin{verbatim}
##    sal.med   n
## 1 34419.57 474
\end{verbatim}

\section{\texorpdfstring{Agrupar casos con
\textbf{group\_by()}}{Agrupar casos con group\_by()}}\label{agrupar-casos-con-group_by}

\begin{Shaded}
\begin{Highlighting}[]
\KeywordTok{summarise}\NormalTok{(}\KeywordTok{group_by}\NormalTok{(empleados, sexo, minoria), }\DataTypeTok{sal.med =} \KeywordTok{mean}\NormalTok{(salario), }\DataTypeTok{n =} \KeywordTok{n}\NormalTok{())}
\end{Highlighting}
\end{Shaded}

\begin{verbatim}
## # A tibble: 4 x 4
## # Groups:   sexo [2]
##   sexo   minoria sal.med     n
##   <fct>  <fct>     <dbl> <int>
## 1 Hombre No       44475.   194
## 2 Hombre Sí       32246.    64
## 3 Mujer  No       26707.   176
## 4 Mujer  Sí       23062.    40
\end{verbatim}

\section{\texorpdfstring{Operador \emph{pipe} \textbf{\%\textgreater{}\%
}(tubería,
redirección)}{Operador pipe \%\textgreater{}\% (tubería, redirección)}}\label{operador-pipe-tuberia-redireccion}

Este operador le permite canalizar la salida de una función a la entrada
de otra función. \texttt{segundo(primero(datos))} se traduce en
\texttt{datos\ \%\textgreater{}\%\ primero\ \%\textgreater{}\%\ segundo}
(lectura de funciones de izquierda a derecha).

Ejemplos:

\begin{Shaded}
\begin{Highlighting}[]
\NormalTok{empleados }\OperatorTok\StringTok{  }\KeywordTok{filter}\NormalTok{(catlab }\OperatorTok{==}\StringTok{ "Directivo"}\NormalTok{) }\OperatorTok
\StringTok{          }\KeywordTok{group_by}\NormalTok{(sexo, minoria) }\OperatorTok
\StringTok{          }\KeywordTok{summarise}\NormalTok{(}\DataTypeTok{sal.med =} \KeywordTok{mean}\NormalTok{(salario), }\DataTypeTok{n =} \KeywordTok{n}\NormalTok{())}
\end{Highlighting}
\end{Shaded}

\begin{verbatim}
## # A tibble: 3 x 4
## # Groups:   sexo [2]
##   sexo   minoria sal.med     n
##   <fct>  <fct>     <dbl> <int>
## 1 Hombre No       65684.    70
## 2 Hombre Sí       76038.     4
## 3 Mujer  No       47214.    10
\end{verbatim}

\begin{Shaded}
\begin{Highlighting}[]
\NormalTok{empleados }\OperatorTok\StringTok{ }\KeywordTok{select}\NormalTok{(sexo, catlab, salario) }\OperatorTok
\StringTok{          }\KeywordTok{filter}\NormalTok{(catlab }\OperatorTok{!=}\StringTok{ "Seguridad"}\NormalTok{) }\OperatorTok
\StringTok{          }\KeywordTok{group_by}\NormalTok{(catlab) }\OperatorTok
\StringTok{          }\KeywordTok{mutate}\NormalTok{(}\DataTypeTok{saldif =}\NormalTok{ salario }\OperatorTok{-}\StringTok{ }\KeywordTok{mean}\NormalTok{(salario)) }\OperatorTok
\StringTok{          }\KeywordTok{ungroup}\NormalTok{() }\OperatorTok
\StringTok{          }\KeywordTok{boxplot}\NormalTok{(saldif }\OperatorTok{~}\StringTok{ }\NormalTok{sexo}\OperatorTok{*}\KeywordTok{droplevels}\NormalTok{(catlab), }\DataTypeTok{data =}\NormalTok{ .)}
\KeywordTok{abline}\NormalTok{(}\DataTypeTok{h =} \DecValTok{0}\NormalTok{, }\DataTypeTok{lty =} \DecValTok{2}\NormalTok{)}
\end{Highlighting}
\end{Shaded}

\includegraphics{04-R-dplyr_files/figure-latex/unnamed-chunk-12-1.pdf}

\begin{center}\rule{0.5\linewidth}{\linethickness}\end{center}

Para mas información sobre \emph{dplyr} ver por ejemplo la `vignette'
del paquete:\\
\href{https://cran.rstudio.com/web/packages/dplyr/vignettes/dplyr.html}{Introduction
to dplyr}.

\chapter{Introducción a Tecnologías
NoSQL}\label{introduccion-a-tecnologias-nosql}

\section{Conceptos y tipos de bases de datos NoSQL (documental,
columnar, clave/valor y de
grafos)}\label{conceptos-y-tipos-de-bases-de-datos-nosql-documental-columnar-clavevalor-y-de-grafos}

NoSQL - ``Not Only SQL'' - es una nueva categoría de bases de datos
no-relacionales y altamente distribuidas.

Las bases de datos NoSQL nacen de la necesidad de:

\begin{itemize}
\item
  Simplicidad en los diseños
\item
  Escalado horizontal
\item
  Mayor control en la disponibilidad
\end{itemize}

Pero cuidado, en muchos escenarios las BBDD relacionales siguen siendo
la mejor opción.

\subsection{Características de las bases de datos
NoSQL}\label{caracteristicas-de-las-bases-de-datos-nosql}

\begin{itemize}
\tightlist
\item
  Libre de esquemas -- no se diseñan las tablas y relaciones por
  adelantado, además de permitir la migración del esquema.
\item
  Proporcionan replicación a través de escalado horizontal.
\item
  Este escalado horizontal se traduce en arquitectura distribuida
\item
  Generalmente ofrecen consistencia débil
\item
  Hacen uso de estructuras de datos sencillas, normalmente pares
  clave/valor a bajo nivel
\item
  Suelen tener un sistema de consultas propio (o SQL-like)
\item
  Siguen el modelo BASE (\emph{B}asic Availability, Soft state, Eventual
  consistency) en lugar de ACID (\emph{A}tomicity, \emph{C}onsistency,
  \emph{I}solation, \emph{D}urability)
\end{itemize}

El modelo BASE consiste en:

\begin{itemize}
\tightlist
\item
  Basic Availability -- el sistema garantiza disponibilidad, en términos
  del teorema CAP.
\item
  Soft state -- el estado del sistema puede cambiar a lo largo del
  tiempo, incluso sin entrada. Esto es provocado por el modelo de
  consistencia eventual.
\item
  Eventual consistency -- el sistema alcanzará un estado consistente con
  el tiempo, siempre y cuando no reciba entrada durante ese tiempo.
\end{itemize}

\subsubsection{Teorema CAP}\label{teorema-cap}

Es imposible para un sistema de cómputo distribuido garantizar
simultáneamente:

\begin{itemize}
\tightlist
\item
  Consistency -- Todos los nodos ven los mismos datos al mismo tiempo
\item
  Availability -- Toda petición obtiene una respuesta en caso tanto de
  éxito como fallo
\item
  Partition Tolerance -- El sistema seguirá funcionando ante pérdidas
  arbitrarias de información o fallos parciales
\end{itemize}

\begin{figure}
\centering
\includegraphics{images/TeoremaCAP.jpg}
\caption{}
\end{figure}

Las razones para escoger NoSQL son:

\begin{itemize}
\tightlist
\item
  Analítica
\item
  Gran cantidad de escrituras, análisis en bloque
\item
  Escalabilidad
\item
  Tan fácil como añadir un nuevo nodo a la red, bajo coste.
\item
  Redundancia
\item
  Están diseñadas teniendo en cuenta la redundancia
\item
  Rápido desarrollo
\item
  Al ser schema-less o schema on-read son más flexibles que schema
  on-write
\item
  Flexibilidad en el almacenamiento de datos
\item
  Almacenan todo tipo de datos: texto, imágenes, BLOBs
\item
  Gran rendimiento en consultas sobre datos que no implican relaciones
  jerárquicas
\item
  Gran rendimiento sobre BBDD desnormalizadas
\item
  Tamaño
\item
  El tamaño del esquema de datos es demasiado grande
\item
  Muchos datos temporales fuera de almacén principal
\end{itemize}

Razones para NO escoger NoSQL: * Consistencia y Disponibilidad de los
datos son críticas * Relaciones entre datos son importantes + E.g. joins
numerosos y/o importantes * En general, cuando el modelo ACID encaja
mejor

\subsection{Tipos de Bases de Datos
NoSQL}\label{tipos-de-bases-de-datos-nosql}

\begin{figure}
\centering
\includegraphics{images/TiposBBDDNoSQL.png}
\caption{}
\end{figure}

\begin{figure}
\centering
\includegraphics{images/TiposBBDDNoSQL2.png}
\caption{}
\end{figure}

\begin{figure}
\centering
\includegraphics{images/451ResearchMap.png}
\caption{}
\end{figure}

\begin{figure}
\centering
\includegraphics{images/DBEnginesRanking.png}
\caption{}
\end{figure}

\begin{figure}
\centering
\includegraphics{images/451ResearchSkills.png}
\caption{}
\end{figure}

\subsection{MongoDB: NoSQL documental}\label{mongodb-nosql-documental}

\begin{figure}
\centering
\includegraphics{images/MongoDB.jpg}
\caption{}
\end{figure}

\subsection{Redis: NoSQL key-value}\label{redis-nosql-key-value}

In-memory data structure store, útil para base de datos de
login-password, sensor-valor, URL-respuesta, con una sintaxis muy
sencilla:

\begin{itemize}
\tightlist
\item
  El comando SET almacena valores
\item
  SET server:name ``luna''
\item
  Recuperamos esos valores con GET
\item
  GET server:name
\item
  INCR incrementa atómicamente un valor
\item
  INCR clients
\item
  DEL elimina claves y sus valores asociados
\item
  DEL clients
\item
  TTL (Time To Live) útil para cachés
\item
  EXPIRE promocion 60
\end{itemize}

\subsection{Cassandra: NoSQL columnar}\label{cassandra-nosql-columnar}

\begin{figure}
\centering
\includegraphics{images/BlogRDMS.png}
\caption{}
\end{figure}

\begin{figure}
\centering
\includegraphics{images/BlogNoSQL.png}
\caption{}
\end{figure}

\subsection{Neo4j: NoSQL grafos}\label{neo4j-nosql-grafos}

\begin{figure}
\centering
\includegraphics{images/Neo4jlogo.png}
\caption{}
\end{figure}

\begin{figure}
\centering
\includegraphics{images/CypherQuery.png}
\caption{}
\end{figure}

\begin{figure}
\centering
\includegraphics{images/CypherResult.png}
\caption{}
\end{figure}

\subsection{Otros: search engines}\label{otros-search-engines}

Son sistemas especializados en búsquedas, procesamiento de lenguaje
natural como ElasticSearch, Solr, Splunk (logs de aplicaciones),
etc\ldots{}

\section{Conexión de R a NoSQL}\label{conexion-de-r-a-nosql}

Step1: Install Packages plyr,XML Step2: Take xml file url Step3: create
XML Internal Document type object in R using xmlParse() Step4 :Convert
xml object to list by using xmlToList() Step5: convert list object to
data frame by using ldply(xl, data.frame)

\begin{Shaded}
\begin{Highlighting}[]
\KeywordTok{install.packages}\NormalTok{(}\StringTok{"XML"}\NormalTok{)}
\end{Highlighting}
\end{Shaded}

\begin{Shaded}
\begin{Highlighting}[]
\KeywordTok{install.packages}\NormalTok{(}\StringTok{"plyr"}\NormalTok{)}
\end{Highlighting}
\end{Shaded}

\chapter{Tecnologías para el Tratamiendo de Datos
Masivos}\label{tecnologias-para-el-tratamiendo-de-datos-masivos}

\section{Tecnologías Big Data (Hadoop, Spark, Hive, Rspark,
Sparklyr)}\label{tecnologias-big-data-hadoop-spark-hive-rspark-sparklyr}

Introducción a los conceptos básicos del ecosistema Hadoop

\section{Visualización y Generación de Cuadros de
Mando}\label{visualizacion-y-generacion-de-cuadros-de-mando}

\section{Introducción al Análisis de Datos
Masivos}\label{introduccion-al-analisis-de-datos-masivos}

\appendix


\chapter{Enlaces}\label{links}

\textbf{Recursos para el aprendizaje de R} (
\url{https://rubenfcasal.github.io/post/ayuda-y-recursos-para-el-aprendizaje-de-r}
): A continuación se muestran algunos recursos que pueden ser útiles
para el aprendizaje de R y la obtención de ayuda\ldots{}

\textbf{\emph{Ayuda online}}:

\begin{itemize}
\item
  Ayuda en línea sobre funciones o paquetes:
  \href{https://www.rdocumentation.org/}{RDocumentation}
\item
  Buscador \href{http://rseek.org/}{RSeek}
\item
  \href{http://stackoverflow.com/questions/tagged/r}{StackOverflow}
\end{itemize}

\textbf{\emph{Cursos}}: algunos cursos gratuitos:

\begin{itemize}
\item
  \href{https://www.coursera.org/}{Coursera}:

  \begin{itemize}
  \item
    \href{https://www.coursera.org/learn/intro-data-science-programacion-estadistica-r}{Introducción
    a Data Science: Programación Estadística con R}
  \item
    \href{https://www.coursera.org/specializations/r}{Mastering Software
    Development in R}
  \end{itemize}
\end{itemize}

\begin{itemize}
\item
  \href{https://www.datacamp.com/courses}{DataCamp}:

  \begin{itemize}
  \tightlist
  \item
    \href{https://www.datacamp.com/courses/introduccion-a-r/}{Introducción
    a R}
  \end{itemize}
\end{itemize}

\begin{itemize}
\item
  \href{http://online.stanford.edu/courses}{Stanford online}:

  \begin{itemize}
  \tightlist
  \item
    \href{http://online.stanford.edu/course/statistical-learning}{Statistical
    Learning}
  \end{itemize}
\end{itemize}

\begin{itemize}
\tightlist
\item
  Curso UCA:
  \href{http://knuth.uca.es/moodle/course/view.php?id=51}{Introducción a
  R, R-commander y shiny}
\end{itemize}

\begin{itemize}
\tightlist
\item
  Udacity:
  \href{https://eu.udacity.com/course/data-analysis-with-r--ud651}{Data
  Analysis with R}
\end{itemize}

\begin{itemize}
\tightlist
\item
  \href{https://swirlstats.com/scn/title.html}{Swirl Courses}: se pueden
  hacer cursos desde el propio R con el paquete
  \href{https://swirlstats.com}{swirl}.
\end{itemize}

Para información sobre cursos en castellano se puede recurrir a la web
de \href{http://r-es.org/}{R-Hispano} en el apartado
\href{http://r-es.org/category/formacion}{formación}. Algunos de los
cursos que aparecen en entradas antiguas son gratuitos. Ver:
\href{http://r-es.org/2016/02/12/cursos-masivos-y-otra-formacion-on-line-sobre-r/}{Cursos
MOOC relacionados con R}.

\textbf{\emph{Libros}}

\begin{itemize}
\item
  \textbf{\emph{Iniciación}}:

  \begin{itemize}
  \item
    R for Data Science (\href{http://r4ds.had.co.nz/}{online},
    \href{http://shop.oreilly.com/product/0636920034407.do}{O'Reilly})
  \item
    2011 - The Art of R Programming. A Tour of Statistical Software
    Design, (\href{https://www.nostarch.com/artofr.htm}{No Starch
    Press})
  \item
    Hands-On Programming with R: Write Your Own Functions and
    Simulations, by Garrett Grolemund
    (\href{http://shop.oreilly.com/product/0636920028574.do}{O'Reilly})
  \end{itemize}
\item
  \textbf{\emph{Avanzados}}:

  \begin{itemize}
  \item
    2008 - Software for Data Analysis: Programming with R - Chambers
    (\href{http://www.springer.com/la/book/9780387759357}{Springer})
  \item
    Advanced R by Hadley Wickham (online:
    \href{http://adv-r.had.co.nz/}{1ª ed},
    \href{https://adv-r.hadley.nz/}{2ª ed},
    \href{https://www.amazon.com/dp/1466586966}{Chapman \& Hall})
  \item
    R packages by Hadley Wickham
    (\href{http://r-pkgs.had.co.nz/}{online},
    \href{http://shop.oreilly.com/product/0636920034421.do}{O'Reilly})
  \end{itemize}
\item
  \textbf{\emph{Bookdown}}: el paquete
  \href{https://bookdown.org}{\texttt{bookdown}} de R permite escribir
  libros empleando \href{http://rmarkdown.rstudio.com}{R Markdown} y
  compartirlos. En \url{https://bookdown.org} está disponible una
  selección de libros escritos con este paquete (un listado más completo
  está disponible \href{https://bookdown.org/home/archive/}{aquí}).
  Algunos libros en este formato en castellano son:

  \begin{itemize}
  \item
    \href{https://rubenfcasal.github.io/simbook}{Prácticas de
    Simulación} (disponible en el repositorio de GitHub
    \href{https://github.com/rubenfcasal/simbook}{rubenfcasal/simbook}).
  \item
    \href{https://rubenfcasal.github.io/bookdown_intro/}{Escritura de
    libros con bookdown} (disponible en el repositorio de GitHub
    \href{https://github.com/rubenfcasal/bookdown_intro}{rubenfcasal/bookdown\_intro}).
  \item
    \href{https://www.datanalytics.com/libro_r/index.html}{R para
    profesionales de los datos: una introducción}.
  \item
    \href{https://bookdown.org/aquintela/EBE}{Estadística Básica
    Edulcorada}.
  \end{itemize}
\end{itemize}

\textbf{\emph{Material online}}: en la web se puede encontrar mucho
material adicional, por ejemplo:

\begin{itemize}
\item
  \href{https://www.r-project.org/other-docs.html}{CRAN: Other R
  documentation}
\item
  Blogs en inglés:

  \begin{itemize}
  \item
    \url{https://www.r-bloggers.com/}
  \item
    \url{https://www.littlemissdata.com/blog/rstudioconf2019}
  \item
    RStudio: \url{https://blog.rstudio.com}
  \item
    Microsoft Revolutions: \url{https://blog.revolutionanalytics.com}
  \end{itemize}
\item
  Blogs en castellano:

  \begin{itemize}
  \item
    \url{https://www.datanalytics.com}
  \item
    \url{http://oscarperpinan.github.io/R}
  \item
    \url{http://rubenfcasal.github.io}
  \end{itemize}
\item
  Listas de correo:

  \begin{itemize}
  \item
    Listas de distribución de r-project.org:
    \url{https://stat.ethz.ch/mailman/listinfo}
  \item
    Búsqueda en R-help:
    \url{http://r.789695.n4.nabble.com/R-help-f789696.html}
  \item
    Búsqueda en R-help-es: \url{https://r-help-es.r-project.narkive.com}

    \url{https://grokbase.com/g/r/r-help-es}
  \item
    Archivos de R-help-es:
    \url{https://stat.ethz.ch/pipermail/r-help-es}
  \end{itemize}
\end{itemize}

\section{RStudio}\label{rstudio-links}

\href{https://www.rstudio.com}{RStudio}:

\begin{itemize}
\item
  \href{https://www.rstudio.com/online-learning}{Online learning},
\item
  \href{https://resources.rstudio.com/webinars}{Webinars},
\item
  \href{https://www.tidyverse.org/}{tidyverse},
\item
  \href{https://db.rstudio.com}{Databases using R},
  \href{https://db.rstudio.com/overview}{dplyr as a database interface},
\item
  \href{https://spark.rstudio.com/}{sparklyr},
\item
  \href{http://shiny.rstudio.com}{shiny},
\item
  \href{https://resources.rstudio.com/rstudio-cheatsheets}{CheatSheets}:
  \href{https://resources.rstudio.com/rstudio-cheatsheets/rmarkdown-2-0-cheat-sheet}{rmarkdown},
  \href{https://resources.rstudio.com/rstudio-cheatsheets/shiny-cheat-sheet}{shiny},
  \href{https://github.com/rstudio/cheatsheets/blob/master/data-transformation.pdf}{dplyr},
  \href{https://github.com/rstudio/cheatsheets/blob/master/data-import.pdf}{tidyr},
  \href{https://resources.rstudio.com/rstudio-cheatsheets/stringr-cheat-sheet}{stringr}.
\end{itemize}

\bibliography{book.bib,packages.bib}


\end{document}
