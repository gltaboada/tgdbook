% Options for packages loaded elsewhere
\PassOptionsToPackage{unicode}{hyperref}
\PassOptionsToPackage{hyphens}{url}
\documentclass[
]{book}
\usepackage{xcolor}
\usepackage{amsmath,amssymb}
\setcounter{secnumdepth}{5}
\usepackage{iftex}
\ifPDFTeX
  \usepackage[T1]{fontenc}
  \usepackage[utf8]{inputenc}
  \usepackage{textcomp} % provide euro and other symbols
\else % if luatex or xetex
  \usepackage{unicode-math} % this also loads fontspec
  \defaultfontfeatures{Scale=MatchLowercase}
  \defaultfontfeatures[\rmfamily]{Ligatures=TeX,Scale=1}
\fi
\usepackage{lmodern}
\ifPDFTeX\else
  % xetex/luatex font selection
\fi
% Use upquote if available, for straight quotes in verbatim environments
\IfFileExists{upquote.sty}{\usepackage{upquote}}{}
\IfFileExists{microtype.sty}{% use microtype if available
  \usepackage[]{microtype}
  \UseMicrotypeSet[protrusion]{basicmath} % disable protrusion for tt fonts
}{}
\makeatletter
\@ifundefined{KOMAClassName}{% if non-KOMA class
  \IfFileExists{parskip.sty}{%
    \usepackage{parskip}
  }{% else
    \setlength{\parindent}{0pt}
    \setlength{\parskip}{6pt plus 2pt minus 1pt}}
}{% if KOMA class
  \KOMAoptions{parskip=half}}
\makeatother
\usepackage{color}
\usepackage{fancyvrb}
\newcommand{\VerbBar}{|}
\newcommand{\VERB}{\Verb[commandchars=\\\{\}]}
\DefineVerbatimEnvironment{Highlighting}{Verbatim}{commandchars=\\\{\}}
% Add ',fontsize=\small' for more characters per line
\usepackage{framed}
\definecolor{shadecolor}{RGB}{248,248,248}
\newenvironment{Shaded}{\begin{snugshade}}{\end{snugshade}}
\newcommand{\AlertTok}[1]{\textcolor[rgb]{0.94,0.16,0.16}{#1}}
\newcommand{\AnnotationTok}[1]{\textcolor[rgb]{0.56,0.35,0.01}{\textbf{\textit{#1}}}}
\newcommand{\AttributeTok}[1]{\textcolor[rgb]{0.13,0.29,0.53}{#1}}
\newcommand{\BaseNTok}[1]{\textcolor[rgb]{0.00,0.00,0.81}{#1}}
\newcommand{\BuiltInTok}[1]{#1}
\newcommand{\CharTok}[1]{\textcolor[rgb]{0.31,0.60,0.02}{#1}}
\newcommand{\CommentTok}[1]{\textcolor[rgb]{0.56,0.35,0.01}{\textit{#1}}}
\newcommand{\CommentVarTok}[1]{\textcolor[rgb]{0.56,0.35,0.01}{\textbf{\textit{#1}}}}
\newcommand{\ConstantTok}[1]{\textcolor[rgb]{0.56,0.35,0.01}{#1}}
\newcommand{\ControlFlowTok}[1]{\textcolor[rgb]{0.13,0.29,0.53}{\textbf{#1}}}
\newcommand{\DataTypeTok}[1]{\textcolor[rgb]{0.13,0.29,0.53}{#1}}
\newcommand{\DecValTok}[1]{\textcolor[rgb]{0.00,0.00,0.81}{#1}}
\newcommand{\DocumentationTok}[1]{\textcolor[rgb]{0.56,0.35,0.01}{\textbf{\textit{#1}}}}
\newcommand{\ErrorTok}[1]{\textcolor[rgb]{0.64,0.00,0.00}{\textbf{#1}}}
\newcommand{\ExtensionTok}[1]{#1}
\newcommand{\FloatTok}[1]{\textcolor[rgb]{0.00,0.00,0.81}{#1}}
\newcommand{\FunctionTok}[1]{\textcolor[rgb]{0.13,0.29,0.53}{\textbf{#1}}}
\newcommand{\ImportTok}[1]{#1}
\newcommand{\InformationTok}[1]{\textcolor[rgb]{0.56,0.35,0.01}{\textbf{\textit{#1}}}}
\newcommand{\KeywordTok}[1]{\textcolor[rgb]{0.13,0.29,0.53}{\textbf{#1}}}
\newcommand{\NormalTok}[1]{#1}
\newcommand{\OperatorTok}[1]{\textcolor[rgb]{0.81,0.36,0.00}{\textbf{#1}}}
\newcommand{\OtherTok}[1]{\textcolor[rgb]{0.56,0.35,0.01}{#1}}
\newcommand{\PreprocessorTok}[1]{\textcolor[rgb]{0.56,0.35,0.01}{\textit{#1}}}
\newcommand{\RegionMarkerTok}[1]{#1}
\newcommand{\SpecialCharTok}[1]{\textcolor[rgb]{0.81,0.36,0.00}{\textbf{#1}}}
\newcommand{\SpecialStringTok}[1]{\textcolor[rgb]{0.31,0.60,0.02}{#1}}
\newcommand{\StringTok}[1]{\textcolor[rgb]{0.31,0.60,0.02}{#1}}
\newcommand{\VariableTok}[1]{\textcolor[rgb]{0.00,0.00,0.00}{#1}}
\newcommand{\VerbatimStringTok}[1]{\textcolor[rgb]{0.31,0.60,0.02}{#1}}
\newcommand{\WarningTok}[1]{\textcolor[rgb]{0.56,0.35,0.01}{\textbf{\textit{#1}}}}
\usepackage{longtable,booktabs,array}
\usepackage{calc} % for calculating minipage widths
% Correct order of tables after \paragraph or \subparagraph
\usepackage{etoolbox}
\makeatletter
\patchcmd\longtable{\par}{\if@noskipsec\mbox{}\fi\par}{}{}
\makeatother
% Allow footnotes in longtable head/foot
\IfFileExists{footnotehyper.sty}{\usepackage{footnotehyper}}{\usepackage{footnote}}
\makesavenoteenv{longtable}
\usepackage{graphicx}
\makeatletter
\newsavebox\pandoc@box
\newcommand*\pandocbounded[1]{% scales image to fit in text height/width
  \sbox\pandoc@box{#1}%
  \Gscale@div\@tempa{\textheight}{\dimexpr\ht\pandoc@box+\dp\pandoc@box\relax}%
  \Gscale@div\@tempb{\linewidth}{\wd\pandoc@box}%
  \ifdim\@tempb\p@<\@tempa\p@\let\@tempa\@tempb\fi% select the smaller of both
  \ifdim\@tempa\p@<\p@\scalebox{\@tempa}{\usebox\pandoc@box}%
  \else\usebox{\pandoc@box}%
  \fi%
}
% Set default figure placement to htbp
\def\fps@figure{htbp}
\makeatother
\setlength{\emergencystretch}{3em} % prevent overfull lines
\providecommand{\tightlist}{%
  \setlength{\itemsep}{0pt}\setlength{\parskip}{0pt}}
\usepackage[]{natbib}
\bibliographystyle{apalike}
\usepackage{booktabs}
\usepackage{amsthm}
%\usepackage{animate}
\ifxetex
  \usepackage{polyglossia}
  \setmainlanguage{spanish}
  % Tabla en lugar de cuadro
  \gappto\captionsspanish{\renewcommand{\tablename}{Tabla}
          \renewcommand{\listtablename}{Índice de tablas}}

\else
  \usepackage[spanish,es-tabla]{babel}
\fi
\makeatletter
\def\thm@space@setup{%
  \thm@preskip=8pt plus 2pt minus 4pt
  \thm@postskip=\thm@preskip
}
\makeatother
\usepackage{bookmark}
\IfFileExists{xurl.sty}{\usepackage{xurl}}{} % add URL line breaks if available
\urlstyle{same}
\hypersetup{
  pdftitle={Prácticas de Tecnologías de Gestión y Manipulación de Datos},
  pdfauthor={Guillermo López Taboada (guillermo.lopez.taboada@udc.es), Diego Darriba (diego.darriba@udc.es) y Rubén F. Casal (ruben.fcasal@udc.es)},
  hidelinks,
  pdfcreator={LaTeX via pandoc}}

\title{Prácticas de Tecnologías de Gestión y Manipulación de Datos}
\author{Guillermo López Taboada (\href{mailto:guillermo.lopez.taboada@udc.es}{\nolinkurl{guillermo.lopez.taboada@udc.es}}), Diego Darriba (\href{mailto:diego.darriba@udc.es}{\nolinkurl{diego.darriba@udc.es}}) y Rubén F. Casal (\href{mailto:ruben.fcasal@udc.es}{\nolinkurl{ruben.fcasal@udc.es}})}
\date{2025-10-24}

\begin{document}
\maketitle

{
\setcounter{tocdepth}{1}
\tableofcontents
}
\chapter*{Prólogo}\label{pruxf3logo}
\addcontentsline{toc}{chapter}{Prólogo}

Este libro contiene algunas de las prácticas de la asignatura de
\href{http://eamo.usc.es/pub/mte/index.php/es/?option=com_content&view=article&id=2202&idm=38&a\%C3\%B1o=2020}{Tecnologías de Gestión de
Datos}
del \href{http://eio.usc.es/pub/mte}{Máster interuniversitario en Técnicas
Estadísticas}).

Este libro ha sido escrito en \href{http://rmarkdown.rstudio.com}{R-Markdown}
empleando el paquete \href{https://bookdown.org/yihui/bookdown/}{\texttt{bookdown}}
y está disponible en el repositorio Github:
\href{https://github.com/gltaboada/tgdbook}{gltaboada/tgdbook}. Se puede
acceder a la versión en línea a través del siguiente enlace:

\url{https://gltaboada.github.io/tgdbook}.

donde puede descargarse en formato
\href{https://gltaboada.github.io/tgdbook/Practicas_de_TGD.pdf}{pdf}.

Para ejecutar los ejemplos mostrados en el libro será necesario tener
instalados los siguientes paquetes:
\href{https://dplyr.tidyverse.org}{\texttt{dplyr}} (colección
\href{https://www.tidyverse.org/}{\texttt{tidyverse}}),
\href{https://tidyr.tidyverse.org}{\texttt{tidyr}},
\href{https://stringr.tidyverse.org}{\texttt{stringr}},
\href{https://readxl.tidyverse.org}{\texttt{readxl}} ,
\href{https://cran.r-project.org/web/packages/openxlsx/index.html}{\texttt{openxlsx}},
\href{https://cran.r-project.org/web/packages/RODBC/index.html}{\texttt{RODBC}},
\href{https://cran.r-project.org/web/packages/sqldf/index.html}{\texttt{sqldf}},
\href{https://r-dbi.github.io/RSQLite}{\texttt{RSQLite}},
\href{https://cran.r-project.org/web/packages/foreign/index.html}{\texttt{foreign}},
\href{https://cran.r-project.org/web/packages/magrittr/index.html}{\texttt{magrittr}},
\href{https://yihui.name/knitr}{\texttt{knitr}} Por ejemplo mediante los comandos:

\begin{Shaded}
\begin{Highlighting}[]
\NormalTok{pkgs }\OtherTok{\textless{}{-}} \FunctionTok{c}\NormalTok{(}\StringTok{\textquotesingle{}dplyr\textquotesingle{}}\NormalTok{, }\StringTok{\textquotesingle{}tidyr\textquotesingle{}}\NormalTok{, }\StringTok{\textquotesingle{}stringr\textquotesingle{}}\NormalTok{, }\StringTok{\textquotesingle{}readxl\textquotesingle{}}\NormalTok{, }\StringTok{\textquotesingle{}openxlsx\textquotesingle{}}\NormalTok{, }\StringTok{\textquotesingle{}magrittr\textquotesingle{}}\NormalTok{, }
          \StringTok{\textquotesingle{}RODBC\textquotesingle{}}\NormalTok{, }\StringTok{\textquotesingle{}sqldf\textquotesingle{}}\NormalTok{, }\StringTok{\textquotesingle{}RSQLite\textquotesingle{}}\NormalTok{, }\StringTok{\textquotesingle{}foreign\textquotesingle{}}\NormalTok{, }\StringTok{\textquotesingle{}knitr\textquotesingle{}}\NormalTok{)}
\CommentTok{\# install.packages(pkgs, dependencies=TRUE)}
\FunctionTok{install.packages}\NormalTok{(}\FunctionTok{setdiff}\NormalTok{(pkgs, }\FunctionTok{installed.packages}\NormalTok{()[,}\StringTok{\textquotesingle{}Package\textquotesingle{}}\NormalTok{]), }\AttributeTok{dependencies =} \ConstantTok{TRUE}\NormalTok{)}
\end{Highlighting}
\end{Shaded}

Para generar el libro (compilar) se recomendaría consultar el libro de
\href{https://rubenfcasal.github.io/bookdown_intro}{``Escritura de libros con
bookdown''} en castellano.

\includegraphics[width=1.22in]{images/by-nc-nd-88x31}

Este obra está bajo una licencia de \href{https://creativecommons.org/licenses/by-nc-nd/4.0/deed.es_ES}{Creative Commons
Reconocimiento-NoComercial-SinObraDerivada 4.0
Internacional}
(esperamos poder liberarlo bajo una licencia menos restrictiva más
adelante\ldots).

\chapter{Introducción}\label{introducciuxf3n}

La información relevante de la materia está disponible en la guía docente y la ficha de la asignatura

En particular, los resultados de aprendizaje son:

\begin{itemize}
\item
  Manejar de forma autónoma y solvente el software necesario para acceder a conjuntos de datos en entornos profesionales y/o en la nube.
\item
  Saber gestionar conjuntos de datos masivos en un entorno multidisciplinar que permita la participación en proyectos profesionales complejos que requieran el uso de técnicas estadísticas.
\item
  Saber relacionar el software de diseño y gestión de bases de datos con el específicamente implementado para el análisis de datos.
\end{itemize}

\section{Contenidos}\label{contenidos}

\begin{enumerate}
\def\labelenumi{\arabic{enumi}.}
\tightlist
\item
  Introducción al lenguaje SQL

  \begin{itemize}
  \tightlist
  \item
    Bases de datos relacionales
  \item
    Sintaxis SQL
  \item
    Conexión con bases de datos desde R
  \end{itemize}
\item
  Introducción a tecnologías NoSQL

  \begin{itemize}
  \tightlist
  \item
    Conceptos y tipos de bases de datos NoSQL (documental, columnar, clave/valor y de grafos)
  \item
    Conexión de R a NoSQL
  \end{itemize}
\item
  Tecnologías para el tratamiento de datos masivos

  \begin{itemize}
  \tightlist
  \item
    Introducción al Aprendizaje Estadístico
  \item
    Tecnologías Big Data (Hadoop, Spark, Sparklyr)
  \item
    Ejercicios de análisis de datos masivos.
  \end{itemize}
\end{enumerate}

\section{Planificación (tentativa)}\label{planificaciuxf3n-tentativa}

La impartición de los contenidos durante el curso dependerá de los conocimientos de partida y la asimilación de los
conceptos. Para completar nuestra visión de los conocimientos previos os requerimos completar este formulario en la primera sesión de clase: \url{https://forms.gle/D5bhiLLBUFuh6k1n8}

\begin{itemize}
\item
  Semana 1 (9/9) : 10 y 11/9 - Presentación e introducción a Tema 1 \& SQL
\item
  Semana 2 (16/9) : 17 y 18/9 - Seminario dplyr.
\item
  Semana 3 (23/9) : 24 y 25/9 - Tema 1 \& SQL.
\item
  Semana 4 (30/9) : 1 y 2/10 - Ejercicios SQL.
\item
  Semana 5 (7/10) : 8 y 9/10 - Tema 2 \& NoSQL - Seminario texto proc. (CSV, excel, Json) y open data .
\item
  Semana 6 (14/10) : 15/10 - Tutorial sparklyr-SQL. + 16/10 Consultas SQL con sparklyr.
\item
  Semana 7 (21/10) : 22 y 23/10 Tema 3 Big Data.
\item
  Semana 8 (28/10) : 29 y 30/10 - Tutoriales sparklyr-ML.
\item
  Semana 9 (4/11) : 6/11 - Ejercicios ML.
\item
  Semana 10(11/11) : 11 y 13/11 - Intro a AE (día 13 1 hora de 13 a 14h).
\item
  Semana 11(18/11) : 18/11 - Intro a AE Manuel, 20 y 21/11 - Ejercicios ML.
\item
  Semana 12 (25/11): 25/11 o cualquier otro día hasta el 5/12, seguramente el 5/12, dudas práctica ML. Backup.
\end{itemize}

Examen 21/1/25 4pm.

Recuperación 1/7/25 4pm.

\subsection{Evaluación}\label{evaluaciuxf3n}

\begin{itemize}
\item
  \textbf{Examen} (60\%): El examen de la materia evaluará los siguientes aspectos:
  Conceptos de la materia: Dominio de los conocimientos teóricos y operativos de la materia.
  Asimilación práctica de materia: Asimilación y comprensión de los conocimientos teóricos y operativos de la materia.
\item
  \textbf{Prácticas de laboratorio} (40\%): Evaluación de las prácticas de laboratorio desarrolladas por los estudiantes.
\end{itemize}

\section{Fuentes de información:}\label{fuentes-de-informaciuxf3n}

\subsection{Básica}\label{buxe1sica}

\begin{itemize}
\item
  Daroczi, G. (2015). Mastering Data Analysis with R. Packt Publishing
\item
  Grolemund, G. y Wickham, H. (2016). \href{https://r4ds.had.co.nz/}{R for Data Science} O'Reilly
\item
  Silberschatz, A., Korth, H. y Sudarshan, S. (2014). Fundamentos de Bases de Datos. Mc Graw Hill
\item
  Rubén Fernández Casal y Julián Costa Bouza. \href{https://rubenfcasal.github.io/aprendizaje_estadistico/}{Apuntes de Aprendizaje Estadístico}
\item
  Luraschi, J., Kuo, K., Ruiz, K. \href{https://therinspark.com/}{Mastering Spark with R} O'Reilly
\item
  Rubén Fernández Casal (\href{https://rubenfcasal.github.io}{R Machinery}):

  \begin{itemize}
  \item
    \href{https://rubenfcasal.github.io/intror}{Introducción al Análisis de Datos con R}
    (con Javier Roca y Julián Costa)
  \item
    \href{https://rubenfcasal.github.io/post/ayuda-y-recursos-para-el-aprendizaje-de-r}{Ayuda y Recursos para el Aprendizaje de R}
  \item
    \href{https://rubenfcasal.github.io/bookdown_intro}{Escritura de libros con el paquete bookdown}
    (con Tomás Cotos)
  \item
    \href{https://rubenfcasal.github.io/bookdown_intro/rmarkdown.html}{Apéndice introducción a Rmarkdown}
  \item
    \href{https://rubenfcasal.github.io/post/presentaciones/AnalisisDatosR.pdf}{Pesentación análisis de datos con R}
  \end{itemize}
\end{itemize}

\subsection{Complementaria:}\label{complementaria}

\begin{itemize}
\tightlist
\item
  Wes McKinney (2017). Python for Data Analysis: Data Wrangling with Pandas, NumPy, and IPython. O'Reilly (2ª ed.)
\item
  Tom White (2015). Hadoop: The Definitive Guide. O'Reilly (4ª ed.)
\item
  Alex Holmes (2014). Hadoop in practice. Manning (2ª ed.)
\item
  Centro de Supercomputación de Galicia (2020). {[}Servicio de Big Data del CESGA{]} (\url{https://bigdata.cesga.es/})
\end{itemize}

\chapter{Manipulación de datos con R}\label{manipR}

En el proceso de análisis de datos, al margen de su obtención y organización, una de las primeras etapas es el acceso y la manipulación de los datos (ver Figura \ref{fig:esquema2}).
En este capítulo se repasarán brevemente las principales herramientas disponibles en el paquete base de R para ello.
Posteriormente en el Capítulo \ref{tidyverse} se mostrará como alternativa el uso del paquete \href{https://dplyr.tidyverse.org/index.html}{\texttt{dplyr}}.

\begin{figure}[!htb]

{\centering \includegraphics[width=0.8\linewidth]{images/esquema2} 

}

\caption{Etapas del proceso}\label{fig:esquema2}
\end{figure}

\section{Lectura, importación y exportación de datos}\label{read}

Además de la introducción directa, R es capaz de
importar datos externos en múltiples formatos:

\begin{itemize}
\item
  bases de datos disponibles en librerías de R
\item
  archivos de texto en formato ASCII
\item
  archivos en otros formatos: Excel, SPSS, Matlab\ldots{}
\item
  bases de datos relacionales: MySQL, Oracle\ldots{}
\item
  formatos web: HTML, XML, JSON\ldots{}
\item
  otros lenguajes de programación: Python, Julia\ldots{}
\end{itemize}

\subsection{Formato de datos de R}\label{formato-de-datos-de-r}

El formato de archivo en el que habitualmente se almacena objetos (datos)
R es binario y está comprimido (en formato \texttt{"gzip"} por defecto).
Para cargar un fichero de datos se emplea normalmente \href{https://www.rdocumentation.org/packages/base/versions/3.6.1/topics/load}{\texttt{load()}}.
A continuación se utiliza el fichero \texttt{empleados.RData} que contiene datos de empleados de un banco.

\begin{Shaded}
\begin{Highlighting}[]
\NormalTok{res }\OtherTok{\textless{}{-}} \FunctionTok{load}\NormalTok{(}\StringTok{"data/empleados.RData"}\NormalTok{)}
\NormalTok{res}
\end{Highlighting}
\end{Shaded}

\begin{verbatim}
## [1] "empleados"
\end{verbatim}

\begin{Shaded}
\begin{Highlighting}[]
\FunctionTok{ls}\NormalTok{()}
\end{Highlighting}
\end{Shaded}

\begin{verbatim}
##  [1] "cite_cran" "cite_fig"  "cite_fig2" "cite_fun" 
##  [5] "cite_fun_" "cite_pkg"  "cite_pkg_" "citefig"  
##  [9] "citefig2"  "empleados" "fig.path"  "inline"   
## [13] "inline2"   "is_html"   "is_latex"  "latexfig" 
## [17] "latexfig2" "res"
\end{verbatim}

y para guardar \href{https://www.rdocumentation.org/packages/base/versions/3.6.1/topics/save}{\texttt{save()}}:

\begin{Shaded}
\begin{Highlighting}[]
\CommentTok{\# Guardar}
\FunctionTok{save}\NormalTok{(empleados, }\AttributeTok{file =} \StringTok{"data/empleados\_new.RData"}\NormalTok{)}
\end{Highlighting}
\end{Shaded}

Aunque, como indica este comando en la ayuda (\texttt{?save}):

\begin{quote}
\emph{For saving single R objects, \href{https://www.rdocumentation.org/packages/base/versions/3.6.1/topics/saveRDS}{\texttt{saveRDS()}}}
\emph{is mostly preferable to save(),}
\emph{notably because of the functional nature of readRDS(), as opposed to load().}
\end{quote}

\begin{Shaded}
\begin{Highlighting}[]
\FunctionTok{saveRDS}\NormalTok{(empleados, }\AttributeTok{file =} \StringTok{"data/empleados\_new.rds"}\NormalTok{)}
\DocumentationTok{\#\# restore it under a different name}
\NormalTok{empleados2 }\OtherTok{\textless{}{-}} \FunctionTok{readRDS}\NormalTok{(}\StringTok{"data/empleados\_new.rds"}\NormalTok{)}
\CommentTok{\# identical(empleados, empleados2)}
\end{Highlighting}
\end{Shaded}

Normalmente, el objeto empleado en R para almacenar datos en memoria
es el \href{https://www.rdocumentation.org/packages/base/versions/3.6.1/topics/data.frame}{\texttt{data.frame}}.

\subsection{Acceso a datos en paquetes}\label{acceso-a-datos-en-paquetes}

R dispone de múltiples conjuntos de datos en distintos paquetes, especialmente en el paquete \texttt{datasets}
que se carga por defecto al abrir R.
Con el comando \texttt{data()} podemos obtener un listado de las bases de datos disponibles.

Para cargar una base de datos concreta se utiliza el comando
\texttt{data(nombre)} (aunque en algunos casos se cargan automáticamente al emplearlos).
Por ejemplo, \texttt{data(cars)} carga la base de datos llamada \texttt{cars} en el entorno de trabajo (\texttt{".GlobalEnv"})
y \texttt{?cars} muestra la ayuda correspondiente con la descripición de la base de datos.

\subsection{Lectura de archivos de texto}\label{cap2-texto}

En R, para leer archivos de texto se suele utilizar la función \texttt{read.table()}.
Suponinedo, por ejemplo, que en el directorio actual está el fichero
\emph{empleados.txt}. La lectura de este fichero vendría dada por el código:

\begin{Shaded}
\begin{Highlighting}[]
\CommentTok{\# Session \textgreater{} Set Working Directory \textgreater{} To Source...?}
\NormalTok{datos }\OtherTok{\textless{}{-}} \FunctionTok{read.table}\NormalTok{(}\AttributeTok{file =} \StringTok{"data/empleados.txt"}\NormalTok{, }\AttributeTok{header =} \ConstantTok{TRUE}\NormalTok{)}
\CommentTok{\# head(datos)}
\FunctionTok{str}\NormalTok{(datos)}
\end{Highlighting}
\end{Shaded}

\begin{verbatim}
## 'data.frame':    474 obs. of  10 variables:
##  $ id      : int  1 2 3 4 5 6 7 8 9 10 ...
##  $ sexo    : chr  "Hombre" "Hombre" "Mujer" "Mujer" ...
##  $ fechnac : chr  "2/3/1952" "5/23/1958" "7/26/1929" "4/15/1947" ...
##  $ educ    : int  15 16 12 8 15 15 15 12 15 12 ...
##  $ catlab  : chr  "Directivo" "Administrativo" "Administrativo" "Administrativo" ...
##  $ salario : num  57000 40200 21450 21900 45000 ...
##  $ salini  : int  27000 18750 12000 13200 21000 13500 18750 9750 12750 13500 ...
##  $ tiempemp: int  98 98 98 98 98 98 98 98 98 98 ...
##  $ expprev : int  144 36 381 190 138 67 114 0 115 244 ...
##  $ minoria : chr  "No" "No" "No" "No" ...
\end{verbatim}

\begin{Shaded}
\begin{Highlighting}[]
\FunctionTok{class}\NormalTok{(datos)}
\end{Highlighting}
\end{Shaded}

\begin{verbatim}
## [1] "data.frame"
\end{verbatim}

Si el fichero estuviese en el directorio \emph{c:\textbackslash datos} bastaría con especificar
\texttt{file\ =\ "c:/datos/empleados.txt"}.
Nótese también que para la lectura del fichero anterior se ha
establecido el argumento \texttt{header=TRUE} para indicar que la primera línea del
fichero contiene los nombres de las variables.

Los argumentos utilizados habitualmente para esta función son:

\begin{itemize}
\item
  \texttt{header}: indica si el fichero tiene cabecera (\texttt{header=TRUE}) o no
  (\texttt{header=FALSE}). Por defecto toma el valor \texttt{header=FALSE}.
\item
  \texttt{sep}: carácter separador de columnas que por defecto es un espacio
  en blanco (\texttt{sep=""}). Otras opciones serían: \texttt{sep=","} si el separador es
  un ``;'', \texttt{sep="*"} si el separador es un ``*'', etc.
\item
  \texttt{dec}: carácter utilizado en el fichero para los números decimales.
  Por defecto se establece \texttt{dec\ =\ "."}. Si los decimales vienen dados
  por ``,'' se utiliza \texttt{dec\ =\ ","}.
\end{itemize}

Resumiendo, los (principales) argumentos por defecto de la función
\texttt{read.table} son los que se muestran en la siguiente línea:

\begin{Shaded}
\begin{Highlighting}[]
\FunctionTok{read.table}\NormalTok{(file, }\AttributeTok{header =} \ConstantTok{FALSE}\NormalTok{, }\AttributeTok{sep =} \StringTok{""}\NormalTok{, }\AttributeTok{dec =} \StringTok{"."}\NormalTok{)  }
\end{Highlighting}
\end{Shaded}

Para más detalles sobre esta función véase
\texttt{help(read.table)}.

Estan disponibles otras funciones con valores por defecto de los parámetros
adecuados para otras situaciones. Por ejemplo, para ficheros separados por tabuladores
se puede utilizar \texttt{read.delim()} o \texttt{read.delim2()}:

\begin{Shaded}
\begin{Highlighting}[]
\FunctionTok{read.delim}\NormalTok{(file, }\AttributeTok{header =} \ConstantTok{TRUE}\NormalTok{, }\AttributeTok{sep =} \StringTok{"}\SpecialCharTok{\textbackslash{}t}\StringTok{"}\NormalTok{, }\AttributeTok{dec =} \StringTok{"."}\NormalTok{)}
\FunctionTok{read.delim2}\NormalTok{(file, }\AttributeTok{header =} \ConstantTok{TRUE}\NormalTok{, }\AttributeTok{sep =} \StringTok{"}\SpecialCharTok{\textbackslash{}t}\StringTok{"}\NormalTok{, }\AttributeTok{dec =} \StringTok{","}\NormalTok{)}
\end{Highlighting}
\end{Shaded}

\subsection{Importación desde SPSS}\label{importaciuxf3n-desde-spss}

El programa R permite lectura de ficheros de datos en formato SPSS (extensión \emph{.sav}) sin necesidad de tener instalado dicho programa en el ordenador. Para ello se necesita:

\begin{itemize}
\item
  cargar la librería \texttt{foreign}
\item
  utilizar la función \texttt{read.spss}
\end{itemize}

Por ejemplo:

\small

\begin{Shaded}
\begin{Highlighting}[]
\FunctionTok{library}\NormalTok{(foreign)}
\NormalTok{datos }\OtherTok{\textless{}{-}} \FunctionTok{read.spss}\NormalTok{(}\AttributeTok{file =} \StringTok{"data/Employee data.sav"}\NormalTok{, }
                   \AttributeTok{to.data.frame =} \ConstantTok{TRUE}\NormalTok{)}
\CommentTok{\# head(datos)}
\FunctionTok{str}\NormalTok{(datos)}
\end{Highlighting}
\end{Shaded}

\begin{verbatim}
## 'data.frame':    474 obs. of  10 variables:
##  $ id      : num  1 2 3 4 5 6 7 8 9 10 ...
##  $ sexo    : Factor w/ 2 levels "Hombre","Mujer": 1 1 2 2 1 1 1 2 2 2 ...
##  $ fechnac : num  1.17e+10 1.19e+10 1.09e+10 1.15e+10 1.17e+10 ...
##  $ educ    : Factor w/ 10 levels "8","12","14",..: 4 5 2 1 4 4 4 2 4 2 ...
##  $ catlab  : Factor w/ 3 levels "Administrativo",..: 3 1 1 1 1 1 1 1 1 1 ...
##  $ salario : Factor w/ 221 levels "15750","15900",..: 179 137 28 31 150 101 121 31 71 45 ...
##  $ salini  : Factor w/ 90 levels "9000","9750",..: 60 42 13 21 48 23 42 2 18 23 ...
##  $ tiempemp: Factor w/ 36 levels "63","64","65",..: 36 36 36 36 36 36 36 36 36 36 ...
##  $ expprev : Factor w/ 208 levels "Ausente","10",..: 38 131 139 64 34 181 13 1 14 91 ...
##  $ minoria : Factor w/ 2 levels "No","Sí": 1 1 1 1 1 1 1 1 1 1 ...
##  - attr(*, "variable.labels")= Named chr [1:10] "Código de empleado" "Sexo" "Fecha de nacimiento" "Nivel educativo" ...
##   ..- attr(*, "names")= chr [1:10] "id" "sexo" "fechnac" "educ" ...
##  - attr(*, "codepage")= int 1252
\end{verbatim}

\normalsize

\textbf{Nota}: Si hay fechas, puede ser recomendable emplear la función \texttt{spss.get()} del paquete \texttt{Hmisc}.

\subsection{Importación desde Excel}\label{importaciuxf3n-desde-excel}

Se pueden leer fichero de Excel (con extensión \emph{.xlsx}) utilizando, por ejemplo, los paquetes:

\begin{itemize}
\tightlist
\item
  \href{https://cran.r-project.org/web/packages/openxlsx/index.html}{\texttt{openxlsx}},
\end{itemize}

\begin{Shaded}
\begin{Highlighting}[]
\FunctionTok{library}\NormalTok{(openxlsx)}
\NormalTok{datos}\OtherTok{\textless{}{-}}\FunctionTok{read.xlsx}\NormalTok{(}\StringTok{"./data/coches.xlsx"}\NormalTok{)}
\FunctionTok{class}\NormalTok{(datos)}
\end{Highlighting}
\end{Shaded}

\begin{verbatim}
## [1] "data.frame"
\end{verbatim}

\begin{itemize}
\tightlist
\item
  \href{https://cran.r-project.org/web/packages/RODBC/index.html}{\texttt{RODBC}} (este paquete se empleará más adelante para acceder a bases de datos),
  entre otros.
\end{itemize}

El siguiente código implementa una función que permite leer todos
los archivos en formato \emph{.xlsx} en un directorio:

\begin{Shaded}
\begin{Highlighting}[]
\FunctionTok{library}\NormalTok{(openxlsx)}
\NormalTok{read\_xlsx }\OtherTok{\textless{}{-}} \ControlFlowTok{function}\NormalTok{(}\AttributeTok{path =} \StringTok{\textquotesingle{}.\textquotesingle{}}\NormalTok{) \{}
\NormalTok{  files }\OtherTok{\textless{}{-}} \FunctionTok{dir}\NormalTok{(path, }\AttributeTok{pattern =} \StringTok{\textquotesingle{}*.xlsx\textquotesingle{}}\NormalTok{) }\CommentTok{\# list.files}
  \CommentTok{\# file.list \textless{}{-} lapply(files, readWorkbook)}
\NormalTok{  file.list }\OtherTok{\textless{}{-}} \FunctionTok{vector}\NormalTok{(}\FunctionTok{length}\NormalTok{(files), }\AttributeTok{mode =} \StringTok{\textquotesingle{}list\textquotesingle{}}\NormalTok{)}
  \ControlFlowTok{for}\NormalTok{ (i }\ControlFlowTok{in} \FunctionTok{seq\_along}\NormalTok{(files)) }
\NormalTok{      file.list[[i]] }\OtherTok{\textless{}{-}} \FunctionTok{readWorkbook}\NormalTok{(files[i])}
\NormalTok{  file.names }\OtherTok{\textless{}{-}} \FunctionTok{sub}\NormalTok{(}\StringTok{\textquotesingle{}}\SpecialCharTok{\textbackslash{}\textbackslash{}}\StringTok{.xlsx$\textquotesingle{}}\NormalTok{, }\StringTok{\textquotesingle{}\textquotesingle{}}\NormalTok{, }\FunctionTok{basename}\NormalTok{(files)) }
  \FunctionTok{names}\NormalTok{(file.list) }\OtherTok{\textless{}{-}}\NormalTok{ file.names}
\NormalTok{  file.list}
\NormalTok{\}}
\end{Highlighting}
\end{Shaded}

Para combinar los archivos, suponiendo que tienen las mismas columnas, podríamos ejecutar una llamada a \href{https://www.rdocumentation.org/packages/base/versions/3.6.1/topics/rbind}{\texttt{rbind()}}(R base):

\begin{Shaded}
\begin{Highlighting}[]
\NormalTok{df }\OtherTok{\textless{}{-}} \FunctionTok{do.call}\NormalTok{(}\StringTok{\textquotesingle{}rbind\textquotesingle{}}\NormalTok{, file.list)}
\end{Highlighting}
\end{Shaded}

o emplear la función \href{https://www.rdocumentation.org/packages/dplyr/versions/0.7.8/topics/bind}{\texttt{bind\_rows()}}
del paquete \href{https://dplyr.tidyverse.org}{\texttt{dplyr}}, donde las columnas se emparejan por nombre, y cualquier columna que falte se rellenará con \texttt{NA}:

\begin{Shaded}
\begin{Highlighting}[]
\NormalTok{df }\OtherTok{\textless{}{-}}\NormalTok{ dplyr}\SpecialCharTok{::}\FunctionTok{bind\_rows}\NormalTok{(file.list)}
\end{Highlighting}
\end{Shaded}

El Capítulo 4, provee de otras utilidades para la manipulación de datos con \texttt{dplyr} \citep{R-dplyr}.

Los datos cargados en R (usualmente un \texttt{data.frame}) se pueden exportar desde Excel fácilmente a un archivo de texto \emph{separado por comas} (extensión \emph{.csv}), evitando utilizar algunos de los paquetes mencionados anteriormente.
Por ejemplo, supongamos que queremos leer el fichero \emph{coches.xls}:

\begin{itemize}
\item
  Desde Excel, se selecciona el menú \texttt{Archivo\ -\textgreater{}\ Guardar\ como\ -\textgreater{}\ Guardar\ como}, y en \texttt{Tipo}, se escoge la opción de archivo CSV. De esta forma se guardarán los datos en el archivo \emph{coches.csv}.
\item
  El fichero \emph{coches.csv} es un fichero de texto plano (se puede
  editar con el bloc de notas, \emph{Notepad}), con cabecera, las columnas separadas por ``;'', y siendo ``,'' el carácter decimal.
\item
  Por lo tanto, la lectura de este fichero se puede hacer con:

\begin{Shaded}
\begin{Highlighting}[]
\NormalTok{datos }\OtherTok{\textless{}{-}} \FunctionTok{read.table}\NormalTok{(}\StringTok{"coches.csv"}\NormalTok{, }\AttributeTok{header =} \ConstantTok{TRUE}\NormalTok{, }
                    \AttributeTok{sep =} \StringTok{";"}\NormalTok{, }\AttributeTok{dec =} \StringTok{","}\NormalTok{)}
\end{Highlighting}
\end{Shaded}
\end{itemize}

Otra posibilidad, es utilizar la función \texttt{read.csv2}. Esta función no es más que una adaptación de la función general \texttt{read.table} con las siguientes
opciones:

\begin{Shaded}
\begin{Highlighting}[]
\FunctionTok{read.csv2}\NormalTok{(file, }\AttributeTok{header =} \ConstantTok{TRUE}\NormalTok{, }\AttributeTok{sep =} \StringTok{";"}\NormalTok{, }\AttributeTok{dec =} \StringTok{","}\NormalTok{, ...)}
\end{Highlighting}
\end{Shaded}

Por lo tanto, la lectura del fichero \emph{coches.csv} se puede hacer de modo
más directo con:

\begin{Shaded}
\begin{Highlighting}[]
\NormalTok{datos }\OtherTok{\textless{}{-}} \FunctionTok{read.csv2}\NormalTok{(}\StringTok{"coches.csv"}\NormalTok{)}
\end{Highlighting}
\end{Shaded}

Esta forma de proceder, exportando a formato CSV, se puede emplear con otras hojas de cálculo o fuentes de datos.
Hay que tener en cuenta que si estas fuentes emplean el formato anglosajón, el separador de campos será \texttt{sep\ =\ ","} y el de decimales \texttt{dec\ =\ "."}, las opciones por defecto en la función \texttt{read.csv()}.

\subsection{Exportación de datos}\label{cap2-exporta}

Puede ser de interés la exportacifn de datos para que puedan ser leídos con otros programas. Para ello, se puede emplear la función \texttt{write.table()}. Esta función es similar, pero funcionando en sentido inverso, a \texttt{read.table()}, ver Sección \ref{cap2-texto}.

Veamos un ejemplo:

\begin{Shaded}
\begin{Highlighting}[]
\NormalTok{tipo }\OtherTok{\textless{}{-}} \FunctionTok{c}\NormalTok{(}\StringTok{"A"}\NormalTok{, }\StringTok{"B"}\NormalTok{, }\StringTok{"C"}\NormalTok{)}
\NormalTok{longitud }\OtherTok{\textless{}{-}} \FunctionTok{c}\NormalTok{(}\FloatTok{120.34}\NormalTok{, }\FloatTok{99.45}\NormalTok{, }\FloatTok{115.67}\NormalTok{)}
\NormalTok{datos }\OtherTok{\textless{}{-}} \FunctionTok{data.frame}\NormalTok{(tipo, longitud)}
\NormalTok{datos}
\end{Highlighting}
\end{Shaded}

\begin{verbatim}
##   tipo longitud
## 1    A   120.34
## 2    B    99.45
## 3    C   115.67
\end{verbatim}

Para guardar el data.frame \texttt{datos} en un fichero de texto se
puede utilizar:

\begin{Shaded}
\begin{Highlighting}[]
\FunctionTok{write.table}\NormalTok{(datos, }\AttributeTok{file =} \StringTok{"datos.txt"}\NormalTok{)}
\end{Highlighting}
\end{Shaded}

Otra posibilidad es utilizar la función:

\begin{Shaded}
\begin{Highlighting}[]
\FunctionTok{write.csv2}\NormalTok{(datos, }\AttributeTok{file =} \StringTok{"datos.csv"}\NormalTok{)}
\end{Highlighting}
\end{Shaded}

que dará lugar al fichero \emph{datos.csv} importable directamente desde Excel. Las opciones anteriores sólo dependen del paquete \texttt{utils}, que se instala por defecto con R base.

\subsection{Python, Julia y otros lenguajes de programación}\label{python-julia-y-otros-lenguajes-de-programaciuxf3n}

R es un lenguaje de programación libre (derivado del lenguaje S en los Laboratorios Bell) que se caracteriza por su capacidad para interactuar con otros lenguajes de programación, incluyendo Python \citep{python} y Julia \citep{julia}.

En el ámbito de la Estadística (como en la denominada \textbf{Ciendica de Datos}), R destaca por su extensa y detallada documentación (en muchos casos como resultado de aportaciones metodológicas y/o avances científicos). Por ejemplo, después de diez años de la primera edición del libro \emph{An Introduction to Statistical Learning con aplicaciones en R (ISLR)} , \citet{james2013introduction}, algunos de los mismos autores publicaron la edición en Python (ISLP), \citet{james2023introduction}.\\
Por otro lado, en 2015, se lanzó el paquete \href{https://rstudio.github.io/reticulate/}{\texttt{reticulate}} disponible en \url{https://rstudio.github.io/reticulate/}, permitiendo la ejecución de código Python desde R (y en 2020 se completó la integración de Python en la interfaz de RStudio).

\begin{Shaded}
\begin{Highlighting}[]
\FunctionTok{library}\NormalTok{(reticulate)}
\NormalTok{os }\OtherTok{\textless{}{-}} \FunctionTok{import}\NormalTok{(}\StringTok{"os"}\NormalTok{)}
\NormalTok{os}\SpecialCharTok{$}\FunctionTok{listdir}\NormalTok{(}\StringTok{"."}\NormalTok{)}
\end{Highlighting}
\end{Shaded}

Si queremos trabajar con Python de forma interactiva, podemos usar \texttt{repl\_python()}. Los objetos creados en Python se pueden usar en R con \texttt{py} de \texttt{reticulate}.

Recientemente, \emph{Julia} se presenta también como una alternativa a considerar.
El paquete \href{NA}{\texttt{JuliaConnectoR}} disponible en \url{https://cran.r-project.org/web/packages/JuliaConnectoR/} facilita la importación de funciones y paquetes completos de Julia a R, es decir, permite el uso de funciones de Julia directamente en R.

R también permite el uso/comunicación de otros lenguajes de programación como Java, C, C++, Fortran, entre otros.

\section{Manipulación de datos}\label{manipulaciuxf3n-de-datos}

Una vez cargada una (o varias) bases de datos hay una series de operaciones que serán de interés para el tratamiento de datos:

\begin{itemize}
\tightlist
\item
  Operaciones con variables:

  \begin{itemize}
  \tightlist
  \item
    crear
  \item
    recodificar (e.g.~categorizar)
  \item
    \ldots{}
  \end{itemize}
\item
  Operaciones con casos:

  \begin{itemize}
  \tightlist
  \item
    ordenar
  \item
    filtrar
  \item
    \ldots{}
  \end{itemize}
\item
  Operaciones con tablas de datos:

  \begin{itemize}
  \tightlist
  \item
    unir
  \item
    combinar
  \item
    consultar
  \item
    \ldots{}
  \end{itemize}
\end{itemize}

A continuación se tratan algunas operaciones \emph{básicas}.

\subsection{Operaciones con variables}\label{operaciones-con-variables}

\subsubsection{Creación (y eliminación) de variables}\label{creaciuxf3n-y-eliminaciuxf3n-de-variables}

Consideremos de nuevo la base de datos \texttt{cars} incluida en el paquete \texttt{datasets}:

\begin{Shaded}
\begin{Highlighting}[]
\FunctionTok{data}\NormalTok{(cars)}
\CommentTok{\# str(cars)}
\FunctionTok{head}\NormalTok{(cars)}
\end{Highlighting}
\end{Shaded}

\begin{verbatim}
##   speed dist
## 1     4    2
## 2     4   10
## 3     7    4
## 4     7   22
## 5     8   16
## 6     9   10
\end{verbatim}

Utilizando el comando \texttt{help(cars)} se obtiene que \texttt{cars} es un data.frame con 50
observaciones y dos variables:

\begin{itemize}
\item
  \texttt{speed}: Velocidad (en millas por hora)
\item
  \texttt{dist}: tiempo hasta detenerse (en pies)
\end{itemize}

Recordemos que, para acceder a la variable \texttt{speed} se puede
hacer directamente con su nombre o bien utilizando notación
``matricial'' (se seleccionan las 6 primeras observaciones por comodidad).

\begin{Shaded}
\begin{Highlighting}[]
\NormalTok{cars}\SpecialCharTok{$}\NormalTok{speed}
\end{Highlighting}
\end{Shaded}

\begin{verbatim}
##  [1]  4  4  7  7  8  9 10 10 10 11 11 12 12 12 12 13 13 13
## [19] 13 14 14 14 14 15 15 15 16 16 17 17 17 18 18 18 18 19
## [37] 19 19 20 20 20 20 20 22 23 24 24 24 24 25
\end{verbatim}

\begin{Shaded}
\begin{Highlighting}[]
\CommentTok{\# cars[, 1]       \# Equivalente}
\CommentTok{\# cars[,"speed"]  \# Equivalente}
\end{Highlighting}
\end{Shaded}

Supongamos ahora que queremos transformar la variable original \texttt{speed}
(millas por hora) en una nueva variable \texttt{velocidad} (kilómetros por
hora) y añadir esta nueva variable al data.frame \texttt{cars}.
La transformación que permite pasar millas a kilómetros es
\texttt{kilómetros=millas/0.62137} que en R se hace directamente con:

\begin{Shaded}
\begin{Highlighting}[]
\NormalTok{(cars}\SpecialCharTok{$}\NormalTok{speed}\SpecialCharTok{/}\FloatTok{0.62137}\NormalTok{)[}\DecValTok{1}\SpecialCharTok{:}\DecValTok{10}\NormalTok{]}
\end{Highlighting}
\end{Shaded}

Finalmente, incluimos la nueva variable que llamaremos
\texttt{velocidad} en \texttt{cars}:

\begin{Shaded}
\begin{Highlighting}[]
\NormalTok{cars}\SpecialCharTok{$}\NormalTok{velocidad }\OtherTok{\textless{}{-}}\NormalTok{ cars}\SpecialCharTok{$}\NormalTok{speed }\SpecialCharTok{/} \FloatTok{0.62137}
\FunctionTok{head}\NormalTok{(cars)}
\end{Highlighting}
\end{Shaded}

\begin{verbatim}
##   speed dist velocidad
## 1     4    2  6.437388
## 2     4   10  6.437388
## 3     7    4 11.265430
## 4     7   22 11.265430
## 5     8   16 12.874777
## 6     9   10 14.484124
\end{verbatim}

También transformaremos la variable \texttt{dist} (en pies) en una nueva
variable \texttt{distancia} (en metros), por lo que la transformación deseada es
\texttt{metros=pies/3.2808}:

\begin{Shaded}
\begin{Highlighting}[]
\NormalTok{cars}\SpecialCharTok{$}\NormalTok{distancia }\OtherTok{\textless{}{-}}\NormalTok{ cars}\SpecialCharTok{$}\NormalTok{dis }\SpecialCharTok{/} \FloatTok{3.2808}
\FunctionTok{head}\NormalTok{(cars)}
\end{Highlighting}
\end{Shaded}

\begin{verbatim}
##   speed dist velocidad distancia
## 1     4    2  6.437388 0.6096074
## 2     4   10  6.437388 3.0480371
## 3     7    4 11.265430 1.2192148
## 4     7   22 11.265430 6.7056815
## 5     8   16 12.874777 4.8768593
## 6     9   10 14.484124 3.0480371
\end{verbatim}

Ahora, eliminaremos las variables originales \texttt{speed} y
\texttt{dist}, y guardaremos el data.frame resultante con el nombre \texttt{coches}.
En primer lugar, veamos varias formas de acceder a las variables de
interés:

\begin{Shaded}
\begin{Highlighting}[]
\NormalTok{cars[, }\FunctionTok{c}\NormalTok{(}\DecValTok{3}\NormalTok{, }\DecValTok{4}\NormalTok{)]}
\NormalTok{cars[, }\FunctionTok{c}\NormalTok{(}\StringTok{"velocidad"}\NormalTok{, }\StringTok{"distancia"}\NormalTok{)]}
\NormalTok{cars[, }\SpecialCharTok{{-}}\FunctionTok{c}\NormalTok{(}\DecValTok{1}\NormalTok{, }\DecValTok{2}\NormalTok{)]}
\end{Highlighting}
\end{Shaded}

Utilizando alguna de las opciones anteriores se obtiene el \texttt{data.frame}
deseado:

\begin{Shaded}
\begin{Highlighting}[]
\NormalTok{coches }\OtherTok{\textless{}{-}}\NormalTok{ cars[, }\FunctionTok{c}\NormalTok{(}\StringTok{"velocidad"}\NormalTok{, }\StringTok{"distancia"}\NormalTok{)]}
\CommentTok{\# head(coches)}
\FunctionTok{str}\NormalTok{(coches)}
\end{Highlighting}
\end{Shaded}

\begin{verbatim}
## 'data.frame':    50 obs. of  2 variables:
##  $ velocidad: num  6.44 6.44 11.27 11.27 12.87 ...
##  $ distancia: num  0.61 3.05 1.22 6.71 4.88 ...
\end{verbatim}

Finalmente, los datos anteriores podrían ser guardados en un fichero
exportable a Excel con el siguiente comando:

\begin{Shaded}
\begin{Highlighting}[]
\FunctionTok{write.csv2}\NormalTok{(coches, }\AttributeTok{file =} \StringTok{"coches.csv"}\NormalTok{)}
\end{Highlighting}
\end{Shaded}

\subsubsection{Recodificación de variables}\label{recodificaciuxf3n-de-variables}

Con el comando \texttt{cut()} podemos crear variables categóricas a partir de variables numéricas.
El parámetro \texttt{breaks} permite especificar los intervalos para la discretización, puede ser un vector con los extremos de los intervalos o un entero con el número de intervalos.
Por ejemplo, para categorizar la variable \texttt{cars\$speed} en tres intervalos equidistantes podemos emplear\footnote{Aunque si el objetivo es obtener las frecuencias de cada intervalo puede ser más eficiente emplear \texttt{hist()} con \texttt{plot\ =\ FALSE}.}:

\begin{Shaded}
\begin{Highlighting}[]
\NormalTok{fspeed }\OtherTok{\textless{}{-}} \FunctionTok{cut}\NormalTok{(cars}\SpecialCharTok{$}\NormalTok{speed, }\DecValTok{3}\NormalTok{, }\AttributeTok{labels =} \FunctionTok{c}\NormalTok{(}\StringTok{"Baja"}\NormalTok{, }\StringTok{"Media"}\NormalTok{, }\StringTok{"Alta"}\NormalTok{))}
\FunctionTok{table}\NormalTok{(fspeed)}
\end{Highlighting}
\end{Shaded}

\begin{verbatim}
## fspeed
##  Baja Media  Alta 
##    11    24    15
\end{verbatim}

Para categorizar esta variable en tres niveles con aproximadamente el mismo número de observaciones podríamos combinar esta función con \texttt{quantile()}:

\begin{Shaded}
\begin{Highlighting}[]
\NormalTok{breaks }\OtherTok{\textless{}{-}} \FunctionTok{quantile}\NormalTok{(cars}\SpecialCharTok{$}\NormalTok{speed, }\AttributeTok{probs =} \DecValTok{0}\SpecialCharTok{:}\DecValTok{3}\SpecialCharTok{/}\DecValTok{3}\NormalTok{)}
\NormalTok{etiquetas3 }\OtherTok{\textless{}{-}} \FunctionTok{c}\NormalTok{(}\StringTok{"Baja"}\NormalTok{, }\StringTok{"Media"}\NormalTok{, }\StringTok{"Alta"}\NormalTok{)}
\NormalTok{fspeed }\OtherTok{\textless{}{-}} \FunctionTok{cut}\NormalTok{(cars}\SpecialCharTok{$}\NormalTok{speed, breaks, }\AttributeTok{labels =}\NormalTok{ etiquetas3)}
\FunctionTok{table}\NormalTok{(fspeed)}
\end{Highlighting}
\end{Shaded}

\begin{verbatim}
## fspeed
##  Baja Media  Alta 
##    17    16    15
\end{verbatim}

Para otro tipo de recodificaciones podríamos emplear la función \texttt{ifelse()} vectorial:

\begin{Shaded}
\begin{Highlighting}[]
\NormalTok{fspeed }\OtherTok{\textless{}{-}} \FunctionTok{ifelse}\NormalTok{(cars}\SpecialCharTok{$}\NormalTok{speed }\SpecialCharTok{\textless{}} \DecValTok{15}\NormalTok{, }\StringTok{"Baja"}\NormalTok{, }\StringTok{"Alta"}\NormalTok{)}
\NormalTok{etiquetas2 }\OtherTok{\textless{}{-}} \FunctionTok{c}\NormalTok{(}\StringTok{"Baja"}\NormalTok{, }\StringTok{"Alta"}\NormalTok{)}
\NormalTok{fspeed }\OtherTok{\textless{}{-}} \FunctionTok{factor}\NormalTok{(fspeed, }\AttributeTok{levels =}\NormalTok{ etiquetas2)}
\FunctionTok{table}\NormalTok{(fspeed)}
\end{Highlighting}
\end{Shaded}

\begin{verbatim}
## fspeed
## Baja Alta 
##   23   27
\end{verbatim}

Alternativamente, en el caso de dos niveles podríamos emplear directamente la función \texttt{factor()}:

\begin{Shaded}
\begin{Highlighting}[]
\NormalTok{fspeed }\OtherTok{\textless{}{-}} \FunctionTok{factor}\NormalTok{(cars}\SpecialCharTok{$}\NormalTok{speed }\SpecialCharTok{\textgreater{}=} \DecValTok{15}\NormalTok{, }
                 \AttributeTok{labels =}\NormalTok{ etiquetas2) }\CommentTok{\# levels = c("FALSE", "TRUE")}
\FunctionTok{table}\NormalTok{(fspeed)}
\end{Highlighting}
\end{Shaded}

\begin{verbatim}
## fspeed
## Baja Alta 
##   23   27
\end{verbatim}

En el caso de múltiples niveles, se podría emplear \texttt{ifelse()} anidados:

\begin{Shaded}
\begin{Highlighting}[]
\NormalTok{fspeed }\OtherTok{\textless{}{-}} \FunctionTok{ifelse}\NormalTok{(cars}\SpecialCharTok{$}\NormalTok{speed }\SpecialCharTok{\textless{}} \DecValTok{10}\NormalTok{, }\StringTok{"Baja"}\NormalTok{,}
                 \FunctionTok{ifelse}\NormalTok{(cars}\SpecialCharTok{$}\NormalTok{speed }\SpecialCharTok{\textless{}} \DecValTok{20}\NormalTok{, }\StringTok{"Media"}\NormalTok{, }\StringTok{"Alta"}\NormalTok{))}
\NormalTok{fspeed }\OtherTok{\textless{}{-}} \FunctionTok{factor}\NormalTok{(fspeed, }\AttributeTok{levels =}\NormalTok{ etiquetas3)}
\FunctionTok{table}\NormalTok{(fspeed)}
\end{Highlighting}
\end{Shaded}

\begin{verbatim}
## fspeed
##  Baja Media  Alta 
##     6    32    12
\end{verbatim}

Otra alternativa, sería emplear la función \href{https://www.rdocumentation.org/packages/car/versions/3.0-9/topics/recode}{\texttt{recode()}} del paquete \texttt{car}.

\begin{Shaded}
\begin{Highlighting}[]
\FunctionTok{library}\NormalTok{(car)}
\NormalTok{fspeed }\OtherTok{\textless{}{-}} \FunctionTok{recode}\NormalTok{(cars}\SpecialCharTok{$}\NormalTok{speed, }\StringTok{"0:10 = \textquotesingle{}Baja\textquotesingle{}; }
\StringTok{                 10:20 = \textquotesingle{}Media\textquotesingle{};}
\StringTok{                 else=\textquotesingle{}Alta\textquotesingle{}}
\StringTok{                 "}\NormalTok{)}
\NormalTok{fspeed }\OtherTok{\textless{}{-}} \FunctionTok{factor}\NormalTok{(fspeed, }\AttributeTok{levels =} \FunctionTok{c}\NormalTok{(}\StringTok{"Baja"}\NormalTok{, }\StringTok{"Media"}\NormalTok{, }\StringTok{"Alta"}\NormalTok{))}
\end{Highlighting}
\end{Shaded}

NOTA: Para acceder directamente a las variables de un \texttt{data.frame} podríamos emplear la función \texttt{attach()} para añadirlo a la ruta de búsqueda y \texttt{detach()} al finalizar.
Sin embargo esta forma de proceder puede causar numerosos inconvenientes, especialmente al modificar la base de datos, por lo que la recomendación sería emplear \texttt{with()}.
Por ejemplo, podríamos calcular el factor anterior empleando:

\begin{Shaded}
\begin{Highlighting}[]
\NormalTok{fspeed }\OtherTok{\textless{}{-}} \FunctionTok{with}\NormalTok{(cars, }\FunctionTok{ifelse}\NormalTok{(speed }\SpecialCharTok{\textless{}} \DecValTok{10}\NormalTok{, }\StringTok{"Baja"}\NormalTok{,}
                 \FunctionTok{ifelse}\NormalTok{(speed }\SpecialCharTok{\textless{}} \DecValTok{20}\NormalTok{, }\StringTok{"Media"}\NormalTok{, }\StringTok{"Alta"}\NormalTok{)))}
\NormalTok{fspeed }\OtherTok{\textless{}{-}} \FunctionTok{factor}\NormalTok{(fspeed, }\AttributeTok{levels =} \FunctionTok{c}\NormalTok{(}\StringTok{"Baja"}\NormalTok{, }\StringTok{"Media"}\NormalTok{, }\StringTok{"Alta"}\NormalTok{))}
\FunctionTok{table}\NormalTok{(fspeed)}
\end{Highlighting}
\end{Shaded}

\begin{verbatim}
## fspeed
##  Baja Media  Alta 
##     6    32    12
\end{verbatim}

\subsection{Operaciones con casos}\label{operaciones-con-casos}

\subsubsection{Ordenación}\label{ordenaciuxf3n}

Continuemos con el data.frame \texttt{cars}.
Se puede comprobar que los datos disponibles están ordenados por
los valores de \texttt{speed}. A continuación haremos la ordenación utilizando
los valores de \texttt{dist}. Para ello, utilizaremos el conocido como vector de
índices de ordenación.
Este vector establece el orden en que tienen que ser elegidos los
elementos para obtener la ordenación deseada.
Veamos primero un ejemplo sencillo:

\begin{Shaded}
\begin{Highlighting}[]
\NormalTok{x }\OtherTok{\textless{}{-}} \FunctionTok{c}\NormalTok{(}\FloatTok{2.5}\NormalTok{, }\FloatTok{4.3}\NormalTok{, }\FloatTok{1.2}\NormalTok{, }\FloatTok{3.1}\NormalTok{, }\FloatTok{5.0}\NormalTok{) }\CommentTok{\# valores originales}
\NormalTok{ii }\OtherTok{\textless{}{-}} \FunctionTok{order}\NormalTok{(x)}
\NormalTok{ii    }\CommentTok{\# vector de ordenación}
\end{Highlighting}
\end{Shaded}

\begin{verbatim}
## [1] 3 1 4 2 5
\end{verbatim}

\begin{Shaded}
\begin{Highlighting}[]
\NormalTok{x[ii] }\CommentTok{\# valores ordenados (por defecto, ascendentemente)}
\end{Highlighting}
\end{Shaded}

\begin{verbatim}
## [1] 1.2 2.5 3.1 4.3 5.0
\end{verbatim}

En el caso de vectores, el procedimiento anterior se podría
hacer directamente con:

\begin{Shaded}
\begin{Highlighting}[]
\FunctionTok{sort}\NormalTok{(x)}
\end{Highlighting}
\end{Shaded}

Sin embargo, para ordenar tablas de datos será necesario la utilización del
vector de índices de ordenación. A continuación, se muestan los datos de \texttt{cars} ordenados por \texttt{dist}:

\begin{Shaded}
\begin{Highlighting}[]
\NormalTok{ii }\OtherTok{\textless{}{-}} \FunctionTok{order}\NormalTok{(cars}\SpecialCharTok{$}\NormalTok{dist) }\CommentTok{\# Vector de índices de ordenación}
\NormalTok{cars2 }\OtherTok{\textless{}{-}}\NormalTok{ cars[ii, ]    }\CommentTok{\# Datos ordenados por dist}
\FunctionTok{head}\NormalTok{(cars2)}
\end{Highlighting}
\end{Shaded}

\begin{verbatim}
##    speed dist velocidad distancia
## 1      4    2  6.437388 0.6096074
## 3      7    4 11.265430 1.2192148
## 2      4   10  6.437388 3.0480371
## 6      9   10 14.484124 3.0480371
## 12    12   14 19.312165 4.2672519
## 5      8   16 12.874777 4.8768593
\end{verbatim}

\subsubsection{Filtrado}\label{filtrado}

El filtrado de datos consiste en elegir una submuestra que cumpla determinadas condiciones. Para ello, se puede utilizar la función \href{https://www.rdocumentation.org/packages/base/versions/3.6.1/topics/subset}{\texttt{subset(x,\ subset,\ select,\ drop\ =\ FALSE,\ ...)}} , que además permite seleccionar variables con el argumento \texttt{select}.

A continuación se muestran un par de ejemplos:

\begin{Shaded}
\begin{Highlighting}[]
\CommentTok{\# datos con dis\textgreater{}85}
\FunctionTok{subset}\NormalTok{(cars, dist }\SpecialCharTok{\textgreater{}} \DecValTok{85}\NormalTok{) }
\end{Highlighting}
\end{Shaded}

\begin{verbatim}
##    speed dist velocidad distancia
## 47    24   92  38.62433  28.04194
## 48    24   93  38.62433  28.34674
## 49    24  120  38.62433  36.57644
\end{verbatim}

\begin{Shaded}
\begin{Highlighting}[]
\CommentTok{\# datos con speed en (10,15) y dist \textgreater{} 45}
\FunctionTok{subset}\NormalTok{(cars, speed }\SpecialCharTok{\textgreater{}} \DecValTok{10} \SpecialCharTok{\&}\NormalTok{ speed }\SpecialCharTok{\textless{}} \DecValTok{15} \SpecialCharTok{\&}\NormalTok{ dist }\SpecialCharTok{\textgreater{}} \DecValTok{45}\NormalTok{)}
\end{Highlighting}
\end{Shaded}

\begin{verbatim}
##    speed dist velocidad distancia
## 19    13   46  20.92151  14.02097
## 22    14   60  22.53086  18.28822
## 23    14   80  22.53086  24.38430
\end{verbatim}

También se pueden hacer el filtrado empleando directamente los
correspondientes vectores de índices:

\begin{Shaded}
\begin{Highlighting}[]
\NormalTok{ii }\OtherTok{\textless{}{-}}\NormalTok{ cars}\SpecialCharTok{$}\NormalTok{dist }\SpecialCharTok{\textgreater{}} \DecValTok{85}
\NormalTok{cars[ii, ]   }\CommentTok{\# dis\textgreater{}85}
\end{Highlighting}
\end{Shaded}

\begin{verbatim}
##    speed dist velocidad distancia
## 47    24   92  38.62433  28.04194
## 48    24   93  38.62433  28.34674
## 49    24  120  38.62433  36.57644
\end{verbatim}

\begin{Shaded}
\begin{Highlighting}[]
\NormalTok{ii }\OtherTok{\textless{}{-}}\NormalTok{ cars}\SpecialCharTok{$}\NormalTok{speed }\SpecialCharTok{\textgreater{}} \DecValTok{10} \SpecialCharTok{\&}\NormalTok{ cars}\SpecialCharTok{$}\NormalTok{speed }\SpecialCharTok{\textless{}} \DecValTok{15} \SpecialCharTok{\&}\NormalTok{ cars}\SpecialCharTok{$}\NormalTok{dist }\SpecialCharTok{\textgreater{}} \DecValTok{45}
\NormalTok{cars[ii, ]  }\CommentTok{\# speed en (10,15) y dist\textgreater{}45}
\end{Highlighting}
\end{Shaded}

\begin{verbatim}
##    speed dist velocidad distancia
## 19    13   46  20.92151  14.02097
## 22    14   60  22.53086  18.28822
## 23    14   80  22.53086  24.38430
\end{verbatim}

En este caso, puede ser de utilidad la función \href{https://www.rdocumentation.org/packages/base/versions/3.6.1/topics/which}{\texttt{which()}}:

\begin{Shaded}
\begin{Highlighting}[]
\NormalTok{it }\OtherTok{\textless{}{-}} \FunctionTok{which}\NormalTok{(ii)}
\FunctionTok{str}\NormalTok{(it)}
\end{Highlighting}
\end{Shaded}

\begin{verbatim}
##  int [1:3] 19 22 23
\end{verbatim}

\begin{Shaded}
\begin{Highlighting}[]
\NormalTok{cars[it, ]}
\end{Highlighting}
\end{Shaded}

\begin{verbatim}
##    speed dist velocidad distancia
## 19    13   46  20.92151  14.02097
## 22    14   60  22.53086  18.28822
## 23    14   80  22.53086  24.38430
\end{verbatim}

\begin{Shaded}
\begin{Highlighting}[]
\CommentTok{\# rownames(cars[it, ])}
\NormalTok{id }\OtherTok{\textless{}{-}} \FunctionTok{which}\NormalTok{(}\SpecialCharTok{!}\NormalTok{ii)}
\FunctionTok{str}\NormalTok{(cars[id, ])}
\end{Highlighting}
\end{Shaded}

\begin{verbatim}
## 'data.frame':    47 obs. of  4 variables:
##  $ speed    : num  4 4 7 7 8 9 10 10 10 11 ...
##  $ dist     : num  2 10 4 22 16 10 18 26 34 17 ...
##  $ velocidad: num  6.44 6.44 11.27 11.27 12.87 ...
##  $ distancia: num  0.61 3.05 1.22 6.71 4.88 ...
\end{verbatim}

\begin{Shaded}
\begin{Highlighting}[]
\CommentTok{\# Equivalentemente:}
\FunctionTok{str}\NormalTok{(cars[}\SpecialCharTok{{-}}\NormalTok{it, ])}
\end{Highlighting}
\end{Shaded}

\begin{verbatim}
## 'data.frame':    47 obs. of  4 variables:
##  $ speed    : num  4 4 7 7 8 9 10 10 10 11 ...
##  $ dist     : num  2 10 4 22 16 10 18 26 34 17 ...
##  $ velocidad: num  6.44 6.44 11.27 11.27 12.87 ...
##  $ distancia: num  0.61 3.05 1.22 6.71 4.88 ...
\end{verbatim}

\begin{Shaded}
\begin{Highlighting}[]
\CommentTok{\# ?which.min}
\end{Highlighting}
\end{Shaded}

Si se realiza una selección de variables como en:

\begin{Shaded}
\begin{Highlighting}[]
\NormalTok{cars[ii, }\StringTok{"speed"}\NormalTok{]}
\end{Highlighting}
\end{Shaded}

\begin{verbatim}
## [1] 13 14 14
\end{verbatim}

es posible que se quiera mantener la estructura original de los datos, para ello,
bastaría con:

\begin{Shaded}
\begin{Highlighting}[]
\NormalTok{cars[ii, }\StringTok{"speed"}\NormalTok{, drop}\OtherTok{=}\ConstantTok{FALSE}\NormalTok{]}
\end{Highlighting}
\end{Shaded}

\begin{verbatim}
##    speed
## 19    13
## 22    14
## 23    14
\end{verbatim}

\begin{Shaded}
\begin{Highlighting}[]
\CommentTok{\# subset(cars, ii, "speed") \# equivalente}
\end{Highlighting}
\end{Shaded}

A veces puede ser necesario dividir (particionar) el conjunto de datos, uno para cada nivel de un grupo (factor), para ello se puede usar la función \texttt{split()}:

\begin{Shaded}
\begin{Highlighting}[]
\NormalTok{speed2 }\OtherTok{\textless{}{-}} \FunctionTok{factor}\NormalTok{(cars}\SpecialCharTok{$}\NormalTok{speed }\SpecialCharTok{\textgreater{}} \DecValTok{20}\NormalTok{, }\AttributeTok{labels =} \FunctionTok{c}\NormalTok{(}\StringTok{"slow"}\NormalTok{,}\StringTok{"fast"}\NormalTok{))}
\FunctionTok{table}\NormalTok{(speed2)}
\end{Highlighting}
\end{Shaded}

\begin{verbatim}
## speed2
## slow fast 
##   43    7
\end{verbatim}

\begin{Shaded}
\begin{Highlighting}[]
\NormalTok{cars2 }\OtherTok{\textless{}{-}} \FunctionTok{split}\NormalTok{(cars,speed2)}
\FunctionTok{class}\NormalTok{(cars2) }\CommentTok{\# lista con 2 data.frames}
\end{Highlighting}
\end{Shaded}

\begin{verbatim}
## [1] "list"
\end{verbatim}

\begin{Shaded}
\begin{Highlighting}[]
\FunctionTok{sapply}\NormalTok{(cars2,class)}
\end{Highlighting}
\end{Shaded}

\begin{verbatim}
##         slow         fast 
## "data.frame" "data.frame"
\end{verbatim}

\begin{Shaded}
\begin{Highlighting}[]
\FunctionTok{sapply}\NormalTok{(cars2,dim)}
\end{Highlighting}
\end{Shaded}

\begin{verbatim}
##      slow fast
## [1,]   43    7
## [2,]    4    4
\end{verbatim}

\begin{Shaded}
\begin{Highlighting}[]
\NormalTok{cars2}\SpecialCharTok{$}\NormalTok{fast}
\end{Highlighting}
\end{Shaded}

\begin{verbatim}
##    speed dist velocidad distancia
## 44    22   66  35.40564  20.11704
## 45    23   54  37.01498  16.45940
## 46    24   70  38.62433  21.33626
## 47    24   92  38.62433  28.04194
## 48    24   93  38.62433  28.34674
## 49    24  120  38.62433  36.57644
## 50    25   85  40.23368  25.90832
\end{verbatim}

De forma inversa, podríamos recuperar el data.frame original con:

\begin{Shaded}
\begin{Highlighting}[]
\FunctionTok{unsplit}\NormalTok{(cars2,speed2)}
\end{Highlighting}
\end{Shaded}

\section{Datos faltantes}\label{missing}

La problemática originada por los datos faltantes (\emph{missing data}) en cualquier conjunto de datos subyace cuando se desea
realizar un análisis estadístico, para más información en R, se puede consultar \href{https://cran.r-project.org/web/views/MissingData.html}{CRAN Task View: Missing Data}

Vamos a ver un ejemplo, empleando el conjunto de datos \texttt{airquality} que contiene datos falntantes en sus dos primeras variables:

\begin{Shaded}
\begin{Highlighting}[]
\FunctionTok{data}\NormalTok{(}\StringTok{"airquality"}\NormalTok{)}
\NormalTok{datos }\OtherTok{\textless{}{-}}\NormalTok{ airquality[,}\DecValTok{1}\SpecialCharTok{:}\DecValTok{3}\NormalTok{]}
\FunctionTok{summary}\NormalTok{(datos)}
\end{Highlighting}
\end{Shaded}

\begin{verbatim}
##      Ozone           Solar.R           Wind       
##  Min.   :  1.00   Min.   :  7.0   Min.   : 1.700  
##  1st Qu.: 18.00   1st Qu.:115.8   1st Qu.: 7.400  
##  Median : 31.50   Median :205.0   Median : 9.700  
##  Mean   : 42.13   Mean   :185.9   Mean   : 9.958  
##  3rd Qu.: 63.25   3rd Qu.:258.8   3rd Qu.:11.500  
##  Max.   :168.00   Max.   :334.0   Max.   :20.700  
##  NA's   :37       NA's   :7
\end{verbatim}

\begin{Shaded}
\begin{Highlighting}[]
\FunctionTok{nrow}\NormalTok{(datos)}
\end{Highlighting}
\end{Shaded}

\begin{verbatim}
## [1] 153
\end{verbatim}

\begin{Shaded}
\begin{Highlighting}[]
\CommentTok{\# Datos faltantes por variable}
\FunctionTok{sapply}\NormalTok{(datos, }\ControlFlowTok{function}\NormalTok{(x) }\FunctionTok{sum}\NormalTok{(}\FunctionTok{is.na}\NormalTok{(x)))}
\end{Highlighting}
\end{Shaded}

\begin{verbatim}
##   Ozone Solar.R    Wind 
##      37       7       0
\end{verbatim}

A continuación se muestra la distribución de los datos perdidos en el data.frame (a lo largo del tiempo, por mes):

\begin{Shaded}
\begin{Highlighting}[]
\FunctionTok{plot}\NormalTok{(}\FunctionTok{ts}\NormalTok{(airquality[,}\DecValTok{1}\SpecialCharTok{:}\DecValTok{2}\NormalTok{]))}
\end{Highlighting}
\end{Shaded}

\begin{center}\includegraphics[width=0.7\linewidth]{02-ManipulacionDatosR_files/figure-latex/unnamed-chunk-49-1} \end{center}

¿Existe un patrón no aleatorio en los datos faltantes del ozono? Esta pregunta puede ser abordada parcialmente utilizando el test de Little \citep{little1998}, disponible en la función \texttt{mcar\_test()} del paquete \texttt{naniar}. Este test permite evaluar si los datos faltantes son generados por un mecanismo completamente aleatorio (MCAR). Si la hipótesis de MCAR es rechazada, esto sugiere que los datos faltantes podrían estar siguiendo un mecanismo MAR (\emph{missing at random}) o MNAR (\emph{non missing at random}).

Sin embargo, en muchos estudios, se omite el paso anterior y se procede directamente con alguno de los siguientes métodos:

\begin{itemize}
\tightlist
\item
  Análisis de casos completos (\emph{complete cases})
\item
  Análisis de casos disponibles (borrado por parejas \emph{pairwise cases})
\item
  Imputación de datos faltantes (por la media, mediana, último valor observado, vecino más cercano, valores predichos usando los datos observados\ldots.)
\end{itemize}

Siguiendo con el ejemplo, ante la presencia de datos faltantes, en R inicialmente no podemos conocer cómo se relacionan las tres primeras variables:''

\begin{Shaded}
\begin{Highlighting}[]
\FunctionTok{cor}\NormalTok{(datos[,}\DecValTok{1}\SpecialCharTok{:}\DecValTok{3}\NormalTok{])}
\end{Highlighting}
\end{Shaded}

\begin{verbatim}
##         Ozone Solar.R Wind
## Ozone       1      NA   NA
## Solar.R    NA       1   NA
## Wind       NA      NA    1
\end{verbatim}

y requiere indicar cómo tratar los datos perdidos. Por ejemplo,
una opción sería realizar un análisis sólo de los casos completos, eliminando todas las observaciones (filas) con algún dato faltante de nuestro conjunto de datos:

\begin{Shaded}
\begin{Highlighting}[]
\NormalTok{datosC }\OtherTok{\textless{}{-}} \FunctionTok{na.omit}\NormalTok{(datos)}
\FunctionTok{nrow}\NormalTok{(datosC) }\CommentTok{\# n fija (sólo se utilizan 111 de las 153 de Wind)}
\end{Highlighting}
\end{Shaded}

\begin{verbatim}
## [1] 111
\end{verbatim}

\begin{Shaded}
\begin{Highlighting}[]
\FunctionTok{cor}\NormalTok{(datosC[,}\DecValTok{1}\SpecialCharTok{:}\DecValTok{3}\NormalTok{])}
\end{Highlighting}
\end{Shaded}

\begin{verbatim}
##              Ozone    Solar.R       Wind
## Ozone    1.0000000  0.3483417 -0.6124966
## Solar.R  0.3483417  1.0000000 -0.1271835
## Wind    -0.6124966 -0.1271835  1.0000000
\end{verbatim}

\begin{Shaded}
\begin{Highlighting}[]
\CommentTok{\# otra forma de hacerlo sería:}
\CommentTok{\# nrow(datos[complete.cases(datos),]) }
\CommentTok{\# cor(datos[,1:3], use ="complete.obs") }
\end{Highlighting}
\end{Shaded}

También, se podría usar toda la información disponible. El tamaño muestral \(n\) sería variable en función de los NA's de cada par de variables:

\begin{Shaded}
\begin{Highlighting}[]
\FunctionTok{cor}\NormalTok{(datos[,}\DecValTok{1}\SpecialCharTok{:}\DecValTok{3}\NormalTok{], }\AttributeTok{use =} \StringTok{"pairwise.complete.obs"}\NormalTok{)}
\end{Highlighting}
\end{Shaded}

\begin{verbatim}
##              Ozone     Solar.R        Wind
## Ozone    1.0000000  0.34834169 -0.60154653
## Solar.R  0.3483417  1.00000000 -0.05679167
## Wind    -0.6015465 -0.05679167  1.00000000
\end{verbatim}

Por ejmmplo, ahora la correlación usa los \(146\) pares de observaciones disponibles para (\texttt{Solar.R},\texttt{Wind}), en lugar de \(111\) del primer caso.

Por último, también se podría realizar una imputación \citep{van2018flexible}. A modo de ejemplo, en el siguiente código, se utiliza la media:

\begin{Shaded}
\begin{Highlighting}[]
\NormalTok{datosI }\OtherTok{\textless{}{-}}\NormalTok{ datos}
\NormalTok{datosI}\SpecialCharTok{$}\NormalTok{Ozone[}\FunctionTok{is.na}\NormalTok{(datos}\SpecialCharTok{$}\NormalTok{Ozone)] }\OtherTok{\textless{}{-}} \FunctionTok{mean}\NormalTok{(datos}\SpecialCharTok{$}\NormalTok{Ozone, }\AttributeTok{na.rm =}\NormalTok{ T)}
\NormalTok{datosI}\SpecialCharTok{$}\NormalTok{Solar.R[}\FunctionTok{is.na}\NormalTok{(datos}\SpecialCharTok{$}\NormalTok{Solar.R)] }\OtherTok{\textless{}{-}} \FunctionTok{mean}\NormalTok{(datosI}\SpecialCharTok{$}\NormalTok{Solar.R, }\AttributeTok{na.rm =}\NormalTok{ T)}
\FunctionTok{cor}\NormalTok{(datosI[,}\DecValTok{1}\SpecialCharTok{:}\DecValTok{3}\NormalTok{])}
\end{Highlighting}
\end{Shaded}

\begin{verbatim}
##              Ozone     Solar.R        Wind
## Ozone    1.0000000  0.30296951 -0.53093584
## Solar.R  0.3029695  1.00000000 -0.05524488
## Wind    -0.5309358 -0.05524488  1.00000000
\end{verbatim}

Notar que para el caso del ozono, se han sustituido los 37 \emph{NA's} (24\% de las observaciones) por un único valor (de ahí que ahora la varianza sea menor a la observada inicialmente, algo que en principio, no sería deseable).

\begin{Shaded}
\begin{Highlighting}[]
\FunctionTok{var}\NormalTok{(datos}\SpecialCharTok{$}\NormalTok{Ozone,}\AttributeTok{na.rm =}\NormalTok{ T)}
\end{Highlighting}
\end{Shaded}

\begin{verbatim}
## [1] 1088.201
\end{verbatim}

\begin{Shaded}
\begin{Highlighting}[]
\FunctionTok{var}\NormalTok{(datosI}\SpecialCharTok{$}\NormalTok{Ozone)}
\end{Highlighting}
\end{Shaded}

\begin{verbatim}
## [1] 823.3096
\end{verbatim}

Los datos faltantes son una realidad común en muchos estudios, aunque nadie los desea. Para tratarlos correctamente, es esencial comprender cómo se obtuvieron los datos observados y por qué algunos datos no fueron registrados antes de iniciar cualquier otro análisis. No abordar adecuadamente los datos faltantes puede tener un efecto perjudicial en nuestro estudio, ya que las conclusiones obtenidas podrían ser no representativas o contener sesgos.

\subsection{\texorpdfstring{Funciones \texttt{apply}}{Funciones apply}}\label{funciones-apply}

\subsubsection{\texorpdfstring{La función \texttt{apply}}{La función apply}}\label{la-funciuxf3n-apply}

Una forma de evitar la utilización de bucles es utilizando la sentencia \texttt{apply} que permite evaluar una misma función en todas las filas, columnas, etc. de un array de forma simultánea.

La sintaxis de esta función es:

\begin{Shaded}
\begin{Highlighting}[]
\FunctionTok{apply}\NormalTok{(X, MARGIN, FUN, ...)}
\end{Highlighting}
\end{Shaded}

\begin{itemize}
\tightlist
\item
  \texttt{X}: matriz (o array).
\item
  \texttt{MARGIN}: un vector indicando las dimensiones donde se aplicará
  la función. 1 indica filas, 2 indica columnas, y \texttt{c(1,2)} indica
  filas y columnas.
\item
  \texttt{FUN}: función que será aplicada.
\item
  \texttt{...}: argumentos opcionales que serán usados por \texttt{FUN}.
\end{itemize}

Veamos la utilización de la función \texttt{apply} con un ejemplo:

\begin{Shaded}
\begin{Highlighting}[]
\NormalTok{x }\OtherTok{\textless{}{-}} \FunctionTok{matrix}\NormalTok{(}\DecValTok{1}\SpecialCharTok{:}\DecValTok{9}\NormalTok{, }\AttributeTok{nrow =} \DecValTok{3}\NormalTok{)}
\NormalTok{x}
\end{Highlighting}
\end{Shaded}

\begin{verbatim}
##      [,1] [,2] [,3]
## [1,]    1    4    7
## [2,]    2    5    8
## [3,]    3    6    9
\end{verbatim}

\begin{Shaded}
\begin{Highlighting}[]
\FunctionTok{apply}\NormalTok{(x, }\DecValTok{1}\NormalTok{, sum)    }\CommentTok{\# Suma por filas}
\end{Highlighting}
\end{Shaded}

\begin{verbatim}
## [1] 12 15 18
\end{verbatim}

\begin{Shaded}
\begin{Highlighting}[]
\FunctionTok{apply}\NormalTok{(x, }\DecValTok{2}\NormalTok{, sum)    }\CommentTok{\# Suma por columnas}
\end{Highlighting}
\end{Shaded}

\begin{verbatim}
## [1]  6 15 24
\end{verbatim}

\begin{Shaded}
\begin{Highlighting}[]
\FunctionTok{apply}\NormalTok{(x, }\DecValTok{2}\NormalTok{, min)    }\CommentTok{\# Mínimo de las columnas}
\end{Highlighting}
\end{Shaded}

\begin{verbatim}
## [1] 1 4 7
\end{verbatim}

\begin{Shaded}
\begin{Highlighting}[]
\FunctionTok{apply}\NormalTok{(x, }\DecValTok{2}\NormalTok{, range)  }\CommentTok{\# Rango (mínimo y máximo) de las columnas}
\end{Highlighting}
\end{Shaded}

\begin{verbatim}
##      [,1] [,2] [,3]
## [1,]    1    4    7
## [2,]    3    6    9
\end{verbatim}

Alternativamente, se puede utilizar opciones más eficientes: \texttt{colSums()}, \texttt{rowSums()}, \texttt{colMeans()} y \texttt{rowMeans()}, como se muestra en el siguiente código de ejemplo:

\begin{Shaded}
\begin{Highlighting}[]
\NormalTok{x }\OtherTok{\textless{}{-}} \FunctionTok{matrix}\NormalTok{(}\DecValTok{1}\SpecialCharTok{:}\FloatTok{1e8}\NormalTok{, }\AttributeTok{ncol =} \DecValTok{10}\NormalTok{, }\AttributeTok{byrow =} \ConstantTok{FALSE}\NormalTok{)}
\NormalTok{t1 }\OtherTok{\textless{}{-}} \FunctionTok{proc.time}\NormalTok{()}
\NormalTok{out}\OtherTok{\textless{}{-}}\FunctionTok{apply}\NormalTok{(x, }\DecValTok{2}\NormalTok{, mean)   }
\FunctionTok{proc.time}\NormalTok{() }\SpecialCharTok{{-}}\NormalTok{ t1}
\end{Highlighting}
\end{Shaded}

\begin{verbatim}
##    user  system elapsed 
##   0.496   0.108   0.604
\end{verbatim}

\begin{Shaded}
\begin{Highlighting}[]
\NormalTok{t2 }\OtherTok{\textless{}{-}} \FunctionTok{proc.time}\NormalTok{()}
\NormalTok{out }\OtherTok{\textless{}{-}} \FunctionTok{colMeans}\NormalTok{(x)}
\FunctionTok{proc.time}\NormalTok{() }\SpecialCharTok{{-}}\NormalTok{ t2}
\end{Highlighting}
\end{Shaded}

\begin{verbatim}
##    user  system elapsed 
##   0.080   0.000   0.079
\end{verbatim}

\subsubsection{\texorpdfstring{Variantes de la función \texttt{apply}}{Variantes de la función apply}}\label{variantes-de-la-funciuxf3n-apply}

\begin{enumerate}
\def\labelenumi{\alph{enumi}.}
\tightlist
\item
  La función \href{https://www.rdocumentation.org/packages/base/versions/3.6.1/topics/lapply}{\texttt{lapply(X,\ FUN,\ ...)}}
  aplica la función \texttt{FUN} a cada elemento de una lista en R y devuelve una lista como resultado (sin necesidad de especificar el argumento MARGIN). Notar que todas las estructuras de datos en R pueden convertirse en listas, por lo que \texttt{lapply()} puede utilizarse en más casos que \texttt{apply()}.
\end{enumerate}

\begin{Shaded}
\begin{Highlighting}[]
\CommentTok{\# lista con las medianas de las variables}
\NormalTok{list }\OtherTok{\textless{}{-}} \FunctionTok{lapply}\NormalTok{(cars, median)}
\FunctionTok{str}\NormalTok{(list)}
\end{Highlighting}
\end{Shaded}

\begin{verbatim}
## List of 4
##  $ speed    : num 15
##  $ dist     : num 36
##  $ velocidad: num 24.1
##  $ distancia: num 11
\end{verbatim}

\begin{enumerate}
\def\labelenumi{\alph{enumi}.}
\setcounter{enumi}{1}
\tightlist
\item
  La función
  \href{https://www.rdocumentation.org/packages/base/versions/3.6.1/topics/sapply}{\texttt{sapply(X,\ FUN,\ ...,\ simplify\ =\ TRUE,\ USE.NAMES\ =\ TRUE)}} permite iterar sobre una lista o vector (alternativa más eficiente a un \texttt{for}):
\end{enumerate}

\begin{Shaded}
\begin{Highlighting}[]
\CommentTok{\# matriz con las medias, medianas y desv. de las variables}
\NormalTok{res }\OtherTok{\textless{}{-}} \FunctionTok{sapply}\NormalTok{(cars, }
          \ControlFlowTok{function}\NormalTok{(x) }\FunctionTok{c}\NormalTok{(}\AttributeTok{mean =} \FunctionTok{mean}\NormalTok{(x), }
                        \AttributeTok{median =} \FunctionTok{median}\NormalTok{(x), }
                        \AttributeTok{sd =} \FunctionTok{sd}\NormalTok{(x)))}
\CommentTok{\# str(res)}
\NormalTok{res}
\end{Highlighting}
\end{Shaded}

\begin{verbatim}
##            speed     dist velocidad distancia
## mean   15.400000 42.98000 24.783945 13.100463
## median 15.000000 36.00000 24.140206 10.972933
## sd      5.287644 25.76938  8.509655  7.854602
\end{verbatim}

\begin{Shaded}
\begin{Highlighting}[]
\NormalTok{cfuns }\OtherTok{\textless{}{-}} \ControlFlowTok{function}\NormalTok{(x, }\AttributeTok{funs =} \FunctionTok{c}\NormalTok{(mean, median, sd))}
            \FunctionTok{sapply}\NormalTok{(funs, }\ControlFlowTok{function}\NormalTok{(f) }\FunctionTok{f}\NormalTok{(x))}
\NormalTok{x }\OtherTok{\textless{}{-}} \DecValTok{1}\SpecialCharTok{:}\DecValTok{10}
\FunctionTok{cfuns}\NormalTok{(x)}
\end{Highlighting}
\end{Shaded}

\begin{verbatim}
## [1] 5.50000 5.50000 3.02765
\end{verbatim}

\begin{Shaded}
\begin{Highlighting}[]
\FunctionTok{sapply}\NormalTok{(cars, cfuns)}
\end{Highlighting}
\end{Shaded}

\begin{verbatim}
##          speed     dist velocidad distancia
## [1,] 15.400000 42.98000 24.783945 13.100463
## [2,] 15.000000 36.00000 24.140206 10.972933
## [3,]  5.287644 25.76938  8.509655  7.854602
\end{verbatim}

\begin{Shaded}
\begin{Highlighting}[]
\NormalTok{nfuns }\OtherTok{\textless{}{-}} \FunctionTok{c}\NormalTok{(}\StringTok{"mean"}\NormalTok{, }\StringTok{"median"}\NormalTok{, }\StringTok{"sd"}\NormalTok{)}
\FunctionTok{sapply}\NormalTok{(nfuns, }
       \ControlFlowTok{function}\NormalTok{(f) }\FunctionTok{eval}\NormalTok{(}\FunctionTok{parse}\NormalTok{(}\AttributeTok{text =} \FunctionTok{paste0}\NormalTok{(f, }\StringTok{"(x)"}\NormalTok{))))}
\end{Highlighting}
\end{Shaded}

\begin{verbatim}
##    mean  median      sd 
## 5.50000 5.50000 3.02765
\end{verbatim}

\begin{enumerate}
\def\labelenumi{\alph{enumi}.}
\setcounter{enumi}{2}
\tightlist
\item
  La función \href{https://www.rdocumentation.org/packages/base/versions/3.6.1/topics/tapply}{\texttt{tapply()}} es
  similar a la función \texttt{apply()} y permite aplicar una función a los datos desagregados,
  utilizando como criterio los distintos niveles de una variable factor. Es decir,
  facilita la creación de tablars resumen por grupos. La sintaxis de esta función es como sigue:
\end{enumerate}

\begin{Shaded}
\begin{Highlighting}[]
    \FunctionTok{tapply}\NormalTok{(X, INDEX, FUN, ...,)}
\end{Highlighting}
\end{Shaded}

\begin{itemize}
\tightlist
\item
  \texttt{X}: matriz (o array).
\item
  \texttt{INDEX}: factor indicando los grupos (niveles).
\item
  \texttt{FUN}: función que será aplicada.
\item
  \texttt{...}: argumentos opcionales .
\end{itemize}

Consideremos, por ejemplo, el data.frame \texttt{ChickWeight} con datos de un
experimento relacionado con la repercusión de varias dietas en el peso
de pollos.

\begin{Shaded}
\begin{Highlighting}[]
\FunctionTok{data}\NormalTok{(ChickWeight)}
\CommentTok{\# str(ChickWeight)}
\FunctionTok{head}\NormalTok{(ChickWeight)}
\end{Highlighting}
\end{Shaded}

\begin{verbatim}
##   weight Time Chick Diet
## 1     42    0     1    1
## 2     51    2     1    1
## 3     59    4     1    1
## 4     64    6     1    1
## 5     76    8     1    1
## 6     93   10     1    1
\end{verbatim}

\begin{Shaded}
\begin{Highlighting}[]
\NormalTok{peso }\OtherTok{\textless{}{-}}\NormalTok{ ChickWeight}\SpecialCharTok{$}\NormalTok{weight}
\NormalTok{dieta }\OtherTok{\textless{}{-}}\NormalTok{ ChickWeight}\SpecialCharTok{$}\NormalTok{Diet}
\FunctionTok{levels}\NormalTok{(dieta) }\OtherTok{\textless{}{-}} \FunctionTok{c}\NormalTok{(}\StringTok{"Dieta 1"}\NormalTok{, }\StringTok{"Dieta 2"}\NormalTok{, }\StringTok{"Dieta 3"}\NormalTok{, }\StringTok{"Dieta 4"}\NormalTok{)}
\FunctionTok{tapply}\NormalTok{(peso, dieta, mean)  }\CommentTok{\# Peso medio por dieta}
\end{Highlighting}
\end{Shaded}

\begin{verbatim}
##  Dieta 1  Dieta 2  Dieta 3  Dieta 4 
## 102.6455 122.6167 142.9500 135.2627
\end{verbatim}

\begin{Shaded}
\begin{Highlighting}[]
\FunctionTok{tapply}\NormalTok{(peso, dieta, summary)}
\end{Highlighting}
\end{Shaded}

\begin{verbatim}
## $`Dieta 1`
##    Min. 1st Qu.  Median    Mean 3rd Qu.    Max. 
##   35.00   57.75   88.00  102.65  136.50  305.00 
## 
## $`Dieta 2`
##    Min. 1st Qu.  Median    Mean 3rd Qu.    Max. 
##    39.0    65.5   104.5   122.6   163.0   331.0 
## 
## $`Dieta 3`
##    Min. 1st Qu.  Median    Mean 3rd Qu.    Max. 
##    39.0    67.5   125.5   142.9   198.8   373.0 
## 
## $`Dieta 4`
##    Min. 1st Qu.  Median    Mean 3rd Qu.    Max. 
##   39.00   71.25  129.50  135.26  184.75  322.00
\end{verbatim}

Alternativamente, se podría emplear la función \texttt{aggregate()} que tiene las ventajas de admitir fórmulas y disponer de un método para series de tiempo.

\begin{Shaded}
\begin{Highlighting}[]
\FunctionTok{help}\NormalTok{(aggregate)}
\FunctionTok{aggregate}\NormalTok{(peso,}\AttributeTok{by=}\FunctionTok{list}\NormalTok{(}\AttributeTok{dieta=}\NormalTok{dieta),}\AttributeTok{FUN =} \StringTok{"mean"}\NormalTok{ )}
\end{Highlighting}
\end{Shaded}

\begin{verbatim}
##     dieta        x
## 1 Dieta 1 102.6455
## 2 Dieta 2 122.6167
## 3 Dieta 3 142.9500
## 4 Dieta 4 135.2627
\end{verbatim}

\begin{Shaded}
\begin{Highlighting}[]
\FunctionTok{aggregate}\NormalTok{(peso}\SpecialCharTok{\textasciitilde{}}\NormalTok{dieta,}\AttributeTok{FUN =} \StringTok{"summary"}\NormalTok{ ) }\CommentTok{\# con formula}
\end{Highlighting}
\end{Shaded}

\begin{verbatim}
##     dieta peso.Min. peso.1st Qu. peso.Median peso.Mean
## 1 Dieta 1   35.0000      57.7500     88.0000  102.6455
## 2 Dieta 2   39.0000      65.5000    104.5000  122.6167
## 3 Dieta 3   39.0000      67.5000    125.5000  142.9500
## 4 Dieta 4   39.0000      71.2500    129.5000  135.2627
##   peso.3rd Qu. peso.Max.
## 1     136.5000  305.0000
## 2     163.0000  331.0000
## 3     198.7500  373.0000
## 4     184.7500  322.0000
\end{verbatim}

\subsection{Tablas (para informes)}\label{tablas-para-informes}

\begin{enumerate}
\def\labelenumi{\alph{enumi}.}
\tightlist
\item
  Tablas con \texttt{kable()}:
\end{enumerate}

A continuación, se muestra un ejemplo, de tabla resumen, con las medias, medianas y desviación típica de las variables:

\begin{Shaded}
\begin{Highlighting}[]
\NormalTok{res }\OtherTok{\textless{}{-}} \FunctionTok{sapply}\NormalTok{(cars, }
          \ControlFlowTok{function}\NormalTok{(x) }\FunctionTok{c}\NormalTok{(}\AttributeTok{mean =} \FunctionTok{mean}\NormalTok{(x), }
                        \AttributeTok{median =} \FunctionTok{median}\NormalTok{(x), }
                        \AttributeTok{sd =} \FunctionTok{sd}\NormalTok{(x)))}
\NormalTok{knitr}\SpecialCharTok{::}\FunctionTok{kable}\NormalTok{(}\FunctionTok{t}\NormalTok{(res), }\AttributeTok{digits =} \DecValTok{1}\NormalTok{, }
             \AttributeTok{col.names =} \FunctionTok{c}\NormalTok{(}\StringTok{"Media"}\NormalTok{, }\StringTok{"Mediana"}\NormalTok{, }\StringTok{"Desv. típica"}\NormalTok{))}
\end{Highlighting}
\end{Shaded}

\begin{tabular}{l|r|r|r}
\hline
  & Media & Mediana & Desv. típica\\
\hline
speed & 15.4 & 15.0 & 5.3\\
\hline
dist & 43.0 & 36.0 & 25.8\\
\hline
velocidad & 24.8 & 24.1 & 8.5\\
\hline
distancia & 13.1 & 11.0 & 7.9\\
\hline
\end{tabular}

Y en este segundo ejemplo, se muestra el resumen de un modelo de regresión lineal simple (distancia de frenado en función de la velocidad del vehículo):

\begin{Shaded}
\begin{Highlighting}[]
\NormalTok{modelo }\OtherTok{\textless{}{-}} \FunctionTok{lm}\NormalTok{(dist }\SpecialCharTok{\textasciitilde{}}\NormalTok{ speed, }\AttributeTok{data =}\NormalTok{ cars)}
\NormalTok{coefs }\OtherTok{\textless{}{-}} \FunctionTok{coef}\NormalTok{(}\FunctionTok{summary}\NormalTok{(modelo))}
\NormalTok{knitr}\SpecialCharTok{::}\FunctionTok{kable}\NormalTok{(coefs, }\AttributeTok{escape =} \ConstantTok{FALSE}\NormalTok{, }\AttributeTok{digits =} \DecValTok{5}\NormalTok{)}
\end{Highlighting}
\end{Shaded}

\begin{tabular}{l|r|r|r|r}
\hline
  & Estimate & Std. Error & t value & Pr(>|t|)\\
\hline
(Intercept) & -17.57909 & 6.75844 & -2.60106 & 0.01232\\
\hline
speed & 3.93241 & 0.41551 & 9.46399 & 0.00000\\
\hline
\end{tabular}

\begin{enumerate}
\def\labelenumi{\alph{enumi}.}
\setcounter{enumi}{1}
\tightlist
\item
  Tablas interactivas con \texttt{datatabe()} del paquete \texttt{DT}:
\end{enumerate}

\begin{Shaded}
\begin{Highlighting}[]
\FunctionTok{library}\NormalTok{(DT)}
\FunctionTok{datatable}\NormalTok{(iris,}\AttributeTok{options =} \FunctionTok{list}\NormalTok{(}\AttributeTok{scrollX =} \ConstantTok{TRUE}\NormalTok{))}
\end{Highlighting}
\end{Shaded}

Hay muchos otros paquetes de R que se pueden utilizar para generar tablas como:
\texttt{kableExtra()}, \texttt{flextable()}, \texttt{reactable()}, \texttt{reactablefmtr()},
\texttt{formattable()}, \texttt{gt()} y \texttt{tinytable()}.

\subsection{Operaciones con tablas de datos}\label{operaciones-con-tablas-de-datos}

\textbf{\emph{Unir tablas}}:

\begin{itemize}
\item
  \href{https://www.rdocumentation.org/packages/base/versions/3.6.1/topics/rbind}{\texttt{rbind()}}: combina vectores, matrices, arrays o data.frames por filas.
\item
  \href{https://www.rdocumentation.org/packages/base/versions/3.6.1/topics/cbind}{\texttt{cbind()}}: Idem por columnas.
\item
  \href{https://www.rdocumentation.org/packages/base/versions/3.6.1/topics/merge}{\texttt{merge()}}: Fusiona dos data.frame por columnas o nombres de fila comunes. También permite otras operaciones de unión (\emph{join}) de bases de datos, algunas de ellas se verán con más detalle en el Capítulo 4.
\end{itemize}

\textbf{\emph{Combinar tablas}}:

\begin{itemize}
\item
  \href{https://www.rdocumentation.org/packages/base/versions/3.6.1/topics/match}{\texttt{match(x,\ table)}} devuelve un vector (de la misma longitud que \texttt{x}) con las (primeras) posiciones de coincidencia de \texttt{x} en \texttt{table} (o \texttt{NA}, por defecto, si no hay coincidencia).

  Para realizar consultas combinando tablas puede ser más cómodo el operador \texttt{\%in\%} (\texttt{?\textquotesingle{}\%in\%\textquotesingle{}}).
\item
  \href{https://www.rdocumentation.org/packages/base/versions/3.6.1/topics/pmatch}{\texttt{pmatch(x,\ table,\ ...)}}: similar al anterior pero con coincidencias parciales de cadenas de texto.
\end{itemize}

\section{Ejemplo WoS data}\label{ejemplo-wos-data}

Ejemplo \href{data/wosdata.R}{\emph{wosdata.R}} en \href{data/wosdata.zip}{\emph{wosdata.zip}}.
Ver Apéndice \ref{scimetr}.

\begin{Shaded}
\begin{Highlighting}[]
\CommentTok{\# library(dplyr)}
\CommentTok{\# library(stringr)}
\CommentTok{\# https://rubenfcasal.github.io/scimetr/articles/scimetr.html}
\CommentTok{\# library(scimetr)}

\NormalTok{db }\OtherTok{\textless{}{-}} \FunctionTok{readRDS}\NormalTok{(}\StringTok{"data/wosdata/db\_udc\_2015.rds"}\NormalTok{)}
\FunctionTok{str}\NormalTok{(db, }\DecValTok{1}\NormalTok{)}
\end{Highlighting}
\end{Shaded}

\begin{verbatim}
## List of 11
##  $ Docs      :'data.frame':  856 obs. of  26 variables:
##  $ Authors   :'data.frame':  4051 obs. of  4 variables:
##  $ AutDoc    :'data.frame':  5511 obs. of  2 variables:
##  $ Categories:'data.frame':  189 obs. of  2 variables:
##  $ CatDoc    :'data.frame':  1495 obs. of  2 variables:
##  $ Areas     :'data.frame':  121 obs. of  2 variables:
##  $ AreaDoc   :'data.frame':  1364 obs. of  2 variables:
##  $ Addresses :'data.frame':  3655 obs. of  5 variables:
##  $ AddAutDoc :'data.frame':  7751 obs. of  3 variables:
##  $ Journals  :'data.frame':  520 obs. of  12 variables:
##  $ label     : chr ""
##  - attr(*, "variable.labels")= Named chr [1:62] "Publication type" "Author" "Book authors" "Editor" ...
##   ..- attr(*, "names")= chr [1:62] "PT" "AU" "BA" "BE" ...
##  - attr(*, "class")= chr "wos.db"
\end{verbatim}

\begin{Shaded}
\begin{Highlighting}[]
\NormalTok{variable.labels }\OtherTok{\textless{}{-}} \FunctionTok{attr}\NormalTok{(db, }\StringTok{"variable.labels"}\NormalTok{)}
\NormalTok{knitr}\SpecialCharTok{::}\FunctionTok{kable}\NormalTok{(}\FunctionTok{as.data.frame}\NormalTok{(variable.labels),}
             \AttributeTok{caption =} \StringTok{"Variable labels"}\NormalTok{)}
\end{Highlighting}
\end{Shaded}

\begin{table}

\caption{\label{tab:unnamed-chunk-69}Variable labels}
\centering
\begin{tabular}[t]{l|l}
\hline
  & variable.labels\\
\hline
PT & Publication type\\
\hline
AU & Author\\
\hline
BA & Book authors\\
\hline
BE & Editor\\
\hline
GP & Group author\\
\hline
AF & Author full\\
\hline
BF & Book authors fullname\\
\hline
CA & Corporate author\\
\hline
TI & Title\\
\hline
SO & Publication name\\
\hline
SE & Series title\\
\hline
BS & Book series\\
\hline
LA & Language\\
\hline
DT & Document type\\
\hline
CT & Conference title\\
\hline
CY & Conference year\\
\hline
CL & Conference place\\
\hline
SP & Conference sponsors\\
\hline
HO & Conference host\\
\hline
DE & Keywords\\
\hline
ID & Keywords Plus\\
\hline
AB & Abstract\\
\hline
C1 & Addresses\\
\hline
RP & Reprint author\\
\hline
EM & Author email\\
\hline
RI & Researcher id numbers\\
\hline
OI & Orcid numbers\\
\hline
FU & Funding agency and grant number\\
\hline
FX & Funding text\\
\hline
CR & Cited references\\
\hline
NR & Number of cited references\\
\hline
TC & Times cited\\
\hline
Z9 & Total times cited count\\
\hline
U1 & Usage Count (Last 180 Days)\\
\hline
U2 & Usage Count (Since 2013)\\
\hline
PU & Publisher\\
\hline
PI & Publisher city\\
\hline
PA & Publisher address\\
\hline
SN & ISSN\\
\hline
EI & eISSN\\
\hline
BN & ISBN\\
\hline
J9 & Journal.ISI\\
\hline
JI & Journal.ISO\\
\hline
PD & Publication date\\
\hline
PY & Year published\\
\hline
VL & Volume\\
\hline
IS & Issue\\
\hline
PN & Part number\\
\hline
SU & Supplement\\
\hline
SI & Special issue\\
\hline
MA & Meeting abstract\\
\hline
BP & Beginning page\\
\hline
EP & Ending page\\
\hline
AR & Article number\\
\hline
DI & DOI\\
\hline
D2 & Book DOI\\
\hline
PG & Page count\\
\hline
WC & WOS category\\
\hline
SC & Research areas\\
\hline
GA & Document delivery number\\
\hline
UT & Access number\\
\hline
PM & Pub Med ID\\
\hline
\end{tabular}
\end{table}

Veamos ahora un par de ejemplos, en el primero se buscan los documentos correspondientes a revistas (que contiene \texttt{Chem} en el título de la revista \emph{journal}). Para ello utilizamos la función \texttt{grepl()} que busca las coincidencias con el patrón \texttt{Chem} dentro de cada elemento de un vector de caracteres.

\begin{Shaded}
\begin{Highlighting}[]
\CommentTok{\# View(db$Journals)}
\NormalTok{iidj }\OtherTok{\textless{}{-}} \FunctionTok{with}\NormalTok{(db}\SpecialCharTok{$}\NormalTok{Journals, idj[}\FunctionTok{grepl}\NormalTok{(}\StringTok{\textquotesingle{}Chem\textquotesingle{}}\NormalTok{, JI)])}
\NormalTok{db}\SpecialCharTok{$}\NormalTok{Journals}\SpecialCharTok{$}\NormalTok{JI[iidj]}
\end{Highlighting}
\end{Shaded}

\begin{verbatim}
##  [1] "J. Am. Chem. Soc."                 
##  [2] "Inorg. Chem."                      
##  [3] "J. Chem. Phys."                    
##  [4] "J. Chem. Thermodyn."               
##  [5] "J. Solid State Chem."              
##  [6] "Chemosphere"                       
##  [7] "Antimicrob. Agents Chemother."     
##  [8] "Trac-Trends Anal. Chem."           
##  [9] "Eur. J. Med. Chem."                
## [10] "J. Chem. Technol. Biotechnol."     
## [11] "J. Antimicrob. Chemother."         
## [12] "Food Chem."                        
## [13] "Cancer Chemother. Pharmacol."      
## [14] "Int. J. Chem. Kinet."              
## [15] "Chem.-Eur. J."                     
## [16] "J. Phys. Chem. A"                  
## [17] "New J. Chem."                      
## [18] "Chem. Commun."                     
## [19] "Chem. Eng. J."                     
## [20] "Comb. Chem. High Throughput Screen"
## [21] "Mini-Rev. Med. Chem."              
## [22] "Phys. Chem. Chem. Phys."           
## [23] "Org. Biomol. Chem."                
## [24] "J. Chem Inf. Model."               
## [25] "ACS Chem. Biol."                   
## [26] "Environ. Chem. Lett."              
## [27] "Anal. Bioanal. Chem."              
## [28] "J. Cheminformatics"                
## [29] "J. Mat. Chem. B"
\end{verbatim}

\begin{Shaded}
\begin{Highlighting}[]
\NormalTok{idd }\OtherTok{\textless{}{-}} \FunctionTok{with}\NormalTok{(db}\SpecialCharTok{$}\NormalTok{Docs, idj }\SpecialCharTok{\%in\%}\NormalTok{ iidj)}
\FunctionTok{which}\NormalTok{(idd)}
\end{Highlighting}
\end{Shaded}

\begin{verbatim}
##  [1]   2   4  16  23  43  69 119 126 138 175 188 190 203 208
## [15] 226 240 272 337 338 341 342 357 382 385 386 387 388 394
## [29] 411 412 428 460 483 518 525 584 600 604 605 616 620 665
## [43] 697 751 753 775 784 796 806 808 847 848
\end{verbatim}

\begin{Shaded}
\begin{Highlighting}[]
\CommentTok{\# View(db$Docs[idd, ])}
\FunctionTok{head}\NormalTok{(db}\SpecialCharTok{$}\NormalTok{Docs[idd, }\SpecialCharTok{{-}}\DecValTok{3}\NormalTok{])}
\end{Highlighting}
\end{Shaded}

\begin{verbatim}
##    idd idj      PT      DT  NR TC Z9 U1 U2     PD   PY VL
## 2    2  37 Journal Article  45  5  5  0  0 DEC 21 2015 54
## 4    4 272 Journal Article  78  2  2  4 21 DEC 14 2015 21
## 16  16 195 Journal Article  34  2  2  0  0    DEC 2015 70
## 23  23 436 Journal Article  48  3  3  0  4    DEC 2015 10
## 43  43 455 Journal  Review 214  0  0  0  8    DEC 2015 13
## 69  69  37 Journal Article  86  2  2  8 28  NOV 2 2015 54
##    IS PN SU SI MA    BP    EP AR
## 2  24             11680 11687   
## 4  51             18662 18670   
## 16 12              3222  3229   
## 23 12              2850  2860   
## 43  4               413   430   
## 69 21             10342 10350   
##                               DI D2 PG           UT an
## 2  10.1021/acs.inorgchem.5b01652     8 367118100013  9
## 4         10.1002/chem.201502937     9 368280400026  8
## 16            10.1093/jac/dkv262     8 368246800008 10
## 23    10.1021/acschembio.5b00624    11 366875400020 10
## 43     10.1007/s10311-015-0526-2    18 365096700004  2
## 69 10.1021/acs.inorgchem.5b01719     9 364175000028  8
\end{verbatim}

En este segundo ejemplo, se buscan los documentos correspondientes a autores (que contiene \texttt{Abad} en su nombre):

\begin{Shaded}
\begin{Highlighting}[]
\CommentTok{\# View(db$Authors)}
\NormalTok{iida }\OtherTok{\textless{}{-}} \FunctionTok{with}\NormalTok{(db}\SpecialCharTok{$}\NormalTok{Authors, ida[}\FunctionTok{grepl}\NormalTok{(}\StringTok{\textquotesingle{}Abad\textquotesingle{}}\NormalTok{, AF)])}
\NormalTok{db}\SpecialCharTok{$}\NormalTok{Authors}\SpecialCharTok{$}\NormalTok{AF[iida]}
\end{Highlighting}
\end{Shaded}

\begin{verbatim}
## [1] "Mato Abad, Virginia" "Abad, Maria-Jose"   
## [3] "Abad Vicente, J."    "Abada, Sabah"
\end{verbatim}

\begin{Shaded}
\begin{Highlighting}[]
\NormalTok{idd }\OtherTok{\textless{}{-}} \FunctionTok{with}\NormalTok{(db}\SpecialCharTok{$}\NormalTok{AutDoc, idd[ida }\SpecialCharTok{\%in\%}\NormalTok{ iida])}
\NormalTok{idd}
\end{Highlighting}
\end{Shaded}

\begin{verbatim}
## [1] 273 291 518 586
\end{verbatim}

\begin{Shaded}
\begin{Highlighting}[]
\CommentTok{\# View(db$Docs[idd, ])}
\FunctionTok{head}\NormalTok{(db}\SpecialCharTok{$}\NormalTok{Docs[idd, }\SpecialCharTok{{-}}\DecValTok{3}\NormalTok{])}
\end{Highlighting}
\end{Shaded}

\begin{verbatim}
##     idd idj      PT               DT  NR TC Z9 U1 U2     PD
## 273 273 282 Journal          Article 107  8  8  0  0    SEP
## 291 291 141 Journal Meeting Abstract   0  0  0  0  1  AUG 1
## 518 518 272 Journal          Article 103  4  4  0  0 APR 20
## 586 586 311 Journal          Article  32  2  2  2 19    APR
##       PY VL    IS PN SU SI   MA   BP   EP AR
## 273 2015 42 15-16               6205 6214   
## 291 2015 36           1    P167    9    9   
## 518 2015 21    17               6535 6546   
## 586 2015 26     4                369  375   
##                             DI D2 PG           UT an
## 273 10.1016/j.eswa.2015.03.011    10 355063700018  7
## 291                                1 361205101026 10
## 518     10.1002/chem.201500155    12 352796100030 10
## 586           10.1002/pat.3462     7 351472700012  6
\end{verbatim}

\chapter{Introducción al lenguaje SQL}\label{introSQL}

Los sistemas de información gestionan repositorios de información en múltiples formatos,
siendo el más popular las bases de datos relacionales a las que se accede mediante SQL (Structured Query Language).

El ejemplo que trabajaremos en este capítulo está disponible en Kaggle: \href{https://www.kaggle.com/code/diegodx/txd-2025-tutorialsql}{kaggle.com/code/diegodx/txd-2025-tutorialsql}

\section{Bases de Datos Relacionales}\label{bases-de-datos-relacionales}

\subsection{Definiciones}\label{definiciones}

\begin{itemize}
\item
  \textbf{Dominio}: contexto (organización, empresa, evento\ldots) objeto de gestión de la información.
\item
  \textbf{Dato}: hecho con significado implícito, registable, relevante en un determinado dominio.
\item
  \textbf{Base de datos}: colección de datos de un determinado dominio relacionados entre sí, organizados de forma que sea posible manipularlos y recuperarlos de forma eficiente.
\item
  Sistema de Gestión de Bases de Datos (\textbf{SGBD}) (en inglés \textbf{RDBMS}, Relational Database Management System): software que permite a los usuarios crear y manipular bases de datos mediante operaciones \textbf{CRUD}:

  \begin{itemize}
  \tightlist
  \item
    \textbf{C}reate: Crear / Insertar datos
  \item
    \textbf{R}read: Consultar / Leer datos
  \item
    \textbf{U}pdate: Actualizar / Modificar datos
  \item
    \textbf{D}elete: Eliminar datos
  \end{itemize}
\end{itemize}

\begin{center}\rule{0.5\linewidth}{0.5pt}\end{center}

\begin{itemize}
\tightlist
\item
  \textbf{Modelo de datos}: abstracción conceptual que propone una manera de organizar y manipular los datos. Definido mediante:

  \begin{itemize}
  \tightlist
  \item
    Estructura: elementos para organizar datos
  \item
    Integridad: reglas para relaciones los elementos
  \item
    Manipulación: operaciones sobre los datos adaptadas a la estructura y reglas
  \end{itemize}
\item
  Modelo \textbf{Entidad Relación} (entidades, relaciones, atributos)
\end{itemize}

\includegraphics[width=4.16667in,height=\textheight,keepaspectratio]{images/modelo-ER.png}

\begin{itemize}
\tightlist
\item
  Modelo de datos lógico o de representación (\textbf{modelo relacional} de Codd)

  \begin{itemize}
  \tightlist
  \item
    Datos en relaciones (tablas)
  \item
    Base matemática formal
  \item
    Flexible
  \end{itemize}
\end{itemize}

\includegraphics[width=6.25in,height=\textheight,keepaspectratio]{images/modelo-relacional.png}

\begin{itemize}
\tightlist
\item
  Modelo de datos físico (tal y como se almacenan los datos)
\end{itemize}

Una fila de la tabla (relación) es una tupla y una columna un atributo.

\includegraphics[width=6.25in,height=\textheight,keepaspectratio]{images/Relacion.png}

Una base de datos es un conjunto de tablas (al menos una).

\includegraphics[width=6.25in,height=\textheight,keepaspectratio]{images/BBDD.png}

La tabla no es una relación porque la relación es un conjunto sin orden y una tabla puede tener filas repetidas y tiene orden.

\begin{center}\rule{0.5\linewidth}{0.5pt}\end{center}

\begin{itemize}
\item
  \textbf{Esquema}: estructura de la base de datos
\item
  \textbf{Estado}: contenido de la base de datos
\item
  Restricción de \textbf{integridad}: regla que debe cumplir la información registrada en la base de datos para garantizar la integridad de la información.
\end{itemize}

\section{Restricciones}\label{restricciones}

Cualquier Base de Datos basada en el modelo relacional debería cumplir como mínimo estas restricciones (además de las propias del dominio):

\begin{itemize}
\tightlist
\item
  \textbf{Restricción de dominio}: Cada atributo debe tener un tipo de valores permitido, asegurando que sólo se almacenan datos válidos y consistentes
\item
  \textbf{Atributos atómicos}: Cada atributo debe almacenar valores indivisibles (nombre completo descomponible en nombre y apellidos, domicilio en calle, CP, localidad, etc\ldots)
\item
  \textbf{Unicidad}: Co pueden existir dos tuplas iguales. Para ello se definen claves:

  \begin{itemize}
  \tightlist
  \item
    Una \textbf{superclave} es un subconjunto de atributos tal que no existen dos tuplas con la misma superclave.
    \textgreater{} Ejercicio. En la relación Empleado(dni, nombre, apellidos, email) ¿cuántas superclaves existen?
  \item
    Una \textbf{clave candidata} es una superclave mínima (superclave mínima es la clave a la que no se le puede eliminar un atributo).
    \textgreater{} ¿Cuántas claves candidatas hay en el ejemplo anterior?
  \item
    La \textbf{clave primaria} es la clave candidata que elegimos que identificar de forma unívoca las tuplas de una relación. Restricción de integridad de entidad: Ningún valor de la clave primaria puede ser un valor nulo.
  \end{itemize}
\item
  Una \textbf{clave foránea} es un conjunto de atributos de una relación R\_1 que, para cada tupla, identifican a otra tupla de una relación R\_2 con la que está relacionada.
  La Restricción de integridad referencial nos dice que la clave foránea ha de corresponderse con la clave primaria de R\_2, y si la clave foránea no es nula ha de refir a una tupla en R\_2.
\end{itemize}

\includegraphics[width=6.25in,height=\textheight,keepaspectratio]{images/ClaveForanea.png}

\includegraphics[width=6.25in,height=\textheight,keepaspectratio]{images/IntegridadReferencial.png}

Si borramos/actualizamos un valor de clave foránea podemos: (a) prohibir el cambio, o (b) poner a nulo la clave foránea (borrado) o propagar el cambio (modificación).

\section{Sistemas Gestores de Bases de Datos (SGDB)}\label{sistemas-gestores-de-bases-de-datos-sgdb}

Utilizar un SGDB tiene múltiples ventajas:

\begin{itemize}
\tightlist
\item
  Administración centralizada de los datos (por un administrador en un servidor/plataforma central que evita la información en silos -redundante/inconsistente)
\item
  Desacoplamiento del almacenamiento físico de los datos (no es necesario conocerlo)
\item
  Simplicidad de acceso (ODBC + SQL, lenguaje declarativo)
\item
  Control de integridad (restricciones genéricas, integridad de entidad y referencial, de dominio, y las del dominio en software)
\item
  Control de acceso concurrente (evita inconsistencia)
\item
  Seguridad (autenticación, roles de acceso)
\item
  Recuperación ante fallos (backup, logs y transacciones -rollback-)
\end{itemize}

Existen muchos SGDBs, pero los más populares son:

\begin{itemize}
\tightlist
\item
  \textbf{SQLite}: es muy ligero, no necesita un servidor para ejecutarse y es muy rápido cuando el conjunto de datos es pequeño.

  \begin{itemize}
  \tightlist
  \item
    Su escalabilidad y concurrencia son muy limitadas
  \item
    Ideal para proyectos pequeños, aplicaciones locales o móviles y prácticas de TXD
  \item
    Usado en Google Chrome, Firefox, Safari, Dropbox (app de escritorio)
  \item
    Es de dominio público
  \end{itemize}
\item
  \textbf{MySQL}: requiere un servidor para ejecutarse, cuenta con soporte en la nube

  \begin{itemize}
  \tightlist
  \item
    Presenta buena escalabilidad y concurrencia
  \item
    Es muy utilizado en aplicaciones web de tamaño medio
  \item
    Usado en Wordpress, y en general un gran porcentaje de páginas y aplicaciones web de tamaño pequeño y mediano. Incluso sitios grandes como Wikipedia o Facebook emplean variantes de MySQL.
  \item
    Licencia GPL (con variantes comerciales)
  \end{itemize}
\item
  \textbf{PostgreSQL}: también requiere un servidor y tiene soporte en la nube

  \begin{itemize}
  \tightlist
  \item
    Su escalabilidad es excelente, óptimo para grandes cantidades de datos
  \item
    Cuenta con muchas extensiones para, por ejemplo, información geográfica (PostGIS), series temporales (TimescaleDB), etc.
  \item
    Ideal para sistemas complejos, analíticos o con alta concurrencia
  \item
    Usado en Reddit, Instagram, Spotify, Netflix, \ldots{}
  \item
    Licencia propia tipo MIT
  \end{itemize}
\item
  Otros SGDBs relacionales: MariaDB (derivado de MySQL), Microsoft SQL Server, Oracle
\end{itemize}

Ranking de popularidad según \href{http://db-engines.com}{DB-Engines}. La puntuación está calculada en función del número de menciones en páginas web, interés general en los sitemas, frecuencia de discusiones técnias, cantidad de ofertas de trabajo, relevancia en redes sociales, etc.

\begin{longtable}[]{@{}
  >{\raggedleft\arraybackslash}p{(\linewidth - 10\tabcolsep) * \real{0.0946}}
  >{\raggedleft\arraybackslash}p{(\linewidth - 10\tabcolsep) * \real{0.0946}}
  >{\raggedright\arraybackslash}p{(\linewidth - 10\tabcolsep) * \real{0.0946}}
  >{\raggedright\arraybackslash}p{(\linewidth - 10\tabcolsep) * \real{0.4189}}
  >{\raggedright\arraybackslash}p{(\linewidth - 10\tabcolsep) * \real{0.2027}}
  >{\raggedright\arraybackslash}p{(\linewidth - 10\tabcolsep) * \real{0.0946}}@{}}
\toprule\noalign{}
\begin{minipage}[b]{\linewidth}\raggedleft
2025
\end{minipage} & \begin{minipage}[b]{\linewidth}\raggedleft
2024
\end{minipage} & \begin{minipage}[b]{\linewidth}\raggedright
SGDB
\end{minipage} & \begin{minipage}[b]{\linewidth}\raggedright
Modelo
\end{minipage} & \begin{minipage}[b]{\linewidth}\raggedright
Puntuacion
\end{minipage} & \begin{minipage}[b]{\linewidth}\raggedright
Var
\end{minipage} \\
\midrule\noalign{}
\endhead
\bottomrule\noalign{}
\endlastfoot
1. & 1. & Oracle & Relacional, Multi-modelo & 1212.77 & -96.67 \\
2. & 2. & MySQL & Relacional, Multi-modelo & 879.66 & -143.09 \\
3. & 3. & Microsoft SQL Server & Relacional, Multi-modelo & 715.05 & -87.04 \\
4. & 4. & PostgreSQL & Relacional, Multi-modelo & 643.20 & -8.96 \\
5. & 5. & MongoDB & Documental, Multi-modelo & 368.01 & -37.20 \\
6. & 7. & Snowflake & Relacional & 198.65 & +58.05 \\
7. & 6. & Redis & Clave-valor, Multi-modelo & 142.33 & -7.30 \\
8. & 14. & Databricks & Multi-modelo & 128.80 & +43.21 \\
9. & 9. & IBM Db2 & Relacional, Multi-modelo & 122.37 & -0.40 \\
10. & 8. & Elasticsearch & Multi-modelo & 116.67 & -15.18 \\
11. & 11. & Apache Cassandra & Wide column, Multi-modelo & 105.16 & +7.56 \\
12. & 10. & SQLite & Relacional & 104.56 & +2.64 \\
13. & 15. & MariaDB & Relacional, Multi-modelo & 87.77 & +2.88 \\
14. & 12. & Microsoft Access & Relacional & 80.79 & -11.36 \\
15. & 17. & Amazon DynamoDB & Multi-modelo & 75.91 & +4.06 \\
\end{longtable}

\section{Sintaxis SQL}\label{sintaxis-sql}

SQL (Structured Query Language) es un lenguaje declarativo.
Es un lenguaje estándar: tiene un estándar oficial definido por ISO y ANSI.
Sin embargo, en la práctica cada SGDB implementa solo una parte de él, y además tiene sus propios dialectos, en los que puede variar por ejemplo lo siguiente:

\begin{itemize}
\tightlist
\item
  Tipos de datos (TEXT, VARCHAR, BLOB, etc.)
\item
  Distinción o no de mayúsculas y minúsculas (cap sensitiveness)
\item
  Cómo se manejan las transacciones
\item
  Funciones (LENGTH(), LEN(), etc.)
\item
  Formato de fechas
\item
  Gestión de triggers
\item
  \ldots{}
\end{itemize}

Sintaxis general:

\begin{itemize}
\tightlist
\item
  Consulta
\end{itemize}

\begin{Shaded}
\begin{Highlighting}[]
  \KeywordTok{SELECT} \OperatorTok{\textless{}}\NormalTok{campo}\OperatorTok{/}\NormalTok{s}\OperatorTok{\textgreater{}}
  \KeywordTok{FROM} \OperatorTok{\textless{}}\NormalTok{tabla}\OperatorTok{\textgreater{}}
  \KeywordTok{WHERE} \OperatorTok{\textless{}}\NormalTok{condición}\OperatorTok{\textgreater{}}
  \KeywordTok{GROUP} \KeywordTok{BY} \OperatorTok{\textless{}}\NormalTok{campo}\OperatorTok{\textgreater{}}
  \KeywordTok{HAVING} \OperatorTok{\textless{}}\NormalTok{condición}\OperatorTok{\textgreater{}}
  \KeywordTok{ORDER} \KeywordTok{BY} \OperatorTok{\textless{}}\NormalTok{campo}\OperatorTok{\textgreater{}}
  \KeywordTok{LIMIT} \OperatorTok{\textless{}}\NormalTok{m}\OperatorTok{\textgreater{}}\NormalTok{ OFFSET }\OperatorTok{\textless{}}\NormalTok{n}\OperatorTok{\textgreater{}}
\end{Highlighting}
\end{Shaded}

\begin{itemize}
\tightlist
\item
  Modificación
\end{itemize}

\begin{Shaded}
\begin{Highlighting}[]
  \KeywordTok{UPDATE} \OperatorTok{\textless{}}\NormalTok{tabla}\OperatorTok{\textgreater{}}
  \KeywordTok{SET} \OperatorTok{\textless{}}\NormalTok{cambios}\OperatorTok{\textgreater{}}
  \KeywordTok{WHERE} \OperatorTok{\textless{}}\NormalTok{condición}\OperatorTok{\textgreater{}}
\end{Highlighting}
\end{Shaded}

\begin{itemize}
\tightlist
\item
  Borrado
\end{itemize}

\begin{Shaded}
\begin{Highlighting}[]
  \KeywordTok{DELETE} \KeywordTok{FROM} \OperatorTok{\textless{}}\NormalTok{tabla}\OperatorTok{\textgreater{}}
  \KeywordTok{WHERE} \OperatorTok{\textless{}}\NormalTok{condición}\OperatorTok{\textgreater{}}
\end{Highlighting}
\end{Shaded}

\section{Cláusulas básicas de SQL}\label{cluxe1usulas-buxe1sicas-de-sql}

\subsection{Lectura}\label{lectura}

\begin{itemize}
\tightlist
\item
  Seleccionar todas las columnas de una tabla:
\end{itemize}

\begin{Shaded}
\begin{Highlighting}[]
  \KeywordTok{SELECT} \OperatorTok{*} \KeywordTok{FROM}\NormalTok{ Track;  }
\end{Highlighting}
\end{Shaded}

\begin{itemize}
\tightlist
\item
  Seleccionar columnas específicas:
\end{itemize}

\begin{Shaded}
\begin{Highlighting}[]
  \KeywordTok{SELECT}\NormalTok{ name, composer }\KeywordTok{FROM}\NormalTok{ Track; }
\end{Highlighting}
\end{Shaded}

\begin{itemize}
\tightlist
\item
  Alias de columna y tabla:
\end{itemize}

\begin{Shaded}
\begin{Highlighting}[]
  \KeywordTok{SELECT}\NormalTok{ name }\KeywordTok{AS}\NormalTok{ Canción }\KeywordTok{FROM}\NormalTok{ Track; }

  \KeywordTok{SELECT}\NormalTok{ T.name }\KeywordTok{FROM}\NormalTok{ Track }\KeywordTok{AS}\NormalTok{ T;}
\end{Highlighting}
\end{Shaded}

\begin{itemize}
\tightlist
\item
  Funciones de agregación:
\end{itemize}

\begin{Shaded}
\begin{Highlighting}[]
  \KeywordTok{SELECT} \FunctionTok{COUNT}\NormalTok{(}\OperatorTok{*}\NormalTok{), }\FunctionTok{SUM}\NormalTok{(UnitPrice), }\FunctionTok{MIN}\NormalTok{(UnitPrice), }\FunctionTok{MAX}\NormalTok{(UnitPrice) }\KeywordTok{FROM}\NormalTok{ Track; }

  \KeywordTok{SELECT} \FunctionTok{AVG}\NormalTok{(milliseconds) }\KeywordTok{AS} \StringTok{\textquotesingle{}Duración Media\textquotesingle{}} \KeywordTok{FROM}\NormalTok{ Track; }
\end{Highlighting}
\end{Shaded}

\begin{itemize}
\tightlist
\item
  Filtrado de duplicados:
\end{itemize}

\begin{Shaded}
\begin{Highlighting}[]
  \KeywordTok{SELECT} \KeywordTok{DISTINCT}\NormalTok{ FirstName }\KeywordTok{FROM}\NormalTok{ Customer; }

  \KeywordTok{SELECT} \FunctionTok{COUNT}\NormalTok{(}\KeywordTok{DISTINCT}\NormalTok{ FirstName) }\KeywordTok{FROM}\NormalTok{ Customer; }
\end{Highlighting}
\end{Shaded}

\begin{itemize}
\tightlist
\item
  Formato:
\end{itemize}

\begin{Shaded}
\begin{Highlighting}[]
  \KeywordTok{SELECT} \FunctionTok{CONCAT}\NormalTok{(FirstName, }\StringTok{\textquotesingle{} \textquotesingle{}}\NormalTok{, LastName) }\KeywordTok{AS}\NormalTok{ Nombre }\KeywordTok{FROM}\NormalTok{ Employee; }

  \KeywordTok{SELECT}\NormalTok{ (FirstName }\OperatorTok{||} \StringTok{\textquotesingle{} \textquotesingle{}} \OperatorTok{||}\NormalTok{ LastName) }\KeywordTok{AS}\NormalTok{ Nombre }\KeywordTok{FROM}\NormalTok{ Employee; }

  \KeywordTok{SELECT} \FunctionTok{ROUND}\NormalTok{(}\FunctionTok{AVG}\NormalTok{(Total), }\DecValTok{2}\NormalTok{) }\KeywordTok{AS} \StringTok{\textquotesingle{}Facturacion Media\textquotesingle{}} \KeywordTok{FROM}\NormalTok{ Invoice; }
\end{Highlighting}
\end{Shaded}

\subsubsection{Filtrado de Resultados}\label{filtrado-de-resultados}

\begin{itemize}
\tightlist
\item
  Seleccionar filas con condiciones:
\end{itemize}

\begin{Shaded}
\begin{Highlighting}[]
\KeywordTok{SELECT}\NormalTok{ name }\KeywordTok{FROM}\NormalTok{ Track}
\KeywordTok{WHERE}\NormalTok{ UnitPrice }\OperatorTok{\textless{}} \FloatTok{2.0}\NormalTok{; }
\end{Highlighting}
\end{Shaded}

\begin{itemize}
\item
  Múltiples condiciones:

  \begin{itemize}
  \tightlist
  \item
    Operadores: `AND', `OR', `LIKE', `NOT', `IS NULL', `IS NOT NULL'
    `BETWEEN x AND y', `IN (lista)'
  \end{itemize}
\end{itemize}

\begin{Shaded}
\begin{Highlighting}[]
  \KeywordTok{SELECT}\NormalTok{ name }\KeywordTok{FROM}\NormalTok{ Track}
  \KeywordTok{WHERE}\NormalTok{ milliseconds }\OperatorTok{\textgreater{}} \DecValTok{120000} \KeywordTok{AND}\NormalTok{ UnitPrice }\OperatorTok{\textless{}} \FloatTok{2.0}\NormalTok{; }
\end{Highlighting}
\end{Shaded}

\begin{Shaded}
\begin{Highlighting}[]
  \KeywordTok{SELECT}\NormalTok{ name }\KeywordTok{FROM}\NormalTok{ Track}
  \KeywordTok{WHERE}\NormalTok{ composer }\KeywordTok{LIKE} \StringTok{\textquotesingle{}Metallica\textquotesingle{}} \KeywordTok{OR}\NormalTok{ composer }\KeywordTok{LIKE} \StringTok{\textquotesingle{}Ulrich\textquotesingle{}}\NormalTok{; }
\end{Highlighting}
\end{Shaded}

\begin{itemize}
\tightlist
\item
  Coincidencias parciales:
\end{itemize}

\begin{Shaded}
\begin{Highlighting}[]
  \KeywordTok{SELECT} \OperatorTok{*} \KeywordTok{FROM}\NormalTok{ Track}
  \KeywordTok{WHERE}\NormalTok{ name }\KeywordTok{LIKE} \StringTok{\textquotesingle{}\%Love\%\textquotesingle{}}\NormalTok{; }
\end{Highlighting}
\end{Shaded}

\begin{itemize}
\item
  `\%' : Reemplazo por un conjunto de caracteres
\item
  '\_' : Reemplazo por un caracter
\item
  Rangos:
\end{itemize}

\begin{Shaded}
\begin{Highlighting}[]
  \KeywordTok{SELECT} \OperatorTok{*} \KeywordTok{FROM}\NormalTok{ Track}
  \KeywordTok{WHERE}\NormalTok{ UnitPrice }\KeywordTok{BETWEEN} \FloatTok{0.5} \KeywordTok{AND} \FloatTok{1.5}\NormalTok{; }
\end{Highlighting}
\end{Shaded}

\begin{itemize}
\tightlist
\item
  Valores en una lista:
\end{itemize}

\begin{Shaded}
\begin{Highlighting}[]
  \KeywordTok{SELECT} \OperatorTok{*} \KeywordTok{FROM}\NormalTok{ Track}
  \KeywordTok{WHERE}\NormalTok{ composer }\KeywordTok{IN}\NormalTok{ (}\StringTok{\textquotesingle{}Metallica\textquotesingle{}}\NormalTok{, }\StringTok{\textquotesingle{}Ulrich\textquotesingle{}}\NormalTok{); }
\end{Highlighting}
\end{Shaded}

\subsubsection{Ordenar Resultados}\label{ordenar-resultados}

\begin{itemize}
\tightlist
\item
  Ordenar por una columna:
\end{itemize}

\begin{Shaded}
\begin{Highlighting}[]
  \KeywordTok{SELECT} \OperatorTok{*} \KeywordTok{FROM}\NormalTok{ Track}
  \KeywordTok{ORDER} \KeywordTok{BY}\NormalTok{ title }\KeywordTok{ASC}\NormalTok{; }

  \KeywordTok{SELECT} \OperatorTok{*} \KeywordTok{FROM}\NormalTok{ Track}
  \KeywordTok{ORDER} \KeywordTok{BY}\NormalTok{ title }\KeywordTok{DESC}\NormalTok{; }
\end{Highlighting}
\end{Shaded}

\begin{itemize}
\tightlist
\item
  Ordenar por múltiples columnas:
\end{itemize}

\begin{Shaded}
\begin{Highlighting}[]
  \KeywordTok{SELECT} \OperatorTok{*} \KeywordTok{FROM}\NormalTok{ Track}
  \KeywordTok{ORDER} \KeywordTok{BY}\NormalTok{ composer }\KeywordTok{ASC}\NormalTok{, title }\KeywordTok{DESC}\NormalTok{; }
\end{Highlighting}
\end{Shaded}

\subsubsection{Número de filas (paginación)}\label{nuxfamero-de-filas-paginaciuxf3n}

\begin{itemize}
\tightlist
\item
  Obtener las N primeras:
\end{itemize}

\begin{Shaded}
\begin{Highlighting}[]
  \KeywordTok{SELECT} \OperatorTok{*} \KeywordTok{FROM}\NormalTok{ Track}
  \KeywordTok{ORDER} \KeywordTok{BY}\NormalTok{ title }\KeywordTok{ASC}
  \KeywordTok{LIMIT} \DecValTok{5}\NormalTok{;}
\end{Highlighting}
\end{Shaded}

\begin{itemize}
\tightlist
\item
  Obtener las N filas siguientes:
\end{itemize}

\begin{Shaded}
\begin{Highlighting}[]
  \KeywordTok{SELECT} \OperatorTok{*} \KeywordTok{FROM}\NormalTok{ Track}
  \KeywordTok{ORDER} \KeywordTok{BY}\NormalTok{ title }\KeywordTok{ASC}
\NormalTok{  OFFSET }\DecValTok{5} \KeywordTok{LIMIT} \DecValTok{5}\NormalTok{;}
\end{Highlighting}
\end{Shaded}

\section{Gestión de tablas}\label{gestiuxf3n-de-tablas}

\begin{itemize}
\tightlist
\item
  Creación de tablas
\end{itemize}

\begin{Shaded}
\begin{Highlighting}[]
\KeywordTok{CREATE} \KeywordTok{TABLE}\NormalTok{ table\_name(}
\NormalTok{column1 datatype,}
\NormalTok{column2 datatype,}
\NormalTok{column3 datatype,}
\OperatorTok{....}\NormalTok{.}
\NormalTok{columnN datatype,}
\KeywordTok{PRIMARY} \KeywordTok{KEY}\NormalTok{( one }\KeywordTok{or}\NormalTok{ more }\KeywordTok{columns}\NormalTok{ )}
\NormalTok{);}
\end{Highlighting}
\end{Shaded}

\begin{itemize}
\tightlist
\item
  Borrado de tablas
\end{itemize}

\begin{Shaded}
\begin{Highlighting}[]
\KeywordTok{DROP} \KeywordTok{TABLE}\NormalTok{ table\_name;}
\end{Highlighting}
\end{Shaded}

\begin{itemize}
\tightlist
\item
  Creación de índices
\end{itemize}

\begin{Shaded}
\begin{Highlighting}[]
\KeywordTok{CREATE} \KeywordTok{UNIQUE} \KeywordTok{INDEX}\NormalTok{ index\_name}
\KeywordTok{ON}\NormalTok{ table\_name ( column1, column2,}\OperatorTok{..}\NormalTok{.columnN);}
\end{Highlighting}
\end{Shaded}

\begin{itemize}
\tightlist
\item
  Modificación de tablas
\end{itemize}

\begin{Shaded}
\begin{Highlighting}[]
\KeywordTok{ALTER} \KeywordTok{TABLE}\NormalTok{ table\_name}
\KeywordTok{DROP} \KeywordTok{INDEX}\NormalTok{ index\_name;}

\KeywordTok{ALTER} \KeywordTok{TABLE}\NormalTok{ table\_name}
\NormalTok{\{}\KeywordTok{ADD}\NormalTok{|}\KeywordTok{DROP}\NormalTok{|}\KeywordTok{MODIFY}\NormalTok{\} column\_name \{data\_ype\};}

\KeywordTok{ALTER} \KeywordTok{TABLE}\NormalTok{ table\_name }\KeywordTok{RENAME} \KeywordTok{TO}\NormalTok{ new\_table\_name;}
\end{Highlighting}
\end{Shaded}

\section{Gestión de datos}\label{gestiuxf3n-de-datos}

\begin{itemize}
\tightlist
\item
  Inserción de tuplas
\end{itemize}

\begin{Shaded}
\begin{Highlighting}[]
\KeywordTok{INSERT} \KeywordTok{INTO}\NormalTok{ table\_name( column1, column2}\OperatorTok{....}\NormalTok{columnN)}
\KeywordTok{VALUES}\NormalTok{ ( value1, value2}\OperatorTok{....}\NormalTok{valueN);}
\end{Highlighting}
\end{Shaded}

\begin{itemize}
\tightlist
\item
  Modificación de datos
\end{itemize}

\begin{Shaded}
\begin{Highlighting}[]
\KeywordTok{UPDATE}\NormalTok{ table\_name}
\KeywordTok{SET}\NormalTok{ column1 }\OperatorTok{=}\NormalTok{ value1, column2 }\OperatorTok{=}\NormalTok{ value2}\OperatorTok{....}\NormalTok{columnN}\OperatorTok{=}\NormalTok{valueN}
\NormalTok{[ }\KeywordTok{WHERE}\NormalTok{  CONDITION ];}
\end{Highlighting}
\end{Shaded}

\begin{itemize}
\tightlist
\item
  Eliminación de datos
\end{itemize}

\begin{Shaded}
\begin{Highlighting}[]
\KeywordTok{DELETE} \KeywordTok{FROM}\NormalTok{ table\_name}
\KeywordTok{WHERE}\NormalTok{  \{CONDITION\};}
\end{Highlighting}
\end{Shaded}

\section{Gestión de Bases de Datos}\label{gestiuxf3n-de-bases-de-datos}

\begin{itemize}
\tightlist
\item
  Creación de una base de datos
\end{itemize}

\begin{Shaded}
\begin{Highlighting}[]
\KeywordTok{CREATE} \KeywordTok{DATABASE}\NormalTok{ database\_name;}
\end{Highlighting}
\end{Shaded}

\begin{itemize}
\tightlist
\item
  Eliminación de una base de datos
\end{itemize}

\begin{Shaded}
\begin{Highlighting}[]
\KeywordTok{DROP} \KeywordTok{DATABASE}\NormalTok{ database\_name;}
\end{Highlighting}
\end{Shaded}

\begin{itemize}
\tightlist
\item
  Selección de base de datos
\end{itemize}

\begin{Shaded}
\begin{Highlighting}[]
\KeywordTok{USE}\NormalTok{ database\_name;}
\end{Highlighting}
\end{Shaded}

\begin{itemize}
\tightlist
\item
  Gestión de transacciones
\end{itemize}

\begin{Shaded}
\begin{Highlighting}[]
\ControlFlowTok{BEGIN}\NormalTok{;}

  \OperatorTok{..}\NormalTok{.}
  
\KeywordTok{COMMIT}\NormalTok{;}

\KeywordTok{ROLLBACK}\NormalTok{;}
\end{Highlighting}
\end{Shaded}

\section{Ejemplos de consultas SQL}\label{ejemplos-de-consultas-sql}

\begin{Shaded}
\begin{Highlighting}[]
\KeywordTok{SELECT}\NormalTok{ Nombre, Apellido1, Apellido2, Municipio, Provincia }
\KeywordTok{FROM}\NormalTok{ Cliente}
\KeywordTok{WHERE}\NormalTok{ Municipio }\OperatorTok{=} \StringTok{\textquotesingle{}Lugo\textquotesingle{}}
\KeywordTok{ORDER} \KeywordTok{BY}\NormalTok{ Apellido1}

\KeywordTok{INSERT}\NormalTok{ Proveedor(Nombre, PersonaContacto, Ciudad, País)}
\KeywordTok{VALUES}\NormalTok{ (}\StringTok{\textquotesingle{}Café Candelas\textquotesingle{}}\NormalTok{, }\StringTok{\textquotesingle{}Ivana Candelas\textquotesingle{}}\NormalTok{, }\StringTok{\textquotesingle{}Lugo\textquotesingle{}}\NormalTok{, }\StringTok{\textquotesingle{}España\textquotesingle{}}\NormalTok{)}

\KeywordTok{UPDATE}\NormalTok{ Pedidos}
\KeywordTok{SET}\NormalTok{ Cantidad }\OperatorTok{=} \DecValTok{2}
\KeywordTok{WHERE}\NormalTok{ IdProducto }\OperatorTok{=} \DecValTok{963}

\KeywordTok{DELETE}\NormalTok{ Cliente}
\KeywordTok{WHERE}\NormalTok{ Email }\OperatorTok{=} \StringTok{\textquotesingle{}alexandregb@gmail.com\textquotesingle{}}
\end{Highlighting}
\end{Shaded}

\section{Conexión con bases de datos desde R}\label{conexiuxf3n-con-bases-de-datos-desde-r}

\subsection{Introducción a SQL en R}\label{introducciuxf3n-a-sql-en-r}

SQL se usa para manipular datos dentro de una base de datos. Si la base de datos no es muy grande se puede cargar toda en un data.frame.
No obstante, por escalabilidad y offloading de la carga de trabajo al servidor SGBD utilizaremos SQL.

Existen varios SGBD (SQLite, Microsoft SQL Server, MySQL, PostgreSQL, etc) los cuales comparten el soporte de SQL (en concreto ANSI SQL) aunque cada gestor extiende SQL
de formas sutiles buscando minar cierta portabilidad de código (\emph{vendor-locking}). En efecto, un código SQL desarrollado para SQLite es probable que falle con MySQL
aunque tras aplicar ligeras modificaciones ya funcionará. Asimismo el mecanismo de conexión, configuración, rendimiento y operación suele diferir entre SGBD.

A continuación se lista una serie de paquetes utilizados en el acceso a los datos, lo que suele ser el principal esfuerzo a realizar cuando se trabaja con SGBD:

\begin{itemize}
\tightlist
\item
  \href{https://cran.r-project.org/web/packages/DBI/index.html}{DBI}
\item
  \href{https://cran.r-project.org/web/packages/RODBC/index.html}{RODBC}
\item
  \href{https://cran.r-project.org/web/packages/dbConnect/index.html}{dbConnect}
\item
  \href{https://cran.r-project.org/web/packages/RSQLite/index.html}{RSQLite}
\item
  \href{https://cran.r-project.org/web/packages/RMySQL/index.html}{RMySQL}
\item
  \href{https://cran.r-project.org/web/packages/RPostgreSQL/index.html}{RPostgreSQL}
\end{itemize}

\subsection{El paquete sqldf}\label{el-paquete-sqldf}

A continuación se presenta una serie de ejercicios con la sintaxis de SQL operando sobre un data.frame con el paquete sqldf. Esto inicialmente no incluye los detalles de conectarse a un SGBD, ni modificar los
datos, solamente el uso de SQL para extraer datos con el objetivo de ser analizados en R.

\begin{Shaded}
\begin{Highlighting}[]
\FunctionTok{library}\NormalTok{(sqldf)}
\end{Highlighting}
\end{Shaded}

\begin{Shaded}
\begin{Highlighting}[]
\FunctionTok{sqldf}\NormalTok{(}\StringTok{\textquotesingle{}SELECT age, circumference FROM Orange WHERE Tree = 1 ORDER BY circumference ASC\textquotesingle{}}\NormalTok{)}
\end{Highlighting}
\end{Shaded}

\begin{verbatim}
##    age circumference
## 1  118            30
## 2  484            58
## 3  664            87
## 4 1004           115
## 5 1231           120
## 6 1372           142
## 7 1582           145
\end{verbatim}

\subsection{SQL Queries}\label{sql-queries}

El comando inicial es SELECT. SQL no es case-sensitive, por lo que esto va a funcionar:

\begin{Shaded}
\begin{Highlighting}[]
\FunctionTok{sqldf}\NormalTok{(}\StringTok{"SELECT * FROM iris"}\NormalTok{)}
\FunctionTok{sqldf}\NormalTok{(}\StringTok{"select * from iris"}\NormalTok{)}
\end{Highlighting}
\end{Shaded}

pero lo siguiente no va a funcionar (a menos que tengamos un objeto IRIS:

\begin{Shaded}
\begin{Highlighting}[]
\FunctionTok{sqldf}\NormalTok{(}\StringTok{"SELECT * FROM IRIS"}\NormalTok{)}
\end{Highlighting}
\end{Shaded}

La sintaxis básica de SELECT es:

\begin{Shaded}
\begin{Highlighting}[]
\NormalTok{SELECT variable1, variable2 FROM data}
\end{Highlighting}
\end{Shaded}

\subsubsection{Asterisco/Wildcard}\label{asteriscowildcard}

Lo extrae todo

\begin{Shaded}
\begin{Highlighting}[]
\NormalTok{bod2 }\OtherTok{\textless{}{-}} \FunctionTok{sqldf}\NormalTok{(}\StringTok{\textquotesingle{}SELECT * FROM BOD\textquotesingle{}}\NormalTok{)}
\end{Highlighting}
\end{Shaded}

\subsubsection{Limit}\label{limit}

Limita el número de resultados

\begin{Shaded}
\begin{Highlighting}[]
\FunctionTok{sqldf}\NormalTok{(}\StringTok{\textquotesingle{}SELECT * FROM iris LIMIT 5\textquotesingle{}}\NormalTok{)}
\end{Highlighting}
\end{Shaded}

\begin{verbatim}
##   Sepal.Length Sepal.Width Petal.Length Petal.Width Species
## 1          5.1         3.5          1.4         0.2  setosa
## 2          4.9         3.0          1.4         0.2  setosa
## 3          4.7         3.2          1.3         0.2  setosa
## 4          4.6         3.1          1.5         0.2  setosa
## 5          5.0         3.6          1.4         0.2  setosa
\end{verbatim}

\subsubsection{Order By}\label{order-by}

Ordena las variables

\begin{Shaded}
\begin{Highlighting}[]
\NormalTok{ORDER BY var1 \{ASC}\SpecialCharTok{/}\NormalTok{DESC\}, var2 \{ASC}\SpecialCharTok{/}\NormalTok{DESC\}}
\end{Highlighting}
\end{Shaded}

\begin{Shaded}
\begin{Highlighting}[]
\FunctionTok{sqldf}\NormalTok{(}\StringTok{"SELECT * FROM Orange ORDER BY age ASC, circumference DESC LIMIT 5"}\NormalTok{)}
\end{Highlighting}
\end{Shaded}

\begin{verbatim}
##   Tree age circumference
## 1    2 118            33
## 2    4 118            32
## 3    1 118            30
## 4    3 118            30
## 5    5 118            30
\end{verbatim}

\subsubsection{Where}\label{where}

Sentencias condicionales, donde se puede incorporar operadores lógicos AND y OR, expresando el orden de evaluación con paréntesis en caso de ser necesario.

\begin{Shaded}
\begin{Highlighting}[]
\FunctionTok{sqldf}\NormalTok{(}\StringTok{\textquotesingle{}SELECT demand FROM BOD WHERE Time \textless{} 3\textquotesingle{}}\NormalTok{)}
\end{Highlighting}
\end{Shaded}

\begin{verbatim}
##   demand
## 1    8.3
## 2   10.3
\end{verbatim}

\begin{Shaded}
\begin{Highlighting}[]
\FunctionTok{sqldf}\NormalTok{(}\StringTok{\textquotesingle{}SELECT * FROM rock WHERE (peri \textgreater{} 5000 AND shape \textless{} .05) OR perm \textgreater{} 1000\textquotesingle{}}\NormalTok{)}
\end{Highlighting}
\end{Shaded}

\begin{verbatim}
##   area     peri    shape perm
## 1 5048  941.543 0.328641 1300
## 2 1016  308.642 0.230081 1300
## 3 5605 1145.690 0.464125 1300
## 4 8793 2280.490 0.420477 1300
\end{verbatim}

Y extendiendo su uso con IN o LIKE (es último sólo con \%), pudiendo aplicárseles el NOT:

\begin{Shaded}
\begin{Highlighting}[]
\FunctionTok{sqldf}\NormalTok{(}\StringTok{\textquotesingle{}SELECT * FROM BOD WHERE Time IN (1,7)\textquotesingle{}}\NormalTok{)}
\end{Highlighting}
\end{Shaded}

\begin{verbatim}
##   Time demand
## 1    1    8.3
## 2    7   19.8
\end{verbatim}

\begin{Shaded}
\begin{Highlighting}[]
\FunctionTok{sqldf}\NormalTok{(}\StringTok{\textquotesingle{}SELECT * FROM BOD WHERE Time NOT IN (1,7)\textquotesingle{}}\NormalTok{)}
\end{Highlighting}
\end{Shaded}

\begin{verbatim}
##   Time demand
## 1    2   10.3
## 2    3   19.0
## 3    4   16.0
## 4    5   15.6
\end{verbatim}

\begin{Shaded}
\begin{Highlighting}[]
\FunctionTok{sqldf}\NormalTok{(}\StringTok{\textquotesingle{}SELECT * FROM chickwts WHERE feed LIKE "\%bean" LIMIT 5\textquotesingle{}}\NormalTok{)}
\end{Highlighting}
\end{Shaded}

\begin{verbatim}
##   weight      feed
## 1    179 horsebean
## 2    160 horsebean
## 3    136 horsebean
## 4    227 horsebean
## 5    217 horsebean
\end{verbatim}

\begin{Shaded}
\begin{Highlighting}[]
\FunctionTok{sqldf}\NormalTok{(}\StringTok{\textquotesingle{}SELECT * FROM chickwts WHERE feed NOT LIKE "\%bean" LIMIT 5\textquotesingle{}}\NormalTok{)}
\end{Highlighting}
\end{Shaded}

\begin{verbatim}
##   weight    feed
## 1    309 linseed
## 2    229 linseed
## 3    181 linseed
## 4    141 linseed
## 5    260 linseed
\end{verbatim}

\section{Ejemplo Scopus data}\label{ejemplo-scopus-data}

Ver ejemplo \href{data/citan.zip}{\emph{citan.zip}} y Apéndice \ref{citan}.

\begin{quote}
``If your data fits in memory
there is no advantage to putting it in a database:
it will only be slower and more frustrating''

--- Hadley Wickham -- \url{https://dbplyr.tidyverse.org/articles/dbplyr.html}
\end{quote}

\section{Ejercicios SQL con RSQLite}\label{ejercicios-sql-con-rsqlite}

\subsection{Setup de RSQLite}\label{setup-de-rsqlite}

Vamos a utilizar \href{https://cran.r-project.org/web/packages/RSQLite/index.html}{RSQLite} desde Kaggle. Pero si lo queréis instalar en local La información para su instalación está \href{https://db.rstudio.com/databases/sqlite/}{en el siguiente enlace}.

\begin{Shaded}
\begin{Highlighting}[]
\FunctionTok{library}\NormalTok{(DBI)}

\CommentTok{\# Create an ephemeral in{-}memory RSQLite database}
\NormalTok{con }\OtherTok{\textless{}{-}} \FunctionTok{dbConnect}\NormalTok{(RSQLite}\SpecialCharTok{::}\FunctionTok{SQLite}\NormalTok{(), }\StringTok{":memory:"}\NormalTok{)}
\FunctionTok{dbListTables}\NormalTok{(con)}
\end{Highlighting}
\end{Shaded}

\begin{verbatim}
## character(0)
\end{verbatim}

\begin{Shaded}
\begin{Highlighting}[]
\FunctionTok{dbWriteTable}\NormalTok{(con, }\StringTok{"mtcars"}\NormalTok{, mtcars)}
\FunctionTok{dbListTables}\NormalTok{(con)}
\end{Highlighting}
\end{Shaded}

\begin{verbatim}
## [1] "mtcars"
\end{verbatim}

\begin{Shaded}
\begin{Highlighting}[]
\FunctionTok{dbListFields}\NormalTok{(con, }\StringTok{"mtcars"}\NormalTok{)}
\end{Highlighting}
\end{Shaded}

\begin{verbatim}
##  [1] "mpg"  "cyl"  "disp" "hp"   "drat" "wt"   "qsec" "vs"   "am"   "gear"
## [11] "carb"
\end{verbatim}

\begin{Shaded}
\begin{Highlighting}[]
\FunctionTok{dbReadTable}\NormalTok{(con, }\StringTok{"mtcars"}\NormalTok{)}
\end{Highlighting}
\end{Shaded}

\begin{verbatim}
##     mpg cyl  disp  hp drat    wt  qsec vs am gear carb
## 1  21.0   6 160.0 110 3.90 2.620 16.46  0  1    4    4
## 2  21.0   6 160.0 110 3.90 2.875 17.02  0  1    4    4
## 3  22.8   4 108.0  93 3.85 2.320 18.61  1  1    4    1
## 4  21.4   6 258.0 110 3.08 3.215 19.44  1  0    3    1
## 5  18.7   8 360.0 175 3.15 3.440 17.02  0  0    3    2
## 6  18.1   6 225.0 105 2.76 3.460 20.22  1  0    3    1
## 7  14.3   8 360.0 245 3.21 3.570 15.84  0  0    3    4
## 8  24.4   4 146.7  62 3.69 3.190 20.00  1  0    4    2
## 9  22.8   4 140.8  95 3.92 3.150 22.90  1  0    4    2
## 10 19.2   6 167.6 123 3.92 3.440 18.30  1  0    4    4
## 11 17.8   6 167.6 123 3.92 3.440 18.90  1  0    4    4
## 12 16.4   8 275.8 180 3.07 4.070 17.40  0  0    3    3
## 13 17.3   8 275.8 180 3.07 3.730 17.60  0  0    3    3
## 14 15.2   8 275.8 180 3.07 3.780 18.00  0  0    3    3
## 15 10.4   8 472.0 205 2.93 5.250 17.98  0  0    3    4
## 16 10.4   8 460.0 215 3.00 5.424 17.82  0  0    3    4
## 17 14.7   8 440.0 230 3.23 5.345 17.42  0  0    3    4
## 18 32.4   4  78.7  66 4.08 2.200 19.47  1  1    4    1
## 19 30.4   4  75.7  52 4.93 1.615 18.52  1  1    4    2
## 20 33.9   4  71.1  65 4.22 1.835 19.90  1  1    4    1
## 21 21.5   4 120.1  97 3.70 2.465 20.01  1  0    3    1
## 22 15.5   8 318.0 150 2.76 3.520 16.87  0  0    3    2
## 23 15.2   8 304.0 150 3.15 3.435 17.30  0  0    3    2
## 24 13.3   8 350.0 245 3.73 3.840 15.41  0  0    3    4
## 25 19.2   8 400.0 175 3.08 3.845 17.05  0  0    3    2
## 26 27.3   4  79.0  66 4.08 1.935 18.90  1  1    4    1
## 27 26.0   4 120.3  91 4.43 2.140 16.70  0  1    5    2
## 28 30.4   4  95.1 113 3.77 1.513 16.90  1  1    5    2
## 29 15.8   8 351.0 264 4.22 3.170 14.50  0  1    5    4
## 30 19.7   6 145.0 175 3.62 2.770 15.50  0  1    5    6
## 31 15.0   8 301.0 335 3.54 3.570 14.60  0  1    5    8
## 32 21.4   4 121.0 109 4.11 2.780 18.60  1  1    4    2
\end{verbatim}

\begin{Shaded}
\begin{Highlighting}[]
\CommentTok{\# You can fetch all results:}
\NormalTok{res }\OtherTok{\textless{}{-}} \FunctionTok{dbSendQuery}\NormalTok{(con, }\StringTok{"SELECT * FROM mtcars WHERE cyl = 4"}\NormalTok{)}
\FunctionTok{dbFetch}\NormalTok{(res)}
\end{Highlighting}
\end{Shaded}

\begin{verbatim}
##     mpg cyl  disp  hp drat    wt  qsec vs am gear carb
## 1  22.8   4 108.0  93 3.85 2.320 18.61  1  1    4    1
## 2  24.4   4 146.7  62 3.69 3.190 20.00  1  0    4    2
## 3  22.8   4 140.8  95 3.92 3.150 22.90  1  0    4    2
## 4  32.4   4  78.7  66 4.08 2.200 19.47  1  1    4    1
## 5  30.4   4  75.7  52 4.93 1.615 18.52  1  1    4    2
## 6  33.9   4  71.1  65 4.22 1.835 19.90  1  1    4    1
## 7  21.5   4 120.1  97 3.70 2.465 20.01  1  0    3    1
## 8  27.3   4  79.0  66 4.08 1.935 18.90  1  1    4    1
## 9  26.0   4 120.3  91 4.43 2.140 16.70  0  1    5    2
## 10 30.4   4  95.1 113 3.77 1.513 16.90  1  1    5    2
## 11 21.4   4 121.0 109 4.11 2.780 18.60  1  1    4    2
\end{verbatim}

\begin{Shaded}
\begin{Highlighting}[]
\FunctionTok{dbClearResult}\NormalTok{(res)}

\CommentTok{\# Or a chunk at a time}
\NormalTok{res }\OtherTok{\textless{}{-}} \FunctionTok{dbSendQuery}\NormalTok{(con, }\StringTok{"SELECT * FROM mtcars WHERE cyl = 4"}\NormalTok{)}
\ControlFlowTok{while}\NormalTok{(}\SpecialCharTok{!}\FunctionTok{dbHasCompleted}\NormalTok{(res))\{}
\NormalTok{  chunk }\OtherTok{\textless{}{-}} \FunctionTok{dbFetch}\NormalTok{(res, }\AttributeTok{n =} \DecValTok{5}\NormalTok{)}
  \FunctionTok{print}\NormalTok{(}\FunctionTok{nrow}\NormalTok{(chunk))}
\NormalTok{\}}
\end{Highlighting}
\end{Shaded}

\begin{verbatim}
## [1] 5
## [1] 5
## [1] 1
\end{verbatim}

\begin{Shaded}
\begin{Highlighting}[]
\CommentTok{\# Clear the result}
\FunctionTok{dbClearResult}\NormalTok{(res)}

\CommentTok{\# Disconnect from the database}
\FunctionTok{dbDisconnect}\NormalTok{(con)}
\end{Highlighting}
\end{Shaded}

\section{Práctica 1: SQL}\label{pruxe1ctica-1-sql}

Vamos a utilizar la base de datos \href{https://www.sqlitetutorial.net/wp-content/uploads/2018/03/chinook.zip}{Chinook} del \href{https://www.sqlitetutorial.net/sqlite-sample-database/}{tutorial de SQLite}

\includegraphics[width=6.25in,height=\textheight,keepaspectratio]{images/sqlite-sample-database-color.jpg}

Los ejercicios pedidos en Kaggle \href{https://www.kaggle.com/gltaboada/sqlite-tutorial-in-r}{kaggle.com/gltaboada/sqlite-tutorial-in-r} se entregarán preferentemente antes del \textbf{14/10} compartiendo un notebook con las soluciones (¡notebooke privado!) con el usuario \textbf{gltaboada}. Antes me tenéis que enviar un email comunicando qué usuario tenéis cada uno. En caso de incidencia me podéis mandar un notebook descargado (.ipynb), o el mecanismo que hayamos acordado previamente.

\chapter{Manipulación de datos con tidyverse}\label{tidyverse}

Este la primera parte de este capítulo (Sección \ref{introTidyverse}), se pretende realizar una breve introducción al \emph{ecosistema} \href{https://dplyr.tidyverse.org}{\textbf{Tidyverse}}, una colección de paquetes diseñados de forma uniforme (con la misma filosofía y estilo) para trabajar conjuntamente.

La referencia recomendada para usuarios de R que deseen iniciarse en el uso de estos paquetes es:

Wickham, H., y Grolemund, G. (2016). \emph{\href{http://r4ds.had.co.nz}{R for data science: import, tidy, transform, visualize, and model data}}, \href{https://es.r4ds.hadley.nz}{online-castellano}, \href{http://shop.oreilly.com/product/0636920034407.do}{O'Reilly}.

En las consecutivas secciones se presentan las alternativas \emph{tidyverse} a la lectura, manipulacióin y escritura de datos tratadas en Capítulo \ref{manipR}.

Más adelante, en la Sección \ref{dplyr} se realiza una breve introducción al paquete \href{https://dplyr.tidyverse.org}{\texttt{dplyr}} y en la Sección \ref{tidyr-pkg} se comentan algunas de las utilidades del paquete \href{https://tidyr.tidyverse.org}{\texttt{tidyr}} que pueden resultar de interés\footnote{Otra alternativa (más rápida) es \href{https://rdatatable.gitlab.io/data.table}{\texttt{data.table}} pero en versiones recientes ya se puede emplear desde \texttt{dplyr}, como se comenta más adelante.}. Finalmente, las secciones \ref{dplyr-join} y \ref{dbplyr} se muestra las utilizades para tratar tablas y bases de datos respectivamente.

\section{Introducción al ecosistema tidyverse}\label{introTidyverse}

El paquete \href{https://tidyverse.tidyverse.org}{\texttt{tidyverse}} está diseñado para facilitar la instalación y carga de los paquetes principales de la colección tidyverse con un solo comando.
Al instalar este paquete se instalan paquetes que forman el denominado núcleo de tidyverse (se cargan con \texttt{library(tidyverse)}):

\begin{itemize}
\tightlist
\item
  \href{https://ggplot2.tidyverse.org}{\texttt{ggplot2}}: visualización de datos.
\item
  \href{https://dplyr.tidyverse.org}{\texttt{dplyr}}: manipulación de datos.
\item
  \href{https://tidyr.tidyverse.org}{\texttt{tidyr}}: reorganización (limpieza) de datos.
\item
  \href{https://readr.tidyverse.org}{\texttt{readr}}: importación de datos.
\item
  \href{https://tibble.tidyverse.org}{\texttt{tibble}}: tablas de datos (extensión de \texttt{data.frame}).
\item
  \href{https://purrr.tidyverse.org}{\texttt{purrr}}: programación funcional.
\item
  \href{https://github.com/tidyverse/stringr}{\texttt{stringr}}: manipulación de cadenas de texto.
\item
  \href{https://github.com/tidyverse/forcats}{\texttt{forcats}}: manipulación de factores.
\item
  \href{https://github.com/tidyverse/lubridate}{\texttt{lubridate}}: manipulación de fechas y horas.
\end{itemize}

y un conjunto de paquetes recomendados:\\
- \href{https://github.com/wesm/feather}{\texttt{feather}}: almacenamiento efeciente de data frames.
- \href{https://github.com/tidyverse/haven}{\texttt{haven}}: lectura y escritura de datos de SPSS, Stata y SAS en R
- \href{https://github.com/tidyverse/modelr}{\texttt{modelr}}: crear pipelines\footnote{serie de pasos conectados (tuberías) que procesan datos y los transforman en un formato deseado para su análisis o modelado} elegantes al modelar datos en R (obsoleto). \href{https://github.com/tidymodels/broom}{\texttt{broom}}\ldots): resumenes estadísticos en formato Tidy

Otros paquetes de interés son:

\begin{itemize}
\tightlist
\item
  \href{https://github.com/tidyverse/readxl}{\texttt{readxl}}: lectura de archivos Excel.
\item
  \href{https://github.com/ropensci/writexl}{\texttt{readxl}}: exportación a Excel.
\item
  \href{https://github.com/tidyverse/hms}{\texttt{hms}}: manipulación de medidas de tiempo.
\item
  \href{https://github.com/r-lib/httr}{\texttt{httr}}: interactuar con web APIs.
\item
  \href{https://github.com/jeroen/jsonlite}{\texttt{jsonlite}}: Lectura y escritura de archivos JSON (\emph{JavaScript Object Notation}).
\item
  \href{https://github.com/tidyverse/rvest}{\texttt{rvest}}: extraación de datos (estructurados) de páginas web \emph{web scraping}.
\item
  \href{https://github.com/r-lib/xml2}{\texttt{xml2}}: lectura y escritura de archivos XML.
\item
  \href{https://github.com/tidyverse/vroom}{\texttt{vroom}}: lectura eficiente de archivos delimitados
\end{itemize}

\begin{Shaded}
\begin{Highlighting}[]
\FunctionTok{library}\NormalTok{(tidyverse)}
\end{Highlighting}
\end{Shaded}

También hay paquetes ``asociados'':

\begin{itemize}
\tightlist
\item
  \href{https://rlang.r-lib.org}{\texttt{rlang}}: herramientas para programación funcional.
\item
  \href{https://tidyselect.r-lib.org}{\texttt{tidyselect}} Sintaxis seleccionar variables (columnas).
\item
  \href{https://tune.tidymodels.org/}{\texttt{tune}}: hiperparámetros en modelos estadísticos
\item
  \href{https://tidymodels.tidymodels.org}{\texttt{tidymodels}} meta-paquete para todo el proceso de modelado.
\end{itemize}

Muchos otros paquetes están adaptando este estilo, por ejemplo, el meta paquete \href{https://tidyverts.org/}{\texttt{tidyverts}}) para el análisis de series temporales (\emph{time series}, TS), que incluye, por ejemplo:

\begin{itemize}
\tightlist
\item
  \href{https://tsibble.tidyverts.org/}{\texttt{tsibble}} (infra)estructuras de datos.
\item
  \href{https://fable.tidyverts.org/}{\texttt{fable}} predicción (\emph{forecasting}).
\item
  \href{https://feasts.tidyverts.org/}{\texttt{feasts}} extracción de características (predictores).
\end{itemize}

El paquete \href{https://github.com/robjhyndman/fpp3package}{\texttt{fpp3}} asociado al libro \href{https://otexts.com/fpp3/}{\textbf{Forecasting: Principles and Practice}} también sigue una filosofía \emph{tidy}.

Otro ejemplo, en este caso, para el tratamiento de datos espaciales, sería el paquete \href{https://github.com/r-spatial/sf/}{\texttt{sf}}, para más detalles ver \href{https://rubenfcasal.github.io/estadistica_espacial/sf-intro.html}{Sección 2.2 Introducción al paquete sf} del libro \textbf{Estadística Espacial con R}

Resumiendo, está muy de moda y puede terminar convirtiéndose en un dialecto del lenguaje R, todo lo que resulte de utilidad es bien venido\ldots{} Aunque se recomienda evitar estos paquetes en las primeras etapas de formación en R.

El estilo de programación tiene como origen la gramática de \href{https://ggplot2.tidyverse.org}{\texttt{ggplot2}} para crear gráficos de forma declarativa, basado a su vez en:

Wilkinson, L. (2005). \emph{The Grammar of Graphics}. \href{https://www.google.es/books/edition/The_Grammar_of_Graphics/YGgUswEACAAJ?hl=es}{Springer}.

Este paquete se ha convertido en un sustituto de los gráficos \href{http://lattice.r-forge.r-project.org/}{\texttt{lattice}}, de utilidad en algunos informes finales, aplicaciones para empresas, o para gráficos muy especializados. Aunque, en condiciones normales, suele ser más rápido generar o programar gráficos estándar de R.

Para iniciarse en este paquete lo recomendado es consultar los capítulos \href{https://r4ds.had.co.nz/data-visualisation.html}{Data Visualización} y \href{https://r4ds.had.co.nz/graphics-for-communication.html}{Graphics for communication} de \href{https://r4ds.had.co.nz}{R for Data Science}.
También puede resultar de interés la \href{https://github.com/rstudio/cheatsheets/blob/master/data-visualization.pdf}{chuleta}).
La referencia que cubre con mayor profundidad este paquete es:

Wickham, H. (2016). \emph{\href{https://ggplot2-book.org}{ggplot2: Elegant graphics for Data Analysis}} (3ª edición, en desarrollo junto a Navarro, D. y Pedersen, T.L.). \href{https://www.amazon.com/gp/product/331924275X}{Springer}.

Otra alternativa sería:

Chang, W. (2023). \emph{\href{https://r-graphics.org}{The R Graphics Cookbook}}. \href{https://www.amazon.com/dp/1491978600}{O'Reilly}.

En \href{https://ggplot2.tidyverse.org}{\texttt{ggplot2}} se emplea el operador \texttt{+} para añadir componentes de los gráficos (ver , en \emph{Tidyverse} se emplea un operador de redirección para añadir operaciones.

\subsection{\texorpdfstring{Operador \emph{pipe} (redirección)}{Operador pipe (redirección)}}\label{pipe}

El operador \texttt{\%\textgreater{}\%} (paquete \href{https://magrittr.tidyverse.org}{\texttt{magrittr}}) permite canalizar la salida de una función a la entrada de otra. Se utiliza para mejorar la legibilidad y la claridad del código al encadenar múltiples operaciones en una secuencia fluida
Por ejemplo, \texttt{segundo(primero(datos))} se traduce en \texttt{datos\ \%\textgreater{}\%\ primero\ \%\textgreater{}\%\ segundo}, lo que facilita la lectura de operaciones al escribir las funciones de izquierda a derecha.

Desde la versión 4.1 de R está disponible un operador interno \texttt{\textbar{}\textgreater{}}.
Por ejemplo, para el conjunto de datos \texttt{empleados.RData} que contiene datos de empleados de un banco. Supongamos, por ejemplo, que estamos interesados en estudiar si hay discriminación por cuestión de sexo o raza.

\begin{Shaded}
\begin{Highlighting}[]
\FunctionTok{load}\NormalTok{(}\StringTok{"data/empleados.RData"}\NormalTok{)}
\CommentTok{\# NOTA: Cuidado con la codificación latin1 (no declarada) }
\CommentTok{\# al abrir archivos creados en versiones anteriores de R \textless{} 4.2: }
\CommentTok{\# load("data/empleados.latin1.RData")}

\CommentTok{\# Listamos las etiquetas}
\CommentTok{\#knitr::kable(attr(empleados, "variable.labels"),}
\CommentTok{\#             col.names = "Etiqueta")}

\CommentTok{\# Eliminamos las etiquetas para que no molesten...}
\CommentTok{\# attr(empleados, "variable.labels") \textless{}{-} NULL  }

\CommentTok{\#empleados |\textgreater{}  }
\CommentTok{\#  subset(catlab == "Directivo", catlab:sexoraza) |\textgreater{}}
\CommentTok{\#  summary()}
\end{Highlighting}
\end{Shaded}

Para que una función sea compatible con este tipo de operadores el primer parámetro debería ser siempre los datos.
Sin embargo, el operador \texttt{\%\textgreater{}\%} permite redirigir el resultado de la operación anterior a un parámetro distinto mediante un \texttt{.}.
Por ejemplo:

\begin{Shaded}
\begin{Highlighting}[]
\CommentTok{\# ?"|\textgreater{}"}
\CommentTok{\# empleados |\textgreater{} subset(catlab != "Seguridad") |\textgreater{} droplevels |\textgreater{} }
\CommentTok{\#     boxplot(salario \textasciitilde{} sexo*catlab, data = .) \# ERROR}

\FunctionTok{library}\NormalTok{(magrittr)}
\NormalTok{empleados }\SpecialCharTok{\%\textgreater{}\%} 
  \FunctionTok{subset}\NormalTok{(catlab }\SpecialCharTok{!=} \StringTok{"Seguridad"}\NormalTok{) }\SpecialCharTok{\%\textgreater{}\%}
  \FunctionTok{droplevels}\NormalTok{() }\SpecialCharTok{\%\textgreater{}\%}
  \FunctionTok{boxplot}\NormalTok{(salario }\SpecialCharTok{\textasciitilde{}}\NormalTok{ sexo}\SpecialCharTok{*}\NormalTok{catlab, }\AttributeTok{data =}\NormalTok{ .)}
\end{Highlighting}
\end{Shaded}

\begin{center}\includegraphics[width=0.8\linewidth]{04-dplyr_files/figure-latex/unnamed-chunk-3-1} \end{center}

\subsection{Lectura y escritura de archivos de texto}\label{readr}

En esta seccón la alternativa \emph{tidyverse}, a la tradicional, vista en las secciones \ref{cap2-texto} y \ref{cap2-exporta} del Capítulo 2.

Para leer archivos de texto en distintos formatos se puede emplear el paquete \href{https://readr.tidyverse.org}{\texttt{readr}}, disponible en la colección de paquetes \href{https://tidyverse.tidyverse.org}{\texttt{tidyverse}}. Para más información, se recomienda consultar el \href{https://r4ds.had.co.nz/data-import.html}{Capítulo 11} del libro \href{http://r4ds.had.co.nz}{\emph{R for Data Science}} \citep{wickham2023r} o la versión en español ``\href{https://es.r4ds.hadley.nz/}{\emph{R Para Ciencia de Datos}}''.

\begin{Shaded}
\begin{Highlighting}[]
\FunctionTok{library}\NormalTok{(readr)}
\CommentTok{\# ?readr}
\NormalTok{datos }\OtherTok{\textless{}{-}} \FunctionTok{read\_csv2}\NormalTok{(}\StringTok{"./data/coches.csv"}\NormalTok{)}
\FunctionTok{class}\NormalTok{(datos) }
\end{Highlighting}
\end{Shaded}

\begin{verbatim}
## [1] "spec_tbl_df" "tbl_df"      "tbl"         "data.frame"
\end{verbatim}

También se puede importación desde Excel fácilmente:

\begin{Shaded}
\begin{Highlighting}[]
\FunctionTok{library}\NormalTok{(readxl)}
\NormalTok{datos}\OtherTok{\textless{}{-}}\FunctionTok{read\_excel}\NormalTok{(}\StringTok{"./data/coches.xlsx"}\NormalTok{)}
\FunctionTok{class}\NormalTok{(datos)}
\end{Highlighting}
\end{Shaded}

\begin{verbatim}
## [1] "tbl_df"     "tbl"        "data.frame"
\end{verbatim}

\begin{Shaded}
\begin{Highlighting}[]
\FunctionTok{excel\_sheets}\NormalTok{(}\StringTok{"./data/coches.xlsx"}\NormalTok{) }\CommentTok{\# listado de hojas}
\end{Highlighting}
\end{Shaded}

\begin{verbatim}
## [1] "coches"
\end{verbatim}

Otra alternativa, sería emplear el paquete \href{https://r-datatable.com}{\texttt{data.table}}.
La función \texttt{fread()} puede considerarse como alternativa a \texttt{read\_csv()}
cuando el proceso de lectura resulta lento, especialmente con datos numéricos pesados. ESta función intenta \emph{adivinar} automáticamente algunos argumentos sin tener que especificarse como, por ejemplo, el delimitador, las filas omitidas y la cabecera. Sin embargo, si requiere especificar el separador del decimal, como a continuación:

\begin{Shaded}
\begin{Highlighting}[]
\FunctionTok{library}\NormalTok{(data.table)}
\CommentTok{\# ?fread}
\NormalTok{datos }\OtherTok{\textless{}{-}} \FunctionTok{fread}\NormalTok{(}\AttributeTok{file =} \StringTok{"./data/coches.csv"}\NormalTok{, }\AttributeTok{dec =} \StringTok{","}\NormalTok{)}
\FunctionTok{class}\NormalTok{(datos) }
\end{Highlighting}
\end{Shaded}

\begin{verbatim}
## [1] "data.table" "data.frame"
\end{verbatim}

Para más información, se recomienda ver la viñeta \href{https://rdatatable.gitlab.io/data.table/articles/datatable-intro.html}{\emph{Introduction to data.table}}.

\subsection{Escritura}\label{writer}

Con el ecosistema \emph{tidyverse}, también con el paquete \href{https://readr.tidyverse.org}{\texttt{readr}} se puede utilizar la función \texttt{write\_csv2()}:

\begin{Shaded}
\begin{Highlighting}[]
\FunctionTok{write\_csv2}\NormalTok{(datos, }\AttributeTok{file =} \StringTok{"datos.csv"}\NormalTok{)}
\end{Highlighting}
\end{Shaded}

y como opción más rápida, se podría usar \texttt{fwrite()} del paqute \texttt{data.table}:

\begin{Shaded}
\begin{Highlighting}[]
\CommentTok{\# datos2 \textless{}{-} data.table(datos)}
\FunctionTok{fwrite}\NormalTok{(datos2, }\AttributeTok{file =} \StringTok{"datos2.csv"}\NormalTok{)}
\end{Highlighting}
\end{Shaded}

Working draft\ldots{}

En este capítulo se realiza una breve introducción al paquete \href{https://dplyr.tidyverse.org/index.html}{\texttt{dplyr}}.
Para mas información, ver por ejemplo la `vignette' del paquete\\
\href{https://cran.rstudio.com/web/packages/dplyr/vignettes/dplyr.html}{Introduction to dplyr},
o el Capítulo \href{http://r4ds.had.co.nz/transform.html}{5 Data transformation} del libro
\href{http://r4ds.had.co.nz}{R for Data Science}\footnote{Una alternativa (más rápida) es emplear
  \href{https://rdatatable.gitlab.io/data.table}{data.table}.}.

\section{Manipulación de datos con dplyr y tidyr}\label{dplyr}

En esta sección se realiza una breve introducción al paquete \href{https://dplyr.tidyverse.org}{\texttt{dplyr}} y se comentan algunas de las utilidades del paquete \href{https://tidyr.tidyverse.org}{\texttt{tidyr}} que pueden resultar de interés\footnote{Otra alternativa (más rápida) es \href{https://rdatatable.gitlab.io/data.table}{\texttt{data.table}} pero en versiones recientes ya se puede emplear desde \texttt{dplyr}, como se comenta más adelante.}.

La referencia recomendada para iniciarse en esta herramienta es el Capítulo \href{http://r4ds.had.co.nz/transform.html}{5 Data transformation} de
\href{http://r4ds.had.co.nz}{R for Data Science}.
También puede resultar de utilidad la viñeta del paquete \href{https://dplyr.tidyverse.org/articles/dplyr.html}{Introduction to dplyr} o la \href{https://posit.co/wp-content/uploads/2022/10/data-transformation-1.pdf}{chuleta} (menú de RStudio \emph{Help \textgreater{} Cheat Sheets \textgreater{} Data Transformation with dplyr}).

\subsection{El paquete dplyr}\label{dplyr-pkg}

\begin{Shaded}
\begin{Highlighting}[]
\FunctionTok{library}\NormalTok{(dplyr)}
\end{Highlighting}
\end{Shaded}

La principal ventaja de \href{https://dplyr.tidyverse.org/index.html}{\texttt{dplyr}} es que permite trabajar (de la misma forma) con datos en distintos formatos:

\begin{itemize}
\item
  \texttt{data.frame}, \href{https://tibble.tidyverse.org/}{\texttt{tibble}}.
\item
  \href{https://rdatatable.gitlab.io/data.table}{\texttt{data.table}}: extensión (paquete \emph{backend}) \href{https://dtplyr.tidyverse.org}{\texttt{dtplyr}}.
\item
  conjuntos de datos más grandes que la memoria disponible: extensiones \href{https://duckdb.org/docs/api/r}{\texttt{duckdb}} y \href{https://arrow.apache.org/docs/r/}{\texttt{arrow}} (incluyendo almacenamiento en la nube, e.g.~\href{https://aws.amazon.com/es/s3}{AWS}).
\item
  bases de datos relacionales (lenguaje SQL, locales o remotas); extensión \href{https://dbplyr.tidyverse.org}{\texttt{dbplyr}}.
\item
  grandes volúmenes de datos (incluso almacenados en múltiples servidores; ecosistema \href{http://hadoop.apache.org/}{Hadoop}/\href{https://spark.apache.org/}{Spark}): extensión \href{https://spark.rstudio.com}{\texttt{sparklyr}} (ver menú de RStudio \emph{Help \textgreater{} Cheat Sheets \textgreater{} Interfacing Spark with sparklyr}).
\end{itemize}

El paquete dplyr permite sustituir operaciones con funciones base de R (como \href{NA}{\texttt{subset}}, \href{NA}{\texttt{split}}, \href{NA}{\texttt{apply}}, \href{NA}{\texttt{sapply}}, \href{NA}{\texttt{lapply}}, \href{NA}{\texttt{tapply}}, \href{NA}{\texttt{aggregate}}\ldots) por una ``gramática'' más sencilla para la manipulación de datos.
En lugar de operar sobre vectores como la mayoría de las funciones base,
opera sobre conjuntos de datos (de forma que es compatible con el operador \texttt{\%\textgreater{}\%}).
Los principales ``verbos'' (funciones) son:

\begin{itemize}
\item
  \href{https://dplyr.tidyverse.org/reference/select.html}{\texttt{select()}}: seleccionar variables (ver también \href{https://dplyr.tidyverse.org/reference/rename.html}{\texttt{rename}}, \href{https://dplyr.tidyverse.org/reference/rename.html}{\texttt{relocate}}, \href{https://dplyr.tidyverse.org/reference/rename.html}{\texttt{pull}}).
\item
  \href{https://dplyr.tidyverse.org/reference/mutate.html}{\texttt{mutate()}}: crear variables (ver también \texttt{transmute()}).
\item
  \href{https://dplyr.tidyverse.org/reference/filter.html}{\texttt{filter()}}: seleccionar casos/filas (ver también \texttt{slice()}).
\item
  \href{https://dplyr.tidyverse.org/reference/arrange.html}{\texttt{arrange()}}: ordenar casos/filas.
\item
  \href{https://dplyr.tidyverse.org/reference/summarise.html}{\texttt{summarise()}}: resumir valores.
\item
  \href{https://dplyr.tidyverse.org/reference/group_by.html}{\texttt{group\_by()}}: permite operaciones por grupo empleando el concepto ``dividir-aplicar-combinar'' (\texttt{ungroup()} elimina el agrupamiento).
\end{itemize}

NOTA: Para entender el funcionamiento de ciertas funciones (como \href{https://dplyr.tidyverse.org/reference/rowwise.html}{\texttt{rowwise()}}) y las posibilidades en el manejo de datos, hay que tener en cuenta que un \texttt{data.frame} no es más que una lista cuyas componentes (variables) tienen la misma longitud.
Realmente las componentes también pueden ser listas de la misma longitud y, por tanto, podemos almacenar casi cualquier estructura de datos en un \texttt{data.frame}.

En la primera parte de este capítulo consideraremos solo \texttt{data.frame} por comodidad.
Emplearemos como ejemplo los datos de empleados de banca almacenados en el fichero \emph{empleados.RData} (y supondremos que estamos interesados en estudiar si hay discriminación por cuestión de sexo o raza).

\begin{Shaded}
\begin{Highlighting}[]
\FunctionTok{load}\NormalTok{(}\StringTok{"data/empleados.RData"}\NormalTok{)}
\FunctionTok{attr}\NormalTok{(empleados, }\StringTok{"variable.labels"}\NormalTok{) }\OtherTok{\textless{}{-}} \ConstantTok{NULL}                  
\end{Highlighting}
\end{Shaded}

En la Sección \ref{dbplyr} final emplearemos una base de datos relacional como ejemplo.

\subsection{Operaciones con variables (columnas)}\label{dplyr-variables}

Podemos \textbf{seleccionar variables con \href{https://dplyr.tidyverse.org/reference/select.html}{\texttt{select()}}}:

\begin{Shaded}
\begin{Highlighting}[]
\NormalTok{emplea2 }\OtherTok{\textless{}{-}}\NormalTok{ empleados }\SpecialCharTok{\%\textgreater{}\%} \FunctionTok{select}\NormalTok{(id, sexo, minoria, tiempemp, }
\NormalTok{                                salini, salario)}
\FunctionTok{head}\NormalTok{(emplea2)}
\end{Highlighting}
\end{Shaded}

\begin{verbatim}
##   id   sexo minoria tiempemp salini salario
## 1  1 Hombre      No       98  27000   57000
## 2  2 Hombre      No       98  18750   40200
## 3  3  Mujer      No       98  12000   21450
## 4  4  Mujer      No       98  13200   21900
## 5  5 Hombre      No       98  21000   45000
## 6  6 Hombre      No       98  13500   32100
\end{verbatim}

Se puede cambiar el nombre (ver también \href{https://dplyr.tidyverse.org/reference/rename.html}{\texttt{rename()}}):

\begin{Shaded}
\begin{Highlighting}[]
\NormalTok{empleados }\SpecialCharTok{\%\textgreater{}\%} \FunctionTok{select}\NormalTok{(sexo, }\AttributeTok{noblanca =}\NormalTok{ minoria, salario) }\SpecialCharTok{\%\textgreater{}\%} \FunctionTok{head}\NormalTok{()}
\end{Highlighting}
\end{Shaded}

\begin{verbatim}
##     sexo noblanca salario
## 1 Hombre       No   57000
## 2 Hombre       No   40200
## 3  Mujer       No   21450
## 4  Mujer       No   21900
## 5 Hombre       No   45000
## 6 Hombre       No   32100
\end{verbatim}

Se pueden emplear los nombres de variables como índices:

\begin{Shaded}
\begin{Highlighting}[]
\NormalTok{empleados }\SpecialCharTok{\%\textgreater{}\%} \FunctionTok{select}\NormalTok{(sexo}\SpecialCharTok{:}\NormalTok{salario) }\SpecialCharTok{\%\textgreater{}\%} \FunctionTok{head}\NormalTok{()}
\end{Highlighting}
\end{Shaded}

\begin{verbatim}
##     sexo    fechnac educ         catlab salario
## 1 Hombre 1952-02-03   15      Directivo   57000
## 2 Hombre 1958-05-23   16 Administrativo   40200
## 3  Mujer 1929-07-26   12 Administrativo   21450
## 4  Mujer 1947-04-15    8 Administrativo   21900
## 5 Hombre 1955-02-09   15 Administrativo   45000
## 6 Hombre 1958-08-22   15 Administrativo   32100
\end{verbatim}

\begin{Shaded}
\begin{Highlighting}[]
\CommentTok{\# empleados \%\textgreater{}\% select({-}(sexo:salario)) \%\textgreater{}\% head()}
\NormalTok{empleados }\SpecialCharTok{\%\textgreater{}\%} \FunctionTok{select}\NormalTok{(}\SpecialCharTok{!}\NormalTok{(sexo}\SpecialCharTok{:}\NormalTok{salario)) }\SpecialCharTok{\%\textgreater{}\%} \FunctionTok{head}\NormalTok{()}
\end{Highlighting}
\end{Shaded}

\begin{verbatim}
##   id salini tiempemp expprev minoria     sexoraza
## 1  1  27000       98     144      No Blanca varón
## 2  2  18750       98      36      No Blanca varón
## 3  3  12000       98     381      No Blanca mujer
## 4  4  13200       98     190      No Blanca mujer
## 5  5  21000       98     138      No Blanca varón
## 6  6  13500       98      67      No Blanca varón
\end{verbatim}

Se pueden emplear distintas herramientas (\emph{\href{https://tidyselect.r-lib.org/reference/language.html}{selection helpers}}) para seleccionar variables (ver paquete \href{https://tidyselect.r-lib.org}{\texttt{tidyselect}}):

\begin{itemize}
\item
  \href{https://tidyselect.r-lib.org/reference/starts_with.html}{\texttt{starts\_with}}, \href{https://tidyselect.r-lib.org/reference/starts_with.html}{\texttt{ends\_with}}, \href{https://tidyselect.r-lib.org/reference/starts_with.html}{\texttt{contains}}, \href{https://tidyselect.r-lib.org/reference/starts_with.html}{\texttt{matches}}, \href{https://tidyselect.r-lib.org/reference/starts_with.html}{\texttt{num\_range}}: variables que coincidan con un patrón.
\item
  \href{https://tidyselect.r-lib.org/reference/all_of.html}{\texttt{all\_of}}, \href{https://tidyselect.r-lib.org/reference/all_of.html}{\texttt{any\_of}}: variables de un vectores de caracteres.
\item
  \href{https://tidyselect.r-lib.org/reference/everything.html}{\texttt{everything}}, \href{https://tidyselect.r-lib.org/reference/everything.html}{\texttt{last\_col}}: todas las variables o la última variable.
\item
  \href{https://tidyselect.r-lib.org/reference/where.html}{\texttt{where()}}: a partir de una función (e.g.~\texttt{where(is.numeric)})
\end{itemize}

Por ejemplo:

\begin{Shaded}
\begin{Highlighting}[]
\NormalTok{empleados }\SpecialCharTok{\%\textgreater{}\%} \FunctionTok{select}\NormalTok{(}\FunctionTok{starts\_with}\NormalTok{(}\StringTok{"s"}\NormalTok{)) }\SpecialCharTok{\%\textgreater{}\%} \FunctionTok{head}\NormalTok{()}
\end{Highlighting}
\end{Shaded}

\begin{verbatim}
##     sexo salario salini     sexoraza
## 1 Hombre   57000  27000 Blanca varón
## 2 Hombre   40200  18750 Blanca varón
## 3  Mujer   21450  12000 Blanca mujer
## 4  Mujer   21900  13200 Blanca mujer
## 5 Hombre   45000  21000 Blanca varón
## 6 Hombre   32100  13500 Blanca varón
\end{verbatim}

Podemos \textbf{crear variables con \href{https://dplyr.tidyverse.org/reference/mutate.html}{\texttt{mutate()}}}:

\begin{Shaded}
\begin{Highlighting}[]
\NormalTok{emplea2 }\SpecialCharTok{\%\textgreater{}\%} 
  \FunctionTok{mutate}\NormalTok{(}\AttributeTok{incsal =}\NormalTok{ salario }\SpecialCharTok{{-}}\NormalTok{ salini, }\AttributeTok{tsal =}\NormalTok{ incsal}\SpecialCharTok{/}\NormalTok{tiempemp) }\SpecialCharTok{\%\textgreater{}\%} 
  \FunctionTok{head}\NormalTok{()}
\end{Highlighting}
\end{Shaded}

\begin{verbatim}
##   id   sexo minoria tiempemp salini salario incsal      tsal
## 1  1 Hombre      No       98  27000   57000  30000 306.12245
## 2  2 Hombre      No       98  18750   40200  21450 218.87755
## 3  3  Mujer      No       98  12000   21450   9450  96.42857
## 4  4  Mujer      No       98  13200   21900   8700  88.77551
## 5  5 Hombre      No       98  21000   45000  24000 244.89796
## 6  6 Hombre      No       98  13500   32100  18600 189.79592
\end{verbatim}

\subsection{Operaciones con casos (filas)}\label{dplyr-casos}

Podemos \textbf{seleccionar casos con \href{https://dplyr.tidyverse.org/reference/filter.html}{\texttt{filter()}}}:

\begin{Shaded}
\begin{Highlighting}[]
\NormalTok{emplea2 }\SpecialCharTok{\%\textgreater{}\%} \FunctionTok{filter}\NormalTok{(sexo }\SpecialCharTok{==} \StringTok{"Mujer"}\NormalTok{, minoria }\SpecialCharTok{==} \StringTok{"Sí"}\NormalTok{) }\SpecialCharTok{\%\textgreater{}\%} \FunctionTok{head}\NormalTok{()}
\end{Highlighting}
\end{Shaded}

\begin{verbatim}
## [1] id       sexo     minoria  tiempemp salini   salario 
## <0 rows> (or 0-length row.names)
\end{verbatim}

Podemos \textbf{reordenar casos con \href{https://dplyr.tidyverse.org/reference/arrange.html}{\texttt{arrange()}}}:

\begin{Shaded}
\begin{Highlighting}[]
\NormalTok{emplea2 }\SpecialCharTok{\%\textgreater{}\%} \FunctionTok{arrange}\NormalTok{(salario) }\SpecialCharTok{\%\textgreater{}\%} \FunctionTok{head}\NormalTok{()}
\end{Highlighting}
\end{Shaded}

\begin{verbatim}
##    id  sexo minoria tiempemp salini salario
## 1 378 Mujer      No       70  10200   15750
## 2 338 Mujer      No       74  10200   15900
## 3  90 Mujer      No       92   9750   16200
## 4 224 Mujer      No       82  10200   16200
## 5 411 Mujer      No       68  10200   16200
## 6 448 Mujer   S\xed       66  10200   16350
\end{verbatim}

\begin{Shaded}
\begin{Highlighting}[]
\NormalTok{emplea2 }\SpecialCharTok{\%\textgreater{}\%} \FunctionTok{arrange}\NormalTok{(}\FunctionTok{desc}\NormalTok{(salini), salario) }\SpecialCharTok{\%\textgreater{}\%} \FunctionTok{head}\NormalTok{()}
\end{Highlighting}
\end{Shaded}

\begin{verbatim}
##    id   sexo minoria tiempemp salini salario
## 1  29 Hombre      No       96  79980  135000
## 2 343 Hombre      No       73  60000  103500
## 3 205 Hombre      No       83  52500   66750
## 4 160 Hombre      No       86  47490   66000
## 5 431 Hombre      No       66  45000   86250
## 6  32 Hombre      No       96  45000  110625
\end{verbatim}

Podemos \textbf{resumir valores con \href{https://dplyr.tidyverse.org/reference/summarise.html}{\texttt{summarise()}}}:

\begin{Shaded}
\begin{Highlighting}[]
\NormalTok{empleados }\SpecialCharTok{\%\textgreater{}\%} \FunctionTok{summarise}\NormalTok{(}\AttributeTok{sal.med =} \FunctionTok{mean}\NormalTok{(salario), }\AttributeTok{n =} \FunctionTok{n}\NormalTok{())}
\end{Highlighting}
\end{Shaded}

\begin{verbatim}
##    sal.med   n
## 1 34419.57 474
\end{verbatim}

Para realizar \textbf{operaciones con múltiples variables podemos emplear \href{https://dplyr.tidyverse.org/reference/across.html}{\texttt{across()}}} (admite selección de variables \href{https://tidyselect.r-lib.org}{\texttt{tidyselect}}):

\begin{Shaded}
\begin{Highlighting}[]
\NormalTok{empleados }\SpecialCharTok{\%\textgreater{}\%} \FunctionTok{summarise}\NormalTok{(}\FunctionTok{across}\NormalTok{(}\FunctionTok{where}\NormalTok{(is.numeric), mean), }\AttributeTok{n =} \FunctionTok{n}\NormalTok{())}
\end{Highlighting}
\end{Shaded}

\begin{verbatim}
##      id     educ  salario   salini tiempemp  expprev   n
## 1 237.5 13.49156 34419.57 17016.09  81.1097 95.86076 474
\end{verbatim}

\begin{Shaded}
\begin{Highlighting}[]
\CommentTok{\# empleados \%\textgreater{}\% summarise(across(where(is.numeric) \& !id, mean), n = n())}
\end{Highlighting}
\end{Shaded}

NOTA: Esta función sustituye a las ``variantes de ámbito'' \texttt{\_at()}, \texttt{\_if()} y \texttt{\_all()} de versiones anteriores de dplyr (como \texttt{summarise\_at()}, \texttt{summarise\_if()}, \texttt{summarise\_all()}, \texttt{mutate\_at()}, \texttt{mutate\_if()}\ldots) y también el uso de \texttt{vars()}.
En el caso de \texttt{filter()} se puede emplear \href{https://dplyr.tidyverse.org/reference/across.html}{\texttt{if\_any()}} e \href{https://dplyr.tidyverse.org/reference/across.html}{\texttt{if\_all()}}.

Podemos \textbf{agrupar casos con \href{https://dplyr.tidyverse.org/reference/group_by.html}{\texttt{group\_by()}}}:

\begin{Shaded}
\begin{Highlighting}[]
\NormalTok{empleados }\SpecialCharTok{\%\textgreater{}\%} \FunctionTok{group\_by}\NormalTok{(sexo, minoria) }\SpecialCharTok{\%\textgreater{}\%} 
    \FunctionTok{summarise}\NormalTok{(}\AttributeTok{sal.med =} \FunctionTok{mean}\NormalTok{(salario), }\AttributeTok{n =} \FunctionTok{n}\NormalTok{()) }\SpecialCharTok{\%\textgreater{}\%}
    \FunctionTok{ungroup}\NormalTok{()}
\end{Highlighting}
\end{Shaded}

\begin{verbatim}
## # A tibble: 4 x 4
##   sexo   minoria sal.med     n
##   <fct>  <fct>     <dbl> <int>
## 1 Hombre "No"     44475.   194
## 2 Hombre "S\xed"  32246.    64
## 3 Mujer  "No"     26707.   176
## 4 Mujer  "S\xed"  23062.    40
\end{verbatim}

\begin{Shaded}
\begin{Highlighting}[]
\NormalTok{empleados }\SpecialCharTok{\%\textgreater{}\%} \FunctionTok{group\_by}\NormalTok{(sexo, minoria) }\SpecialCharTok{\%\textgreater{}\%} 
    \FunctionTok{summarise}\NormalTok{(}\AttributeTok{sal.med =} \FunctionTok{mean}\NormalTok{(salario), }\AttributeTok{n =} \FunctionTok{n}\NormalTok{(), }\AttributeTok{.groups =} \StringTok{"drop"}\NormalTok{)}
\end{Highlighting}
\end{Shaded}

\begin{verbatim}
## # A tibble: 4 x 4
##   sexo   minoria sal.med     n
##   <fct>  <fct>     <dbl> <int>
## 1 Hombre "No"     44475.   194
## 2 Hombre "S\xed"  32246.    64
## 3 Mujer  "No"     26707.   176
## 4 Mujer  "S\xed"  23062.    40
\end{verbatim}

\begin{Shaded}
\begin{Highlighting}[]
\CommentTok{\# dplyr \textgreater{}= 1.1.0 \# packageVersion("dplyr")}
\CommentTok{\# empleados \%\textgreater{}\% summarise(sal.med = mean(salario), n = n(), }
\CommentTok{\#                         .by = c(sexo, minoria))}
\end{Highlighting}
\end{Shaded}

Por defecto la agrupación se mantiene para el resto de operaciones, habría que emplear \texttt{ungroup()} (o el argumento \texttt{.groups\ =\ "drop"}) para eliminarla (se puede emplear \texttt{group\_vars()} o \texttt{str()} para ver la agrupación).
Desde dplyr 1.1.0 (2023-01-29) está disponible un parámetro \texttt{.by/by} en \texttt{mutate()}, \texttt{summarise()}, \texttt{filter()} y \texttt{slice()} como alternativa a agrupar y desagrupar posteriormente.
Para más detalles ver \href{https://dplyr.tidyverse.org/reference/dplyr_by.html}{Per-operation grouping with .by/by}.

\subsection{Datos faltantes}\label{tidyr-missing}

Continuamos con el ejemplo de la Sección @ref\{missing\}.
\emph{tidyverse} dispone de muchas herramientas para el tratamiento de los datos faltantes.

\begin{Shaded}
\begin{Highlighting}[]
\FunctionTok{data}\NormalTok{(}\StringTok{"airquality"}\NormalTok{)}
\NormalTok{datos }\OtherTok{\textless{}{-}}\NormalTok{ airquality}
\FunctionTok{library}\NormalTok{(visdat)}
\FunctionTok{vis\_dat}\NormalTok{(airquality)}
\CommentTok{\# vis\_miss(airquality)}
\end{Highlighting}
\end{Shaded}

Visualización (amigable) de la estrutura de datos:

\begin{Shaded}
\begin{Highlighting}[]
\FunctionTok{library}\NormalTok{(naniar)}
\FunctionTok{bind\_shadow}\NormalTok{(airquality)}
\end{Highlighting}
\end{Shaded}

\begin{verbatim}
## # A tibble: 153 x 12
##    Ozone Solar.R  Wind  Temp Month   Day Ozone_NA Solar.R_NA Wind_NA Temp_NA
##    <int>   <int> <dbl> <int> <int> <int> <fct>    <fct>      <fct>   <fct>  
##  1    41     190   7.4    67     5     1 !NA      !NA        !NA     !NA    
##  2    36     118   8      72     5     2 !NA      !NA        !NA     !NA    
##  3    12     149  12.6    74     5     3 !NA      !NA        !NA     !NA    
##  4    18     313  11.5    62     5     4 !NA      !NA        !NA     !NA    
##  5    NA      NA  14.3    56     5     5 NA       NA         !NA     !NA    
##  6    28      NA  14.9    66     5     6 !NA      NA         !NA     !NA    
##  7    23     299   8.6    65     5     7 !NA      !NA        !NA     !NA    
##  8    19      99  13.8    59     5     8 !NA      !NA        !NA     !NA    
##  9     8      19  20.1    61     5     9 !NA      !NA        !NA     !NA    
## 10    NA     194   8.6    69     5    10 NA       !NA        !NA     !NA    
## # i 143 more rows
## # i 2 more variables: Month_NA <fct>, Day_NA <fct>
\end{verbatim}

\begin{Shaded}
\begin{Highlighting}[]
\CommentTok{\# nabular(airquality)}
\end{Highlighting}
\end{Shaded}

Distribución por variables de los datos faltantes:

\begin{Shaded}
\begin{Highlighting}[]
\FunctionTok{miss\_var\_table}\NormalTok{(airquality) }
\end{Highlighting}
\end{Shaded}

\begin{verbatim}
## # A tibble: 3 x 3
##   n_miss_in_var n_vars pct_vars
##           <int>  <int>    <dbl>
## 1             0      4     66.7
## 2             7      1     16.7
## 3            37      1     16.7
\end{verbatim}

\begin{Shaded}
\begin{Highlighting}[]
\FunctionTok{prop\_miss\_case}\NormalTok{(airquality)}
\end{Highlighting}
\end{Shaded}

\begin{verbatim}
## [1] 0.2745098
\end{verbatim}

\begin{Shaded}
\begin{Highlighting}[]
\FunctionTok{gg\_miss\_upset}\NormalTok{(airquality) }
\end{Highlighting}
\end{Shaded}

\begin{center}\includegraphics[width=0.8\linewidth]{04-dplyr_files/figure-latex/unnamed-chunk-23-1} \end{center}

Distribución conjunta de los valores faltantes para la radiación solar y ozono:

\begin{Shaded}
\begin{Highlighting}[]
\FunctionTok{library}\NormalTok{(naniar)}
\FunctionTok{library}\NormalTok{(ggplot2)}
\FunctionTok{ggplot}\NormalTok{(airquality, }
       \FunctionTok{aes}\NormalTok{(}\AttributeTok{x =}\NormalTok{ Solar.R, }
           \AttributeTok{y =}\NormalTok{ Ozone)) }\SpecialCharTok{+} 
  \FunctionTok{geom\_miss\_point}\NormalTok{()}
\end{Highlighting}
\end{Shaded}

\begin{center}\includegraphics[width=0.8\linewidth]{04-dplyr_files/figure-latex/unnamed-chunk-24-1} \end{center}

Distribución mensual de los valores faltantes:

\begin{Shaded}
\begin{Highlighting}[]
\CommentTok{\# gg\_miss\_var(airquality)}
\FunctionTok{gg\_miss\_var}\NormalTok{(airquality, }\AttributeTok{facet =}\NormalTok{ Month)}
\end{Highlighting}
\end{Shaded}

\begin{center}\includegraphics[width=0.8\linewidth]{04-dplyr_files/figure-latex/unnamed-chunk-25-1} \end{center}

\section{Herramientas tidyr}\label{tidyr-pkg}

Algunas funciones del paquete \href{https://tidyr.tidyverse.org}{\texttt{tidyr}} que pueden resultar de especial interés son:

\begin{itemize}
\tightlist
\item
  \href{https://tidyr.tidyverse.org/reference/pivot_wider.html}{\texttt{pivot\_wider()}}: permite transformar valores de grupos de casos a nuevas variables.
\item
  \href{https://tidyr.tidyverse.org/reference/pivot_longer.html}{\texttt{pivot\_longer()}}: realiza la transformación inversa, colapsar varias columnas en una.
\end{itemize}

Ver la viñeta \href{https://tidyr.tidyverse.org/articles/pivot.html}{Pivoting} para más detalles.

\begin{itemize}
\tightlist
\item
  \href{https://tidyr.tidyverse.org/reference/separate.html}{\texttt{separate()}}: permite separar una columna de texto en varias (ver también \href{https://tidyr.tidyverse.org/reference/extract.html}{\texttt{extract()}}).
\end{itemize}

Ver \href{ejemplos/mortalidad/mortalidad.R}{mortalidad.R} en \href{https://github.com/rubenfcasal/book_notasr/tree/main/ejemplos}{ejemplos}.

\section{Operaciones con tablas de datos}\label{dplyr-join}

Se emplean funciones \texttt{xxx\_join()} (ver la documentación del paquete
\href{https://dplyr.tidyverse.org/reference/join.html}{Join two tbls together},
o la vignette \href{https://dplyr.tidyverse.org/articles/two-table.html}{Two-table verbs}):

\begin{itemize}
\item
  \texttt{inner\_join()}: devuelve las filas de \texttt{x} que tienen valores coincidentes en \texttt{y},
  y todas las columnas de \texttt{x} e \texttt{y}. Si hay varias coincidencias entre \texttt{x} e \texttt{y},
  se devuelven todas las combinaciones.
\item
  \texttt{left\_join()}: devuelve todas las filas de \texttt{x} y todas las columnas de \texttt{x} e \texttt{y}.
  Las filas de \texttt{x} sin correspondencia en \texttt{y} contendrán \texttt{NA} en las nuevas columnas.
  Si hay varias coincidencias entre \texttt{x} e \texttt{y}, se devuelven todas las combinaciones
  (duplicando las filas).

  \texttt{right\_join()} hace lo contrario, devuelve todas las filas de \texttt{y}.

  \texttt{full\_join()} devuelve todas las filas de \texttt{x} e \texttt{y} (duplicando o asignando \texttt{NA} si es necesario).
\item
  \texttt{semi\_join()}: devuelve las filas de \texttt{x} que tienen valores coincidentes en \texttt{y},
  manteniendo sólo las columnas de \texttt{x} (al contrario que \texttt{inner\_join()} no duplica filas).

  \texttt{anti\_join()} hace lo contrario, devuelve las filas sin correspondencia.
\end{itemize}

El parámetro \texttt{by} determina las variables clave para las correspondencias.
Si no se establece se considerarán todas las que tengan el mismo nombre en ambas tablas.
Se puede establecer a un vector de nombres coincidentes y en caso de que los nombres sean distintos a un vector con nombres de la forma \texttt{c("clave\_x"\ =\ "clave\_y")}.

Adicionalmente, si las tablas \texttt{x} e \texttt{y} tienen las mismas variables, se pueden combinar las observaciones con operaciones de conjuntos:

\begin{itemize}
\item
  \texttt{intersect(x,\ y)}: observaciones en \texttt{x} y en \texttt{y}.
\item
  \texttt{union(x,\ y)}: observaciones en \texttt{x} o \texttt{y} no duplicadas.
\item
  \texttt{setdiff(x,\ y)}: observaciones en \texttt{x} pero no en \texttt{y}.
\end{itemize}

\section{Bases de datos con dplyr}\label{dbplyr}

Para poder usar tablas en bases de datos relacionales con \texttt{dplyr} hay que emplear el paquete \href{https://dbplyr.tidyverse.org}{dbplyr} (convierte automáticamente el código de dplyr en consultas SQL).

Algunos enlaces:

\begin{itemize}
\item
  \href{https://solutions.posit.co/connections/db}{Best Practices in Working with Databases}
\item
  \href{https://dbplyr.tidyverse.org/articles/dbplyr.html}{Introduction to dbplyr}
\item
  \href{https://datacarpentry.org/R-ecology-lesson/index.html}{Data Carpentry}:
  \href{https://datacarpentry.org/R-ecology-lesson/05-r-and-databases.html}{SQL databases and R},
\item
  \href{https://intellixus.com/2018/06/29/r-and-data-when-should-we-use-relational-databases}{R and Data -- When Should we Use Relational Databases?}
\end{itemize}

\subsection{Ejemplos}\label{ejemplos}

Como ejemplo emplearemos la base de datos de \href{https://www.sqlitetutorial.net/sqlite-sample-database/}{SQLite Sample Database Tutorial}, almacenada en el archivo \href{data/chinook.db}{\emph{chinook.db}}.

\begin{Shaded}
\begin{Highlighting}[]
\CommentTok{\# install.packages(\textquotesingle{}dbplyr\textquotesingle{})}
\FunctionTok{library}\NormalTok{(dplyr)}
\FunctionTok{library}\NormalTok{(dbplyr)}
\end{Highlighting}
\end{Shaded}

En primer lugar hay que conectar la base de datos:

\begin{Shaded}
\begin{Highlighting}[]
\NormalTok{chinook }\OtherTok{\textless{}{-}}\NormalTok{ DBI}\SpecialCharTok{::}\FunctionTok{dbConnect}\NormalTok{(RSQLite}\SpecialCharTok{::}\FunctionTok{SQLite}\NormalTok{(), }\StringTok{"data/chinook.db"}\NormalTok{)}
\end{Highlighting}
\end{Shaded}

Podemos listar las tablas:

\begin{Shaded}
\begin{Highlighting}[]
\FunctionTok{src\_dbi}\NormalTok{(chinook)}
\end{Highlighting}
\end{Shaded}

\begin{verbatim}
## src:  sqlite 3.47.1 [/home/diego/UDC/Teaching/MTE/TGD/tgdbook-guillermo/data/chinook.db]
## tbls: albums, artists, customers, employees, genres, invoice_items, invoices,
##   media_types, playlist_track, playlists, sqlite_sequence, sqlite_stat1, tracks
\end{verbatim}

Para enlazar una tabla:

\begin{Shaded}
\begin{Highlighting}[]
\NormalTok{invoices }\OtherTok{\textless{}{-}} \FunctionTok{tbl}\NormalTok{(chinook, }\StringTok{"invoices"}\NormalTok{)}
\NormalTok{invoices}
\end{Highlighting}
\end{Shaded}

\begin{verbatim}
## # Source:   table<`invoices`> [?? x 9]
## # Database: sqlite 3.47.1 [/home/diego/UDC/Teaching/MTE/TGD/tgdbook-guillermo/data/chinook.db]
##    InvoiceId CustomerId InvoiceDate      BillingAddress BillingCity BillingState
##        <int>      <int> <chr>            <chr>          <chr>       <chr>       
##  1         1          2 2009-01-01 00:0~ Theodor-Heuss~ Stuttgart   <NA>        
##  2         2          4 2009-01-02 00:0~ Ullevålsveien~ Oslo        <NA>        
##  3         3          8 2009-01-03 00:0~ Grétrystraat ~ Brussels    <NA>        
##  4         4         14 2009-01-06 00:0~ 8210 111 ST NW Edmonton    AB          
##  5         5         23 2009-01-11 00:0~ 69 Salem Stre~ Boston      MA          
##  6         6         37 2009-01-19 00:0~ Berger Straße~ Frankfurt   <NA>        
##  7         7         38 2009-02-01 00:0~ Barbarossastr~ Berlin      <NA>        
##  8         8         40 2009-02-01 00:0~ 8, Rue Hanovre Paris       <NA>        
##  9         9         42 2009-02-02 00:0~ 9, Place Loui~ Bordeaux    <NA>        
## 10        10         46 2009-02-03 00:0~ 3 Chatham Str~ Dublin      Dublin      
## # i more rows
## # i 3 more variables: BillingCountry <chr>, BillingPostalCode <chr>,
## #   Total <dbl>
\end{verbatim}

Ojo \texttt{{[}??\ x\ 9{]}}: de momento no conoce el número de filas.

\begin{Shaded}
\begin{Highlighting}[]
\FunctionTok{nrow}\NormalTok{(invoices)}
\end{Highlighting}
\end{Shaded}

\begin{verbatim}
## [1] NA
\end{verbatim}

\begin{enumerate}
\def\labelenumi{\arabic{enumi}.}
\tightlist
\item
  Podemos mostrar la consulta SQL correspondiente a una operación:
\end{enumerate}

\begin{Shaded}
\begin{Highlighting}[]
\FunctionTok{show\_query}\NormalTok{(}\FunctionTok{head}\NormalTok{(invoices))}
\end{Highlighting}
\end{Shaded}

\begin{verbatim}
## <SQL>
## SELECT `invoices`.*
## FROM `invoices`
## LIMIT 6
\end{verbatim}

\begin{Shaded}
\begin{Highlighting}[]
\CommentTok{\# str(head(invoices))}
\end{Highlighting}
\end{Shaded}

Al trabajar con bases de datos, dplyr intenta ser lo más vago posible:

\begin{itemize}
\item
  No exporta datos a R a menos que se pida explícitamente (\texttt{collect()}).
\item
  Retrasa cualquier operación lo máximo posible:
  agrupa todo lo que se desea hacer y luego hace una única petición a la base de datos.
\end{itemize}

\begin{Shaded}
\begin{Highlighting}[]
\NormalTok{invoices }\SpecialCharTok{\%\textgreater{}\%}\NormalTok{ head }\SpecialCharTok{\%\textgreater{}\%}\NormalTok{ collect}
\end{Highlighting}
\end{Shaded}

\begin{verbatim}
## # A tibble: 6 x 9
##   InvoiceId CustomerId InvoiceDate       BillingAddress BillingCity BillingState
##       <int>      <int> <chr>             <chr>          <chr>       <chr>       
## 1         1          2 2009-01-01 00:00~ Theodor-Heuss~ Stuttgart   <NA>        
## 2         2          4 2009-01-02 00:00~ Ullevålsveien~ Oslo        <NA>        
## 3         3          8 2009-01-03 00:00~ Grétrystraat ~ Brussels    <NA>        
## 4         4         14 2009-01-06 00:00~ 8210 111 ST NW Edmonton    AB          
## 5         5         23 2009-01-11 00:00~ 69 Salem Stre~ Boston      MA          
## 6         6         37 2009-01-19 00:00~ Berger Straße~ Frankfurt   <NA>        
## # i 3 more variables: BillingCountry <chr>, BillingPostalCode <chr>,
## #   Total <dbl>
\end{verbatim}

\begin{Shaded}
\begin{Highlighting}[]
\NormalTok{invoices }\SpecialCharTok{\%\textgreater{}\%}\NormalTok{ count }\CommentTok{\# número de filas}
\end{Highlighting}
\end{Shaded}

\begin{verbatim}
## # Source:   SQL [?? x 1]
## # Database: sqlite 3.47.1 [/home/diego/UDC/Teaching/MTE/TGD/tgdbook-guillermo/data/chinook.db]
##       n
##   <int>
## 1   412
\end{verbatim}

\begin{enumerate}
\def\labelenumi{\arabic{enumi}.}
\setcounter{enumi}{1}
\tightlist
\item
  Por ejemplo, para obtener el importe mínimo, máximo y la media de las facturas:
\end{enumerate}

\begin{Shaded}
\begin{Highlighting}[]
\NormalTok{res }\OtherTok{\textless{}{-}}\NormalTok{ invoices }\SpecialCharTok{\%\textgreater{}\%} \FunctionTok{summarise}\NormalTok{(}\AttributeTok{min =} \FunctionTok{min}\NormalTok{(Total, }\AttributeTok{na.rm =} \ConstantTok{TRUE}\NormalTok{), }
                        \AttributeTok{max =} \FunctionTok{max}\NormalTok{(Total, }\AttributeTok{na.rm =} \ConstantTok{TRUE}\NormalTok{), }
                        \AttributeTok{med =} \FunctionTok{mean}\NormalTok{(Total, }\AttributeTok{na.rm =} \ConstantTok{TRUE}\NormalTok{))}
\CommentTok{\# show\_query(res)}
\NormalTok{res  }\SpecialCharTok{\%\textgreater{}\%}\NormalTok{ collect}
\end{Highlighting}
\end{Shaded}

\begin{verbatim}
## # A tibble: 1 x 3
##     min   max   med
##   <dbl> <dbl> <dbl>
## 1  0.99  25.9  5.65
\end{verbatim}

\begin{enumerate}
\def\labelenumi{\arabic{enumi}.}
\setcounter{enumi}{2}
\tightlist
\item
  Para obtener el total de las facturas de cada uno de los países:
\end{enumerate}

\begin{Shaded}
\begin{Highlighting}[]
\NormalTok{res }\OtherTok{\textless{}{-}}\NormalTok{ invoices }\SpecialCharTok{\%\textgreater{}\%} \FunctionTok{group\_by}\NormalTok{(BillingCountry) }\SpecialCharTok{\%\textgreater{}\%} 
          \FunctionTok{summarise}\NormalTok{(}\AttributeTok{n =} \FunctionTok{n}\NormalTok{(), }\AttributeTok{total =} \FunctionTok{sum}\NormalTok{(Total, }\AttributeTok{na.rm =} \ConstantTok{TRUE}\NormalTok{))}
\CommentTok{\# show\_query(res)}
\NormalTok{res  }\SpecialCharTok{\%\textgreater{}\%}\NormalTok{ collect}
\end{Highlighting}
\end{Shaded}

\begin{verbatim}
## # A tibble: 24 x 3
##    BillingCountry     n total
##    <chr>          <int> <dbl>
##  1 Argentina          7  37.6
##  2 Australia          7  37.6
##  3 Austria            7  42.6
##  4 Belgium            7  37.6
##  5 Brazil            35 190. 
##  6 Canada            56 304. 
##  7 Chile              7  46.6
##  8 Czech Republic    14  90.2
##  9 Denmark            7  37.6
## 10 Finland            7  41.6
## # i 14 more rows
\end{verbatim}

\begin{enumerate}
\def\labelenumi{\arabic{enumi}.}
\setcounter{enumi}{3}
\tightlist
\item
  Para obtener un listado con Nombre y Apellidos de cliente y el importe de cada una de sus facturas (Hint: WHERE customer.CustomerID=invoices.CustomerID):
\end{enumerate}

\begin{Shaded}
\begin{Highlighting}[]
\NormalTok{customers }\OtherTok{\textless{}{-}} \FunctionTok{tbl}\NormalTok{(chinook, }\StringTok{"customers"}\NormalTok{)}
\FunctionTok{tbl\_vars}\NormalTok{(customers) }
\end{Highlighting}
\end{Shaded}

\begin{verbatim}
## <dplyr:::vars>
##  [1] "CustomerId"   "FirstName"    "LastName"     "Company"      "Address"     
##  [6] "City"         "State"        "Country"      "PostalCode"   "Phone"       
## [11] "Fax"          "Email"        "SupportRepId"
\end{verbatim}

\begin{Shaded}
\begin{Highlighting}[]
\NormalTok{res }\OtherTok{\textless{}{-}}\NormalTok{ customers }\SpecialCharTok{\%\textgreater{}\%} 
  \FunctionTok{inner\_join}\NormalTok{(invoices, }\AttributeTok{by =} \StringTok{"CustomerId"}\NormalTok{) }\SpecialCharTok{\%\textgreater{}\%} 
  \FunctionTok{select}\NormalTok{(FirstName, LastName, Country, Total) }
\FunctionTok{show\_query}\NormalTok{(res)}
\end{Highlighting}
\end{Shaded}

\begin{verbatim}
## <SQL>
## SELECT `FirstName`, `LastName`, `Country`, `Total`
## FROM `customers`
## INNER JOIN `invoices`
##   ON (`customers`.`CustomerId` = `invoices`.`CustomerId`)
\end{verbatim}

\begin{Shaded}
\begin{Highlighting}[]
\NormalTok{res  }\SpecialCharTok{\%\textgreater{}\%}\NormalTok{ collect}
\end{Highlighting}
\end{Shaded}

\begin{verbatim}
## # A tibble: 412 x 4
##    FirstName LastName  Country Total
##    <chr>     <chr>     <chr>   <dbl>
##  1 Luís      Gonçalves Brazil   3.98
##  2 Luís      Gonçalves Brazil   3.96
##  3 Luís      Gonçalves Brazil   5.94
##  4 Luís      Gonçalves Brazil   0.99
##  5 Luís      Gonçalves Brazil   1.98
##  6 Luís      Gonçalves Brazil  13.9 
##  7 Luís      Gonçalves Brazil   8.91
##  8 Leonie    Köhler    Germany  1.98
##  9 Leonie    Köhler    Germany 13.9 
## 10 Leonie    Köhler    Germany  8.91
## # i 402 more rows
\end{verbatim}

\begin{enumerate}
\def\labelenumi{\arabic{enumi}.}
\setcounter{enumi}{4}
\tightlist
\item
  Para listar los 10 mejores clientes (aquellos a los que se les ha facturado más cantidad) indicando Nombre, Apellidos, Pais y el importe total de su facturación:
\end{enumerate}

\begin{Shaded}
\begin{Highlighting}[]
\NormalTok{customers }\SpecialCharTok{\%\textgreater{}\%} \FunctionTok{inner\_join}\NormalTok{(invoices, }\AttributeTok{by =} \StringTok{"CustomerId"}\NormalTok{) }\SpecialCharTok{\%\textgreater{}\%} \FunctionTok{group\_by}\NormalTok{(CustomerId) }\SpecialCharTok{\%\textgreater{}\%} 
    \FunctionTok{summarise}\NormalTok{(FirstName, LastName, country, }\AttributeTok{total =} \FunctionTok{sum}\NormalTok{(Total, }\AttributeTok{na.rm =} \ConstantTok{TRUE}\NormalTok{)) }\SpecialCharTok{\%\textgreater{}\%}  
    \FunctionTok{arrange}\NormalTok{(}\FunctionTok{desc}\NormalTok{(total)) }\SpecialCharTok{\%\textgreater{}\%} \FunctionTok{head}\NormalTok{(}\DecValTok{10}\NormalTok{) }\SpecialCharTok{\%\textgreater{}\%}\NormalTok{ collect}
\end{Highlighting}
\end{Shaded}

\begin{enumerate}
\def\labelenumi{\arabic{enumi}.}
\setcounter{enumi}{5}
\item
  Listar los 10 mejores clientes (aquellos a los que se les ha facturado más cantidad)
  indicando Nombre, Apellidos, Pais y el importe total de su facturación.

\begin{Shaded}
\begin{Highlighting}[]
\NormalTok{customers }\SpecialCharTok{\%\textgreater{}\%} \FunctionTok{inner\_join}\NormalTok{(invoices, }\AttributeTok{by =} \StringTok{"CustomerId"}\NormalTok{) }\SpecialCharTok{\%\textgreater{}\%} \FunctionTok{group\_by}\NormalTok{(CustomerId) }\SpecialCharTok{\%\textgreater{}\%} 
    \FunctionTok{summarise}\NormalTok{(FirstName, LastName, Country, }\AttributeTok{total =} \FunctionTok{sum}\NormalTok{(Total, }\AttributeTok{na.rm =} \ConstantTok{TRUE}\NormalTok{)) }\SpecialCharTok{\%\textgreater{}\%}  
    \FunctionTok{arrange}\NormalTok{(}\FunctionTok{desc}\NormalTok{(total)) }\SpecialCharTok{\%\textgreater{}\%} \FunctionTok{head}\NormalTok{(}\DecValTok{10}\NormalTok{) }\SpecialCharTok{\%\textgreater{}\%}\NormalTok{ collect}
\end{Highlighting}
\end{Shaded}

\begin{verbatim}
## # A tibble: 10 x 5
##    CustomerId FirstName LastName   Country        total
##         <int> <chr>     <chr>      <chr>          <dbl>
##  1          6 Helena    Holý       Czech Republic  49.6
##  2         26 Richard   Cunningham USA             47.6
##  3         57 Luis      Rojas      Chile           46.6
##  4         45 Ladislav  Kovács     Hungary         45.6
##  5         46 Hugh      O'Reilly   Ireland         45.6
##  6         24 Frank     Ralston    USA             43.6
##  7         28 Julia     Barnett    USA             43.6
##  8         37 Fynn      Zimmermann Germany         43.6
##  9          7 Astrid    Gruber     Austria         42.6
## 10         25 Victor    Stevens    USA             42.6
\end{verbatim}
\item
  Listar los géneros musicales por orden decreciente de popularidad
  (definida la popularidad como el número de canciones de ese género),
  indicando el porcentaje de las canciones de ese género.

\begin{Shaded}
\begin{Highlighting}[]
\NormalTok{tracks }\OtherTok{\textless{}{-}} \FunctionTok{tbl}\NormalTok{(chinook, }\StringTok{"tracks"}\NormalTok{)}
\NormalTok{tracks }\SpecialCharTok{\%\textgreater{}\%} \FunctionTok{inner\_join}\NormalTok{(}\FunctionTok{tbl}\NormalTok{(chinook, }\StringTok{"genres"}\NormalTok{), }\AttributeTok{by =} \StringTok{"GenreId"}\NormalTok{) }\SpecialCharTok{\%\textgreater{}\%} \FunctionTok{count}\NormalTok{(Name.y) }\SpecialCharTok{\%\textgreater{}\%} 
    \FunctionTok{arrange}\NormalTok{(}\FunctionTok{desc}\NormalTok{(n)) }\SpecialCharTok{\%\textgreater{}\%}\NormalTok{ collect }\SpecialCharTok{\%\textgreater{}\%} \FunctionTok{mutate}\NormalTok{(}\AttributeTok{freq =}\NormalTok{ n }\SpecialCharTok{/} \FunctionTok{sum}\NormalTok{(n))}
\end{Highlighting}
\end{Shaded}

\begin{verbatim}
## # A tibble: 25 x 3
##    Name.y                 n   freq
##    <chr>              <int>  <dbl>
##  1 Rock                1297 0.370 
##  2 Latin                579 0.165 
##  3 Metal                374 0.107 
##  4 Alternative & Punk   332 0.0948
##  5 Jazz                 130 0.0371
##  6 TV Shows              93 0.0265
##  7 Blues                 81 0.0231
##  8 Classical             74 0.0211
##  9 Drama                 64 0.0183
## 10 R&B/Soul              61 0.0174
## # i 15 more rows
\end{verbatim}
\item
  Listar los 10 artistas con mayor número de canciones
  de forma descendente según el número de canciones.

\begin{Shaded}
\begin{Highlighting}[]
\NormalTok{tracks }\SpecialCharTok{\%\textgreater{}\%} \FunctionTok{inner\_join}\NormalTok{(}\FunctionTok{tbl}\NormalTok{(chinook, }\StringTok{"albums"}\NormalTok{), }\AttributeTok{by =} \StringTok{"AlbumId"}\NormalTok{) }\SpecialCharTok{\%\textgreater{}\%} 
    \FunctionTok{inner\_join}\NormalTok{(}\FunctionTok{tbl}\NormalTok{(chinook, }\StringTok{"artists"}\NormalTok{), }\AttributeTok{by =} \StringTok{"ArtistId"}\NormalTok{) }\SpecialCharTok{\%\textgreater{}\%} 
    \FunctionTok{count}\NormalTok{(Name.y) }\SpecialCharTok{\%\textgreater{}\%} \FunctionTok{arrange}\NormalTok{(}\FunctionTok{desc}\NormalTok{(n)) }\SpecialCharTok{\%\textgreater{}\%}\NormalTok{ collect}
\end{Highlighting}
\end{Shaded}

\begin{verbatim}
## # A tibble: 204 x 2
##    Name.y              n
##    <chr>           <int>
##  1 Iron Maiden       213
##  2 U2                135
##  3 Led Zeppelin      114
##  4 Metallica         112
##  5 Lost               92
##  6 Deep Purple        92
##  7 Pearl Jam          67
##  8 Lenny Kravitz      57
##  9 Various Artists    56
## 10 The Office         53
## # i 194 more rows
\end{verbatim}
\end{enumerate}

Al finalizar hay que desconectar la base de datos:

\begin{Shaded}
\begin{Highlighting}[]
\NormalTok{DBI}\SpecialCharTok{::}\FunctionTok{dbDisconnect}\NormalTok{(chinook)            }
\end{Highlighting}
\end{Shaded}

\chapter{Introducción a Tecnologías NoSQL}\label{introducciuxf3n-a-tecnologuxedas-nosql}

Son tecnologías de almacenamiento de datos en servicios web altamente escalables.

\section{Conceptos y tipos de bases de datos NoSQL (documental, columnar, clave/valor y de grafos)}\label{conceptos-y-tipos-de-bases-de-datos-nosql-documental-columnar-clavevalor-y-de-grafos}

NoSQL - ``Not Only SQL'' - es una nueva categoría de bases de datos no-relacionales y altamente distribuidas.

Las bases de datos NoSQL nacen de la necesidad de:

\begin{itemize}
\item
  Simplicidad en los diseños
\item
  Escalado horizontal
\item
  Mayor control en la disponibilidad
\end{itemize}

Pero cuidado, en muchos escenarios las BBDD relacionales siguen siendo la mejor opción.

\subsection{Características de las bases de datos NoSQL}\label{caracteruxedsticas-de-las-bases-de-datos-nosql}

\begin{itemize}
\tightlist
\item
  Libre de esquemas -- no se diseñan las tablas y relaciones por adelantado, además de permitir la migración del esquema.
\item
  Proporcionan replicación a través de escalado horizontal.
\item
  Este escalado horizontal se traduce en arquitectura distribuida
\item
  Generalmente ofrecen consistencia débil
\item
  Hacen uso de estructuras de datos sencillas, normalmente pares clave/valor a bajo nivel
\item
  Suelen tener un sistema de consultas propio (o SQL-like)
\item
  Siguen el modelo BASE (\emph{B}asic Availability, Soft state, Eventual consistency) en lugar de ACID (\emph{A}tomicity, \emph{C}onsistency, \emph{I}solation, \emph{D}urability)
\end{itemize}

El modelo BASE consiste en:

\begin{itemize}
\tightlist
\item
  Basic Availability -- el sistema garantiza disponibilidad, en términos del teorema CAP.
\item
  Soft state -- el estado del sistema puede cambiar a lo largo del tiempo, incluso sin entrada. Esto es provocado por el modelo de consistencia eventual.
\item
  Eventual consistency -- el sistema alcanzará un estado consistente con el tiempo, siempre y cuando no reciba entrada durante ese tiempo.
\end{itemize}

\subsubsection{Teorema CAP}\label{teorema-cap}

Es imposible para un sistema de cómputo distribuido garantizar simultáneamente:

\begin{itemize}
\tightlist
\item
  Consistency -- Todos los nodos ven los mismos datos al mismo tiempo
\item
  Availability -- Toda petición obtiene una respuesta en caso tanto de éxito como fallo
\item
  Partition Tolerance -- El sistema seguirá funcionando ante pérdidas arbitrarias de información o fallos parciales
\end{itemize}

\pandocbounded{\includegraphics[keepaspectratio]{images/TeoremaCAP.jpg}}

Las razones para escoger NoSQL son:

\begin{itemize}
\tightlist
\item
  Analítica
\item
  Gran cantidad de escrituras, análisis en bloque
\item
  Escalabilidad
\item
  Tan fácil como añadir un nuevo nodo a la red, bajo coste.
\item
  Redundancia
\item
  Están diseñadas teniendo en cuenta la redundancia
\item
  Rápido desarrollo
\item
  Al ser schema-less o schema on-read son más flexibles que schema on-write
\item
  Flexibilidad en el almacenamiento de datos
\item
  Almacenan todo tipo de datos: texto, imágenes, BLOBs
\item
  Gran rendimiento en consultas sobre datos que no implican relaciones jerárquicas
\item
  Gran rendimiento sobre BBDD desnormalizadas
\item
  Tamaño
\item
  El tamaño del esquema de datos es demasiado grande
\item
  Muchos datos temporales fuera de almacén principal
\end{itemize}

Razones para NO escoger NoSQL:
* Consistencia y Disponibilidad de los datos son críticas
* Relaciones entre datos son importantes
+ E.g. joins numerosos y/o importantes
* En general, cuando el modelo ACID encaja mejor

\subsection{Tipos de Bases de Datos NoSQL}\label{tipos-de-bases-de-datos-nosql}

\pandocbounded{\includegraphics[keepaspectratio]{images/TiposBBDDNoSQL.png}}

\pandocbounded{\includegraphics[keepaspectratio]{images/TiposBBDDNoSQL2.png}}

\pandocbounded{\includegraphics[keepaspectratio]{images/451ResearchMap.png}}

\pandocbounded{\includegraphics[keepaspectratio]{images/DBEnginesRanking.png}}

\pandocbounded{\includegraphics[keepaspectratio]{images/451ResearchSkills.png}}

\subsection{MongoDB: NoSQL documental}\label{mongodb-nosql-documental}

\pandocbounded{\includegraphics[keepaspectratio]{images/MongoDB.jpg}}

\subsection{Redis: NoSQL key-value}\label{redis-nosql-key-value}

In-memory data structure store, útil para base de datos de login-password, sensor-valor, URL-respuesta, con una sintaxis muy sencilla:

\begin{itemize}
\tightlist
\item
  El comando SET almacena valores
\item
  SET server:name ``luna''
\item
  Recuperamos esos valores con GET
\item
  GET server:name
\item
  INCR incrementa atómicamente un valor
\item
  INCR clients
\item
  DEL elimina claves y sus valores asociados
\item
  DEL clients
\item
  TTL (Time To Live) útil para cachés
\item
  EXPIRE promocion 60
\end{itemize}

\subsection{Cassandra: NoSQL columnar}\label{cassandra-nosql-columnar}

\pandocbounded{\includegraphics[keepaspectratio]{images/BlogRDMS.png}}

\pandocbounded{\includegraphics[keepaspectratio]{images/BlogNoSQL.png}}

\subsection{Neo4j: NoSQL grafos}\label{neo4j-nosql-grafos}

\pandocbounded{\includegraphics[keepaspectratio]{images/Neo4jlogo.png}}

\pandocbounded{\includegraphics[keepaspectratio]{images/CypherQuery.png}}

\pandocbounded{\includegraphics[keepaspectratio]{images/CypherResult.png}}

\subsection{Otros: search engines}\label{otros-search-engines}

Son sistemas especializados en búsquedas, procesamiento de lenguaje natural como ElasticSearch, Solr, Splunk (logs de aplicaciones), etc\ldots{}

\section{Conexión de R a MongoDB}\label{conexiuxf3n-de-r-a-mongodb}

A través del paquete \href{https://cran.rstudio.com/web/packages/mongolite/mongolite.pdf}{mongolite}, aquí tenéis un \href{https://datascienceplus.com/using-mongodb-with-r/}{Tutorial}

\begin{Shaded}
\begin{Highlighting}[]
\FunctionTok{install.packages}\NormalTok{(}\StringTok{"mongolite"}\NormalTok{)}
\end{Highlighting}
\end{Shaded}

\begin{Shaded}
\begin{Highlighting}[]
\FunctionTok{library}\NormalTok{(mongolite)}

\CommentTok{\# Connect to a local MongoDB}

\NormalTok{my\_collection }\OtherTok{=} \FunctionTok{mongo}\NormalTok{(}\AttributeTok{collection =} \StringTok{"restaurants"}\NormalTok{, }\AttributeTok{db =} \StringTok{"Restaurants"}\NormalTok{) }\CommentTok{\# create connection, database and collection}
\NormalTok{my\_collection}\SpecialCharTok{$}\NormalTok{count}
\end{Highlighting}
\end{Shaded}

\section{Ejercicios prácticos con MongoDB}\label{ejercicios-pruxe1cticos-con-mongodb}

Estos ejercicios se pueden hacer en un notebook Kaggle accediendo a un clúster de MongoDB en el \href{https://cloud.mongodb.com}{cloud de MongoDB}. Se carga la base de datos de ejemplo y se puede hacer con la colección de restaurantes (o alternativamente con otras colleciones).

\begin{enumerate}
\def\labelenumi{\arabic{enumi}.}
\item
  Mostrar todos los documentos de la colección restaurants (u otra)
\item
  Mostrar nombre de restaurante, barrio y cocina de la colección restaurants (o los campos de otra colección)
\item
  Mostrar los primeros 5 restaurantes del barrio Bronx.
\item
  Mostrar los restaurantes con una longitud menor que -75.7541
\item
  Mostrar los restaurantes con una puntuación superior a 90
\item
  Mostrar los restaurantes de comida American o Chineese del barrio Queens.
\item
  Mostrar los restaurantes con un grado ``A'' y puntuación 9 obtenida en fecha 2014-08-11T00:00:00Z
\item
  Propón un JSON para descargar (de algún repositorio OpenData o disponible en un API), indícame la URL, si has de hacer algún proceso antes de importarlo en MongoDB, cómo lo importas, dame un pantallazo del análisis exploratorio de ese JSON y una query que harías contra ese JSON (la query en MongoDB, Compass o RmongoDB)
\end{enumerate}

\chapter{Tecnologías para el Tratamiendo de Datos Masivos}\label{tecnologuxedas-para-el-tratamiendo-de-datos-masivos}

En este apartado trataremos los siguientes epígrafes:

\begin{enumerate}
\def\labelenumi{\arabic{enumi}.}
\item
  Introducción al Aprendizaje Estadístico
\item
  Tecnologías Big Data (Hadoop, Spark, Sparklyr)
\item
  Ejercicios de análisis de datos masivos.
\end{enumerate}

\section{Introducción al Aprendizaje Estadístico}\label{introducciuxf3n-al-aprendizaje-estaduxedstico}

El material para este apartado está disponible en el
\href{https://rubenfcasal.github.io/aprendizaje_estadistico/intro-AE.html}{Capítulo 1 del libro ``Aprendizaje Estadístico''} de Rubén Fernández Casal.

Para seguir este capítulo es altamente recomendable tener instalado Rattle, para ello consultad el apéndice de instalación de R al final de este libro.

\includegraphics[width=0.6\linewidth,height=\textheight,keepaspectratio]{images/T3-CientificoDatos.png}

\section{Tecnologías Big Data (Hadoop/Spark y Visualización)}\label{tecnologuxedas-big-data-hadoopspark-y-visualizaciuxf3n}

\subsection{Tecnologías Hadoop, Spark, y Sparklyr}\label{tecnologuxedas-hadoop-spark-y-sparklyr}

A continuación se introducen los conceptos básicos de las tecnologías Hadoop, Spark y Sparklyr:

\begin{itemize}
\item
  Hadoop: framework open-source desarrollado en Java principalmente que soporta aplicaciones distribuidas sobre miles de nodos y a escala Petabyte. Está inspirado en el diseño de las operaciones de MapReduce de Google y el Google File System (GFS). Entre sus principales componentes destaca HDFS Hadoop Distributed File System, sistema de ficheros distribuido sobre múltiples nodos y accesible a nivel de aplicación. También destaca YARN como gestor de recursos, para ejecutar aplicaciones. Destacar que la versión original, Hadoop 1, estaba basada extensivamente en Map Reduce, Hadoop 2 colocó en su core a YARN y Hadoop 3 está orientado a la provisión de Plataforma como servicio y ejecución simultánea de múltiples cargas de trabajo distribuidas sobre recursos solicitados bajo demanda.
\item
  Hive: es un sistema de almancenamiento y explotación de datos del estilo de un data warehouse open source diseñado para ser ejecutado en entornos Hadoop. Permite agrupar, consultar y analizar datos almacenados en Hadoop File System y en Amazon S3 (almacenamiento de objetos en general) en esquema en estrella. Su lenguaje de consulta de datos, Hive Query Language o (HQL).
\item
  Spark: framework de computación distribuida open-source para el procesamiento de datos masivos sobre Hadoop con un paralelismo implícito sobre su estructura de datos (Resilient Distributed Dataset o RDD), permite operar en paralelo sobre una colección de datos sin saber en qué servidores están disponibles dichos datos y de una forma tolerante a fallos. Es uno de los principales frameworks de programación de entornos Hadoop al estar optimizado su procesamiento sobre memoria (en lugar de sobre archivos en disco) para obtener velocidad, tanto en sus vertientes Spark streaming y Spark SQL, como Spark Machile Learning MLlib. Dispone de interfaces en Java, Scala, Python y R, siendo las interfaces de R Rspark y Sparklyr.
\item
  SparkR: es un paquete, el primero que apareció, para conectar R con Spark. Intenta ser lo más parecida a la interfaz estándard de R de manipulación de datos.
\item
  Sparklyr: es una librería para conectar R con Spark posterior a SparkR. Intenta ser lo más parecida a dplyr y embeber SQL en las consultas, soportando una mayor cantidad de paquetes. Por este motivo es el proyecto más activo actualmente, sustituyendo a SparkR.
\end{itemize}

\pandocbounded{\includegraphics[keepaspectratio]{images/T3-ecosistema.png}}

\includegraphics[width=0.6\linewidth,height=\textheight,keepaspectratio]{images/T3-DMvsBD.jpg}

\subsection{Big Data y Machine Learning}\label{big-data-y-machine-learning}

El Machine Learning o Aprendizaje Máquina es aquella parte de la inteligencia artificial con capacidad de aprender de los datos.

\includegraphics[width=0.7\linewidth,height=\textheight,keepaspectratio]{images/T3-AI-ML.png}

\pandocbounded{\includegraphics[keepaspectratio]{images/T3-MLvsDL.png}}

\pandocbounded{\includegraphics[keepaspectratio]{images/T3-machinelearning.png}}

\pandocbounded{\includegraphics[keepaspectratio]{images/T3-machinelearningalgorithms.png}}

Y un ejemplo de cómo se trabaja en machine learning:

\includegraphics[width=0.7\linewidth,height=\textheight,keepaspectratio]{images/T3-Supervised_ML.png}

\includegraphics[width=0.7\linewidth,height=\textheight,keepaspectratio]{images/T3-ML-indicadores.png}

\subsection{Rattle como alternativa a RapidMiner en R}\label{rattle-como-alternativa-a-rapidminer-en-r}

Las instrucciones para instalar R está en el \href{https://gltaboada.github.io/tgdbook/instalaci\%C3\%B3n-de-r.html}{Apéndice 3 de este documento}

Un tutorial adecuado para introducirse en Rattle es \href{https://www.dummies.com/programming/using-rattle-iris-r-programming/}{éste}

\pandocbounded{\includegraphics[keepaspectratio]{images/T3-rattle1.png}}

Con el tutorial se pueden ver las capacidades de rattle de explorar los datos, como se puede apreciar a continuación.

\pandocbounded{\includegraphics[keepaspectratio]{images/T3-rattle2.png}}

\pandocbounded{\includegraphics[keepaspectratio]{images/T3-rattle3.png}}

\subsection{Visualización y Generación de Cuadros de Mando}\label{visualizaciuxf3n-y-generaciuxf3n-de-cuadros-de-mando}

Se sigue un tutorial de la herramienta \href{https://docs.microsoft.com/es-es/power-bi/desktop-tutorial-analyzing-sales-data-from-excel-and-an-odata-feed}{PowerBI, con datos de Excel y OData Feed}

Como documentación de se soporte se cuenta con la web de \href{https://docs.microsoft.com/es-es/power-bi/}{PowerBI} y \href{https://ccance.net/manuales/powerbi/capitulo_01_introduccion.pdf}{un tutorial adicional}

\section{Introducción al Análisis de Datos Masivos}\label{introducciuxf3n-al-anuxe1lisis-de-datos-masivos}

En primer lugar se ha de considerar explorar los datos y realizar minería con ellos, y eso es posible hacerlo vía sparklyr.

Este apartado, eminentemente práctico, lo trabajaremos a través de \href{https://www.kaggle.com/gltaboada/t3-practice3-flights}{la práctica 3 de TGD}.

\appendix


Working draft\ldots{}

\chapter{Enlaces}\label{links}

\textbf{Recursos para el aprendizaje de R} ( \url{https://rubenfcasal.github.io/post/ayuda-y-recursos-para-el-aprendizaje-de-r}
): A continuación se muestran algunos recursos que pueden ser útiles para el aprendizaje de R y la obtención de ayuda\ldots{}

\textbf{\emph{Ayuda online}}:

\begin{itemize}
\item
  Ayuda en línea sobre funciones o paquetes: \href{https://www.rdocumentation.org/}{RDocumentation}
\item
  Buscador \href{http://rseek.org/}{RSeek}
\item
  \href{http://stackoverflow.com/questions/tagged/r}{StackOverflow}
\end{itemize}

\textbf{\emph{Cursos}}:
algunos cursos gratuitos:

\begin{itemize}
\item
  \href{https://www.coursera.org/}{Coursera}:

  \begin{itemize}
  \item
    \href{https://www.coursera.org/learn/intro-data-science-programacion-estadistica-r}{Introducción a Data Science: Programación Estadística con R}
  \item
    \href{https://www.coursera.org/specializations/r}{Mastering Software Development in R}
  \end{itemize}
\end{itemize}

\begin{itemize}
\item
  \href{https://www.datacamp.com/courses}{DataCamp}:

  \begin{itemize}
  \tightlist
  \item
    \href{https://www.datacamp.com/courses/introduccion-a-r/}{Introducción a R}
  \end{itemize}
\end{itemize}

\begin{itemize}
\item
  \href{http://online.stanford.edu/courses}{Stanford online}:

  \begin{itemize}
  \tightlist
  \item
    \href{http://online.stanford.edu/course/statistical-learning}{Statistical Learning}
  \end{itemize}
\end{itemize}

\begin{itemize}
\tightlist
\item
  Curso UCA: \href{http://knuth.uca.es/moodle/course/view.php?id=51}{Introducción a R, R-commander y shiny}
\end{itemize}

\begin{itemize}
\tightlist
\item
  Curso \href{https://r-coder.com/curso-r}{R CODER}
\end{itemize}

\begin{itemize}
\tightlist
\item
  Udacity: \href{https://eu.udacity.com/course/data-analysis-with-r--ud651}{Data Analysis with R}
\end{itemize}

\begin{itemize}
\tightlist
\item
  \href{https://swirlstats.com/scn/title.html}{Swirl Courses}:
  se pueden hacer cursos desde el propio R con el paquete
  \href{https://swirlstats.com}{swirl}.
\end{itemize}

Para información sobre cursos en castellano se puede recurrir a la web de \href{http://r-es.org/}{R-Hispano} en el apartado \href{http://r-es.org/category/formacion}{formación}. Algunos de los cursos que aparecen en entradas antiguas son gratuitos.
Ver: \href{http://r-es.org/2016/02/12/cursos-masivos-y-otra-formacion-on-line-sobre-r/}{Cursos MOOC relacionados con R}.

\textbf{\emph{Libros}}

\begin{itemize}
\item
  \textbf{\emph{Iniciación}}:

  \begin{itemize}
  \item
    R for Data Science by Hadley Wickham and Garrett Grolemund
    (\href{http://r4ds.had.co.nz/}{online}, \href{http://shop.oreilly.com/product/0636920034407.do}{O'Reilly}). \href{https://es.r4ds.hadley.nz}{online-castellano}, \href{http://shop.oreilly.com/product/0636920034407.do}{O'Reilly}.
  \item
    2011 - The Art of R Programming. A Tour of Statistical Software Design,
    (\href{https://www.nostarch.com/artofr.htm}{No Starch Press})
  \item
    Hands-On Programming with R: Write Your Own Functions and Simulations,
    by Garrett Grolemund
    (\href{http://shop.oreilly.com/product/0636920028574.do}{O'Reilly})
  \end{itemize}
\item
  \textbf{\emph{Avanzados}}:

  \begin{itemize}
  \item
    Advanced R by Hadley Wickham
    (online: \href{http://adv-r.had.co.nz/}{1ª ed},
    \href{https://adv-r.hadley.nz/}{2ª ed},
    \href{https://www.amazon.com/dp/1466586966}{Chapman \& Hall})
  \item
    2008 - Software for Data Analysis: Programming with R - Chambers
    (\href{http://www.springer.com/la/book/9780387759357}{Springer})
  \item
    R packages by Hadley Wickham
    (\href{http://r-pkgs.had.co.nz/}{online},
    \href{http://shop.oreilly.com/product/0636920034421.do}{O'Reilly})
  \end{itemize}
\item
  \textbf{\emph{Bookdown}}:
  el paquete \href{https://bookdown.org}{\texttt{bookdown}} de R permite escribir libros empleando
  \href{http://rmarkdown.rstudio.com}{R Markdown} y compartirlos.
  En \url{https://bookdown.org} está disponible una selección de libros escritos con este paquete
  (un listado más completo está disponible \href{https://bookdown.org/home/archive/}{aquí}).
  Algunos libros en este formato en castellano son:

  \begin{itemize}
  \item
    \href{https://rubenfcasal.github.io/intror}{Introducción al Análisis de Datos con R}
    (disponible en el repositorio de GitHub
    \href{https://github.com/rubenfcasal/intror}{rubenfcasal/intror}).
  \item
    \href{https://rubenfcasal.github.io/simbook}{Prácticas de Simulación}
    (disponible en el repositorio de GitHub
    \href{https://github.com/rubenfcasal/simbook}{rubenfcasal/simbook}).
  \item
    \href{https://rubenfcasal.github.io/bookdown_intro/}{Escritura de libros con bookdown}
    (disponible en el repositorio de GitHub
    \href{https://github.com/rubenfcasal/bookdown_intro}{rubenfcasal/bookdown\_intro}).
  \item
    \href{https://www.datanalytics.com/libro_r/index.html}{R para profesionales de los datos: una introducción}.
  \item
    \href{https://bookdown.org/aquintela/EBE}{Estadística Básica Edulcorada}.
  \end{itemize}
\end{itemize}

\textbf{\emph{Material online}}:
en la web se puede encontrar mucho material adicional, por ejemplo:

\begin{itemize}
\item
  \href{https://www.r-project.org/other-docs.html}{CRAN: Other R documentation}
\item
  Blogs en inglés:

  \begin{itemize}
  \item
    \url{https://www.r-bloggers.com/}
  \item
    \url{https://www.littlemissdata.com/blog/rstudioconf2019}
  \item
    RStudio: \url{https://blog.rstudio.com}
  \item
    Microsoft Revolutions: \url{https://blog.revolutionanalytics.com}
  \end{itemize}
\item
  Blogs en castellano:

  \begin{itemize}
  \item
    \url{https://www.datanalytics.com}
  \item
    \url{http://oscarperpinan.github.io/R}
  \item
    \url{http://rubenfcasal.github.io}
  \end{itemize}
\item
  Listas de correo:

  \begin{itemize}
  \item
    Listas de distribución de r-project.org: \url{https://stat.ethz.ch/mailman/listinfo}
  \item
    Búsqueda en R-help: \url{http://r.789695.n4.nabble.com/R-help-f789696.html}
  \item
    Búsqueda en R-help-es: \url{https://r-help-es.r-project.narkive.com}

    \url{https://grokbase.com/g/r/r-help-es}
  \item
    Archivos de R-help-es: \url{https://stat.ethz.ch/pipermail/r-help-es}
  \end{itemize}
\end{itemize}

\section{RStudio}\label{rstudio-links}

\href{https://www.rstudio.com}{\textbf{\emph{RStudio}}}:

\begin{itemize}
\item
  \href{https://www.rstudio.com/online-learning}{Online learning}
\item
  \href{https://resources.rstudio.com/webinars}{Webinars}
\item
  \href{https://spark.rstudio.com/}{sparklyr}
\item
  \href{http://shiny.rstudio.com}{shiny}
\end{itemize}

\href{https://www.tidyverse.org/}{\textbf{\emph{tidyverse}}}:

\begin{itemize}
\item
  \href{https://dplyr.tidyverse.org}{dplyr}
\item
  \href{https://tibble.tidyverse.org}{tibble}
\item
  \href{https://tidyr.tidyverse.org}{tidyr}
\item
  \href{https://stringr.tidyverse.org}{stringr}
\item
  \href{https://readr.tidyverse.org}{readr}
\item
  \href{https://db.rstudio.com}{Databases using R},
  \href{https://db.rstudio.com/overview}{dplyr as a database interface},
  \href{https://db.rstudio.com/dplyr}{Databases using dplyr}
\end{itemize}

\href{https://resources.rstudio.com/rstudio-cheatsheets}{\textbf{\emph{CheatSheets}}}:

\begin{itemize}
\item
  \href{https://resources.rstudio.com/rstudio-cheatsheets/rmarkdown-2-0-cheat-sheet}{rmarkdown}
\item
  \href{https://resources.rstudio.com/rstudio-cheatsheets/shiny-cheat-sheet}{shiny}
\item
  \href{https://github.com/rstudio/cheatsheets/blob/master/data-transformation.pdf}{dplyr}
\item
  \href{https://github.com/rstudio/cheatsheets/blob/master/data-import.pdf}{tidyr}
\item
  \href{https://resources.rstudio.com/rstudio-cheatsheets/stringr-cheat-sheet}{stringr}
\end{itemize}

\section{Bibliometría}\label{bibliom-links}

\begin{itemize}
\item
  \href{https://cran.r-project.org/web/packages/CITAN/index.html}{CITAN}
\item
  \href{https://rubenfcasal.github.io/scimetr/index.html}{scimetr}
\item
  \href{http://www.bibliometrix.org}{bibliometrix}
\item
  \href{https://vt-arc.github.io/wosr/index.html}{wosr}
\item
  \href{https://github.com/juba/rwos}{rwos}
\item
  \href{https://docs.ropensci.org/rcrossref}{rcrossref}
\item
  \href{https://ropensci.org/packages/}{ropensci}: Literature
\item
  \href{https://cran.r-project.org/web/packages/Diderot/index.html}{Diderot}
\item
  \ldots{}
\end{itemize}

\chapter{Instalación de R}\label{instalaciuxf3n-de-r}

En la web del proyecto R
(\href{http://www.r-project.org}{www.r-project.org}) está disponible
mucha información sobre este entorno estadístico.

\begin{longtable}[]{@{}cc@{}}
\toprule\noalign{}
\endhead
\bottomrule\noalign{}
\endlastfoot
\includegraphics[width=0.68\linewidth,height=\textheight,keepaspectratio]{images/rproject.png} & \includegraphics[width=0.63\linewidth,height=\textheight,keepaspectratio]{images/cran.png} \\
\href{https://r-project.org}{R-project} & \href{https://cran.r-project.org}{CRAN} \\
\end{longtable}

Las descargas se realizan a través de la web del CRAN (The Comprehensive
R Archive Network), con múltiples mirrors:

\begin{itemize}
\tightlist
\item
  \emph{Oficina de software libre} (CIXUG) \href{http://ftp.cixug.es/CRAN/}{ftp.cixug.es/CRAN}.
\item
  \emph{Spanish National Research Network (Madrid)} (RedIRIS) es
  \href{http://cran.es.r-project.org/}{cran.es.r-project.org}.
\end{itemize}

\section{Instalación de R en Windows}\label{instalaciuxf3n-de-r-en-windows}

Seleccionando \href{http://ftp.cixug.es/CRAN/bin/windows/}{Download R for Windows} y posteriormente
\href{http://ftp.cixug.es/CRAN/bin/windows/base/}{base} accedemos
al enlace con el instalador de R para Windows.

\includegraphics[width=0.4\linewidth,height=\textheight,keepaspectratio]{images/R351.png}

\subsection{Asistente de instalación}\label{asistente-de-instalaciuxf3n}

Durante el proceso de instalación la recomendación (para evitar posibles problemas) es seleccionar ventanas simples SDI en lugar de múltiples ventanas MDI (hay que \emph{utilizar opciones de configuración}).

\begin{longtable}[]{@{}cc@{}}
\toprule\noalign{}
\endhead
\bottomrule\noalign{}
\endlastfoot
\includegraphics[width=0.8\linewidth,height=\textheight,keepaspectratio]{images/image3.png} & \includegraphics[width=0.8\linewidth,height=\textheight,keepaspectratio]{images/image4.png} \\
\includegraphics[width=0.8\linewidth,height=\textheight,keepaspectratio]{images/image5.png} & \includegraphics[width=0.8\linewidth,height=\textheight,keepaspectratio]{images/image6.png} \\
\end{longtable}

Una vez terminada la instalación, al abrir R aparece la ventana de la consola (simula una ventana de comandos de Unix) que permite ejecutar comandos de R.

\subsection{Instalación de paquetes}\label{instalaciuxf3n-de-paquetes}

Después de la instalación de R, puede ser necesario instalar paquetes adicionales (puede ser recomendable ejecutar R \emph{como Administrador} para evitar problemas de permiso de escritura en la carpeta \emph{library}\footnote{Alternativamente se podrían proporcionar a los usuarios del equipo el permiso \emph{control total} en la carpeta de instalación de R.}).

Para ejecutar los ejemplos mostrados en el libro será necesario tener instalados los siguientes paquetes:
\href{https://dplyr.tidyverse.org}{\texttt{dplyr}} (colección \href{https://www.tidyverse.org/}{\texttt{tidyverse}}),
\href{https://tidyr.tidyverse.org}{\texttt{tidyr}},
\href{https://stringr.tidyverse.org}{\texttt{stringr}},
\href{https://readxl.tidyverse.org}{\texttt{readxl}} ,
\href{https://cran.r-project.org/web/packages/openxlsx/index.html}{\texttt{openxlsx}}, \href{https://cran.r-project.org/web/packages/RODBC/index.html}{\texttt{RODBC}},
\href{https://cran.r-project.org/web/packages/sqldf/index.html}{\texttt{sqldf}},
\href{https://r-dbi.github.io/RSQLite}{\texttt{RSQLite}},
\href{https://cran.r-project.org/web/packages/foreign/index.html}{\texttt{foreign}},
\href{https://cran.r-project.org/web/packages/magrittr/index.html}{\texttt{magrittr}},
\href{https://rattle.togaware.com}{\texttt{rattle}},
\href{https://yihui.name/knitr}{\texttt{knitr}}
Por ejemplo mediante los comandos:

\begin{Shaded}
\begin{Highlighting}[]
\NormalTok{pkgs }\OtherTok{\textless{}{-}} \FunctionTok{c}\NormalTok{(}\StringTok{\textquotesingle{}dplyr\textquotesingle{}}\NormalTok{, }\StringTok{\textquotesingle{}tidyr\textquotesingle{}}\NormalTok{, }\StringTok{\textquotesingle{}stringr\textquotesingle{}}\NormalTok{, }\StringTok{\textquotesingle{}readxl\textquotesingle{}}\NormalTok{, }\StringTok{\textquotesingle{}openxlsx\textquotesingle{}}\NormalTok{, }\StringTok{\textquotesingle{}magrittr\textquotesingle{}}\NormalTok{, }
          \StringTok{\textquotesingle{}RODBC\textquotesingle{}}\NormalTok{, }\StringTok{\textquotesingle{}sqldf\textquotesingle{}}\NormalTok{, }\StringTok{\textquotesingle{}RSQLite\textquotesingle{}}\NormalTok{, }\StringTok{\textquotesingle{}foreign\textquotesingle{}}\NormalTok{, }\StringTok{\textquotesingle{}rattle\textquotesingle{}}\NormalTok{, }\StringTok{\textquotesingle{}knitr\textquotesingle{}}\NormalTok{)}
\CommentTok{\# install.packages(pkgs, dependencies=TRUE)}
\FunctionTok{install.packages}\NormalTok{(}\FunctionTok{setdiff}\NormalTok{(pkgs, }\FunctionTok{installed.packages}\NormalTok{()[,}\StringTok{\textquotesingle{}Package\textquotesingle{}}\NormalTok{]), }\AttributeTok{dependencies =} \ConstantTok{TRUE}\NormalTok{)}
\end{Highlighting}
\end{Shaded}

(puede que haya que seleccionar el repositorio de descarga, e.g.~\emph{Spain (Madrid)}).

La forma tradicional es esta:

\begin{enumerate}
\def\labelenumi{\arabic{enumi}.}
\item
  Se inicia R y se selecciona \emph{Paquetes -\textgreater{} Instalar paquetes}
\item
  Se selecciona el repositorio.
\item
  Se selecciona el paquete y automáticamente se instala.
\end{enumerate}

\texttt{Rattle} depende de la libraría gráfica GTK+, al iniciarlo por primera vez
con el comando \texttt{library(rattle)} nos pregunta si queremos instalarla:

\includegraphics[width=0.15\linewidth,height=\textheight,keepaspectratio]{images/image7.png}

Pulsamos OK y reiniciamos R.

\begin{center}\rule{0.5\linewidth}{0.5pt}\end{center}

\section{Instalación en Mac OS X}\label{instalaciuxf3n-en-mac-os-x}

Instalar R de
\url{http://cran.es.r‐project.org/bin/macosx}
siguiendo los pasos anteriores.

Para instalar \texttt{rattle} seguir estos pasos (\url{https://zhiyzuo.github.io/installation-rattle}):

\begin{enumerate}
\def\labelenumi{\arabic{enumi}.}
\item
  Instalar Homebrew:
  \url{https://brew.sh/}.
\item
  Ejecutar el siguiente código en la consola:

\begin{Shaded}
\begin{Highlighting}[]
\FunctionTok{system}\NormalTok{(}\StringTok{\textquotesingle{}brew install gtk+\textquotesingle{}}\NormalTok{)}

\FunctionTok{local}\NormalTok{(\{}
  \ControlFlowTok{if}\NormalTok{ (}\FunctionTok{Sys.info}\NormalTok{()[[}\StringTok{\textquotesingle{}sysname\textquotesingle{}}\NormalTok{]] }\SpecialCharTok{!=} \StringTok{\textquotesingle{}Darwin\textquotesingle{}}\NormalTok{) }\FunctionTok{return}\NormalTok{()}

\NormalTok{  .Platform}\SpecialCharTok{$}\NormalTok{pkgType }\OtherTok{=} \StringTok{\textquotesingle{}mac.binary.el{-}capitan\textquotesingle{}}
  \FunctionTok{unlockBinding}\NormalTok{(}\StringTok{\textquotesingle{}.Platform\textquotesingle{}}\NormalTok{, }\FunctionTok{baseenv}\NormalTok{())}
  \FunctionTok{assign}\NormalTok{(}\StringTok{\textquotesingle{}.Platform\textquotesingle{}}\NormalTok{, .Platform, }\StringTok{\textquotesingle{}package:base\textquotesingle{}}\NormalTok{)}
  \FunctionTok{lockBinding}\NormalTok{(}\StringTok{\textquotesingle{}.Platform\textquotesingle{}}\NormalTok{, }\FunctionTok{baseenv}\NormalTok{())}

  \FunctionTok{options}\NormalTok{(}
    \AttributeTok{pkgType =} \StringTok{\textquotesingle{}both\textquotesingle{}}\NormalTok{, }
    \AttributeTok{install.packages.compile.from.source =} \StringTok{\textquotesingle{}always\textquotesingle{}}\NormalTok{,}
    \AttributeTok{repos =} \StringTok{\textquotesingle{}https://macos.rbind.org\textquotesingle{}}
\NormalTok{  )}
\NormalTok{\})}

\FunctionTok{install.packages}\NormalTok{(}\FunctionTok{c}\NormalTok{(}\StringTok{\textquotesingle{}RGtk2\textquotesingle{}}\NormalTok{, }\StringTok{\textquotesingle{}cairoDevice\textquotesingle{}}\NormalTok{, }\StringTok{\textquotesingle{}rattle\textquotesingle{}}\NormalTok{))}
\end{Highlighting}
\end{Shaded}
\end{enumerate}

\begin{center}\rule{0.5\linewidth}{0.5pt}\end{center}

\section{Instalación (opcional) de un entorno o editor de comandos}\label{instalaciuxf3n-opcional-de-un-entorno-o-editor-de-comandos}

Aunque la consola de R dispone de un editor básico de códido (script),
puede ser recomendable trabajar con un editor de comandos más cómodo y
flexible.

Un entorno de R muy recomendable es el \textbf{RStudio},
\href{http://rstudio.org}{\emph{http://rstudio.org}}:

\includegraphics[width=0.6\linewidth,height=\textheight,keepaspectratio]{images/image8.png}

Para instalarlo descargar el archivo de instalación de
\href{http://rstudio.org/download/desktop}{\emph{http://rstudio.org/download/desktop}}.

\subsection{Opciones adicionales}\label{opciones-adicionales}

Nos puede interesar modificar las opciones por defecto en RStudio, por ejemplo que los gráficos se muestren en una ventana de R o que se emplee el navegador por defecto, para ello habría que modificar (con permisos de administrador) los archivos de configuración \emph{Tools.R} y \emph{Options.R}
(en Windows se encuentran en la carpeta \emph{C:\textbackslash Program Files\textbackslash RStudio\textbackslash R}).

Para utilizar el dispositivo gráfico de R, modificar \emph{Tools.R}:

\begin{Shaded}
\begin{Highlighting}[]
\CommentTok{\# set our graphics device as the default and cause it to be created/set}
\FunctionTok{.rs.addFunction}\NormalTok{( }\StringTok{"initGraphicsDevice"}\NormalTok{, }\ControlFlowTok{function}\NormalTok{()}
\NormalTok{\{}
   \CommentTok{\# options(device="RStudioGD")}
   \CommentTok{\# grDevices::deviceIsInteractive("RStudioGD")}
\NormalTok{  grDevices}\SpecialCharTok{::}\FunctionTok{deviceIsInteractive}\NormalTok{()}
\NormalTok{\})}
\end{Highlighting}
\end{Shaded}

Para utilizar el navegador del equipo en lugar del visor integrado de de R, modificar \emph{Options.R}:

\begin{Shaded}
\begin{Highlighting}[]
\CommentTok{\# \# custom browseURL implementation}
\CommentTok{\# options(browser = function(url)}
\CommentTok{\# \{}
\CommentTok{\#    .Call("rs\_browseURL", url) ;}
\CommentTok{\# \})}
\end{Highlighting}
\end{Shaded}


\bibliography{book.bib,packages.bib}

\end{document}
